# 以斯帖记
<!-- TOC -->

- [以斯帖记](#以斯帖记)
    - [以斯帖记第1章](#以斯帖记第1章)
    - [以斯帖记第2章](#以斯帖记第2章)
    - [以斯帖记第3章](#以斯帖记第3章)
    - [以斯帖记第4章](#以斯帖记第4章)
    - [以斯帖记第5章](#以斯帖记第5章)
    - [以斯帖记第6章](#以斯帖记第6章)
    - [以斯帖记第7章](#以斯帖记第7章)
    - [以斯帖记第8章](#以斯帖记第8章)
    - [以斯帖记第9章](#以斯帖记第9章)
    - [以斯帖记第10章](#以斯帖记第10章)

<!-- /TOC -->
## 以斯帖记第1章
##### 斯1:1
> This is what happened during the time of Xerxes, the Xerxes who ruled over 127 provinces stretching from India to Cush:
>
> 亚哈随鲁作王,从印度直到古实,统管一百二十七省。


##### 斯1:2
> At that time King Xerxes reigned from his royal throne in the citadel of Susa,
>
> 亚哈随鲁王在书珊城的宫登基。


##### 斯1:3
> and in the third year of his reign he gave a banquet for all his nobles and officials. The military leaders of Persia and Media, the princes, and the nobles of the provinces were present.
>
> 在位第三年,为他一切首领、臣仆设摆筵席,有波斯和玛代的权贵,就是各省的贵胄与首领,在他面前。


##### 斯1:4
> For a full 180 days he displayed the vast wealth of his kingdom and the splendor and glory of his majesty.
>
> 他把他荣耀之国的丰富和他美好威严的尊贵,给他们看了许多日,就是一百八十日。


##### 斯1:5
> When these days were over, the king gave a banquet, lasting seven days, in the enclosed garden of the king's palace, for all the people from the least to the greatest, who were in the citadel of Susa.
>
> 这日子满了,又为所有住书珊城的大小人民,在御园的院子里设摆筵席七日。


##### 斯1:6
> The garden had hangings of white and blue linen, fastened with cords of white linen and purple material to silver rings on marble pillars. There were couches of gold and silver on a mosaic pavement of porphyry, marble, mother-of-pearl and other costly stones.
>
> 有白色、绿色、蓝色的帐子,用细麻绳、紫色绳从银环内系在白玉石柱上,有金银的床榻摆在红、白、黄、黑玉石的铺石地上。


##### 斯1:7
> Wine was served in goblets of gold, each one different from the other, and the royal wine was abundant, in keeping with the king's liberality.
>
> 用金器皿赐酒,器皿各有不同。御酒甚多,足显王的厚意。


##### 斯1:8
> By the king's command each guest was allowed to drink in his own way, for the king instructed all the wine stewards to serve each man what he wished.
>
> 喝酒有例,不准勉强人,因王吩咐宫里的一切臣宰,让人各随己意。


##### 斯1:9
> Queen Vashti also gave a banquet for the women in the royal palace of King Xerxes.
>
> 王后瓦实提在亚哈随鲁王的宫内,也为妇女设摆筵席。


##### 斯1:10
> On the seventh day, when King Xerxes was in high spirits from wine, he commanded the seven eunuchs who served him--Mehuman, Biztha, Harbona, Bigtha, Abagtha, Zethar and Carcas--
>
> 第七日,亚哈随鲁王饮酒,心中快乐,就吩咐在他面前侍立的七个太监米户幔、比斯他、哈波拿、比革他、亚拔他、西达、甲迦,


##### 斯1:11
> to bring before him Queen Vashti, wearing her royal crown, in order to display her beauty to the people and nobles, for she was lovely to look at.
>
> 请王后瓦实提头戴王后的冠冕到王面前,使各等臣民看她的美貌,因为她容貌甚美。


##### 斯1:12
> But when the attendants delivered the king's command, Queen Vashti refused to come. Then the king became furious and burned with anger.
>
> 王后瓦实提却不肯遵太监所传的王命而来,所以王甚发怒,心如火烧。


##### 斯1:13
> Since it was customary for the king to consult experts in matters of law and justice, he spoke with the wise men who understood the times
>
> 那时,在王左右常见王面,国中坐高位的,有波斯和玛代的七个大臣,就是甲示拿,示达,押玛他,他施斯,米力,玛西拿,米母干,


##### 斯1:14
> and were closest to the king--Carshena, Shethar, Admatha, Tarshish, Meres, Marsena and Memucan, the seven nobles of Persia and Media who had special access to the king and were highest in the kingdom.
>
> 都是达时务的明哲人。按王的常规,办事必先询问知例明法的人。王问他们说,


##### 斯1:15
> "According to law, what must be done to Queen Vashti?" he asked. "She has not obeyed the command of King Xerxes that the eunuchs have taken to her."
>
> “王后瓦实提不遵太监所传的王命,照例应当怎样办理呢?”


##### 斯1:16
> Then Memucan replied in the presence of the king and the nobles, "Queen Vashti has done wrong, not only against the king but also against all the nobles and the peoples of all the provinces of King Xerxes.
>
> 米母干在王和众首领面前回答说:“王后瓦实提这事不但得罪王,并且有害于王各省的臣民。


##### 斯1:17
> For the queen's conduct will become known to all the women, and so they will despise their husbands and say, 'King Xerxes commanded Queen Vashti to be brought before him, but she would not come.'
>
> 因为王后这事必传到众妇人的耳中,说亚哈随鲁王吩咐王后瓦实提到王面前,她却不来。她们就藐视自己的丈夫。


##### 斯1:18
> This very day the Persian and Median women of the nobility who have heard about the queen's conduct will respond to all the king's nobles in the same way. There will be no end of disrespect and discord.
>
> 今日,波斯和玛代的众夫人听见王后这事,必向王的大臣照样行,从此必大开藐视和忿怒之端。


##### 斯1:19
> "Therefore, if it pleases the king, let him issue a royal decree and let it be written in the laws of Persia and Media, which cannot be repealed, that Vashti is never again to enter the presence of King Xerxes. Also let the king give her royal position to someone else who is better than she.
>
> 王若以为美,就降旨写在波斯和玛代人的例中,永不更改,不准瓦实提再到王面前,将她王后的位分赐给比她还好的人。


##### 斯1:20
> Then when the king's edict is proclaimed throughout all his vast realm, all the women will respect their husbands, from the least to the greatest."
>
> 所降的旨意传遍通国(国度本来广大),所有的妇人,无论丈夫贵贱都必尊敬他。”


##### 斯1:21
> The king and his nobles were pleased with this advice, so the king did as Memucan proposed.
>
> 王和众首领都以米母干的话为美,王就照这话去行。


##### 斯1:22
> He sent dispatches to all parts of the kingdom, to each province in its own script and to each people in its own language, proclaiming in each people's tongue that every man should be ruler over his own household.
>
> 发诏书,用各省的文字、各族的方言通知各省,使为丈夫的在家中作主,各说本地的方言。


## 以斯帖记第2章
##### 斯2:1
> Later when the anger of King Xerxes had subsided, he remembered Vashti and what she had done and what he had decreed about her.
>
> 这事以后,亚哈随鲁王的忿怒止息,就想念瓦实提和她所行的,并怎样降旨办她。


##### 斯2:2
> Then the king's personal attendants proposed, "Let a search be made for beautiful young virgins for the king.
>
> 于是王的侍臣对王说:“不如为王寻找美貌的处女。


##### 斯2:3
> Let the king appoint commissioners in every province of his realm to bring all these beautiful girls into the harem at the citadel of Susa. Let them be placed under the care of Hegai, the king's eunuch, who is in charge of the women; and let beauty treatments be given to them.
>
> 王可以派官在国中的各省,招聚美貌的处女到书珊城(或作“宫”)的女院,交给掌管女子的太监希该,给她们当用的香品。


##### 斯2:4
> Then let the girl who pleases the king be queen instead of Vashti." This advice appealed to the king, and he followed it.
>
> 王所喜爱的女子可以立为王后,代替瓦实提。”王以这事为美,就如此行。


##### 斯2:5
> Now there was in the citadel of Susa a Jew of the tribe of Benjamin, named Mordecai son of Jair, the son of Shimei, the son of Kish,
>
> 书珊城有一个犹大人,名叫末底改,是便雅悯人基士的曾孙、示每的孙子、睚珥的儿子。


##### 斯2:6
> who had been carried into exile from Jerusalem by Nebuchadnezzar king of Babylon, among those taken captive with Jehoiachin king of Judah.
>
> 从前巴比伦王尼布甲尼撒将犹大王耶哥尼雅(又名“约雅斤”)和百姓从耶路撒冷掳去,末底改也在其内。


##### 斯2:7
> Mordecai had a cousin named Hadassah, whom he had brought up because she had neither father nor mother. This girl, who was also known as Esther, was lovely in form and features, and Mordecai had taken her as his own daughter when her father and mother died.
>
> 末底改抚养他叔叔的女儿哈大沙(后名以斯帖),因为她没有父母。这女子又容貌俊美,她父母死了,末底改就收她为自己的女儿。


##### 斯2:8
> When the king's order and edict had been proclaimed, many girls were brought to the citadel of Susa and put under the care of Hegai. Esther also was taken to the king's palace and entrusted to Hegai, who had charge of the harem.
>
> 王的谕旨传出,就招聚许多女子到书珊城,交给掌管女子的希该;以斯帖也送入王宫,交付希该。


##### 斯2:9
> The girl pleased him and won his favor. Immediately he provided her with her beauty treatments and special food. He assigned to her seven maids selected from the king's palace and moved her and her maids into the best place in the harem.
>
> 希该喜悦以斯帖,就恩待她,急忙给她需用的香品和她所当得的份,又派所当得的七个宫女服事她,使她和她的宫女搬入女院上好的房屋。


##### 斯2:10
> Esther had not revealed her nationality and family background, because Mordecai had forbidden her to do so.
>
> 以斯帖未曾将籍贯宗族告诉人,因为末底改嘱咐她不可叫人知道。


##### 斯2:11
> Every day he walked back and forth near the courtyard of the harem to find out how Esther was and what was happening to her.
>
> 末底改天天在女院前边行走,要知道以斯帖平安不平安,并后事如何。


##### 斯2:12
> Before a girl's turn came to go in to King Xerxes, she had to complete twelve months of beauty treatments prescribed for the women, six months with oil of myrrh and six with perfumes and cosmetics.
>
> 众女子照例先洁净身体十二个月:六个月用没药油,六个月用香料和洁身之物。满了日期,然后挨次进去见亚哈随鲁王。


##### 斯2:13
> And this is how she would go to the king: Anything she wanted was given her to take with her from the harem to the king's palace.
>
> 女子进去见王是这样:从女院到王宫的时候,凡她所要的都必给她。


##### 斯2:14
> In the evening she would go there and in the morning return to another part of the harem to the care of Shaashgaz, the king's eunuch who was in charge of the concubines. She would not return to the king unless he was pleased with her and summoned her by name.
>
> 晚上进去,次日回到女子第二院,交给掌管妃嫔的太监沙甲;除非王喜爱她,再提名召她,就不再进去见王。


##### 斯2:15
> When the turn came for Esther (the girl Mordecai had adopted, the daughter of his uncle Abihail) to go to the king, she asked for nothing other than what Hegai, the king's eunuch who was in charge of the harem, suggested. And Esther won the favor of everyone who saw her.
>
> 末底改叔叔亚比孩的女儿,就是末底改收为自己女儿的以斯帖,按次序当进去见王的时候,除了掌管女子的太监希该所派定给她的,她别无所求。凡看见以斯帖的都喜悦她。


##### 斯2:16
> She was taken to King Xerxes in the royal residence in the tenth month, the month of Tebeth, in the seventh year of his reign.
>
> 亚哈随鲁王第七年十月,就是提别月,以斯帖被引入宫见王。


##### 斯2:17
> Now the king was attracted to Esther more than to any of the other women, and she won his favor and approval more than any of the other virgins. So he set a royal crown on her head and made her queen instead of Vashti.
>
> 王爱以斯帖过于爱众女,她在王眼前蒙宠爱比众处女更甚。王就把王后的冠冕戴在她头上,立她为王后,代替瓦实提。


##### 斯2:18
> And the king gave a great banquet, Esther's banquet, for all his nobles and officials. He proclaimed a holiday throughout the provinces and distributed gifts with royal liberality.
>
> 王因以斯帖的缘故给众首领和臣仆设摆大筵席,又豁免各省的租税,并照王的厚意大颁赏赐。


##### 斯2:19
> When the virgins were assembled a second time, Mordecai was sitting at the king's gate.
>
> 第二次招聚处女的时候,末底改坐在朝门。


##### 斯2:20
> But Esther had kept secret her family background and nationality just as Mordecai had told her to do, for she continued to follow Mordecai's instructions as she had done when he was bringing her up.
>
> 以斯帖照着末底改所嘱咐的,还没有将籍贯宗族告诉人,因为以斯帖遵末底改的命,如抚养她的时候一样。


##### 斯2:21
> During the time Mordecai was sitting at the king's gate, Bigthana and Teresh, two of the king's officers who guarded the doorway, became angry and conspired to assassinate King Xerxes.
>
> 当那时候,末底改坐在朝门,王的太监中有两个守门的辟探和提列,恼恨亚哈随鲁王,想要下手害他。


##### 斯2:22
> But Mordecai found out about the plot and told Queen Esther, who in turn reported it to the king, giving credit to Mordecai.
>
> 末底改知道了,就告诉王后以斯帖。以斯帖奉末底改的名,报告于王。


##### 斯2:23
> And when the report was investigated and found to be true, the two officials were hanged on a gallows. All this was recorded in the book of the annals in the presence of the king.
>
> 究察这事,果然是实,就把二人挂在木头上,将这事在王面前写于历史上。


## 以斯帖记第3章
##### 斯3:1
> After these events, King Xerxes honored Haman son of Hammedatha, the Agagite, elevating him and giving him a seat of honor higher than that of all the other nobles.
>
> 这事以后,亚哈随鲁王抬举亚甲族哈米大他的儿子哈曼,使他高升,叫他的爵位超过与他同事的一切臣宰。


##### 斯3:2
> All the royal officials at the king's gate knelt down and paid honor to Haman, for the king had commanded this concerning him. But Mordecai would not kneel down or pay him honor.
>
> 在朝门的一切臣仆,都跪拜哈曼,因为王如此吩咐;惟独末底改不跪不拜。


##### 斯3:3
> Then the royal officials at the king's gate asked Mordecai, "Why do you disobey the king's command?"
>
> 在朝门的臣仆问末底改说:“你为何违背王的命令呢?”


##### 斯3:4
> Day after day they spoke to him but he refused to comply. Therefore they told Haman about it to see whether Mordecai's behavior would be tolerated, for he had told them he was a Jew.
>
> 他们天天劝他,他还是不听。他们就告诉哈曼,要看末底改的事站得住站不住,因他已经告诉他们自己是犹大人。


##### 斯3:5
> When Haman saw that Mordecai would not kneel down or pay him honor, he was enraged.
>
> 哈曼见末底改不跪不拜,他就怒气填胸。


##### 斯3:6
> Yet having learned who Mordecai's people were, he scorned the idea of killing only Mordecai. Instead Haman looked for a way to destroy all Mordecai's people, the Jews, throughout the whole kingdom of Xerxes.
>
> 他们已将末底改的本族告诉哈曼。他以为下手害末底改一人是小事,就要灭绝亚哈随鲁王通国所有的犹大人,就是末底改的本族。


##### 斯3:7
> In the twelfth year of King Xerxes, in the first month, the month of Nisan, they cast the pur  (that is, the lot) in the presence of Haman to select a day and month. And the lot fell on the twelfth month, the month of Adar.
>
> 亚哈随鲁王十二年正月,就是尼散月,人在哈曼面前,按日日月月掣普珥,就是掣签,要定何月何日为吉,择定了十二月,就是亚达月。


##### 斯3:8
> Then Haman said to King Xerxes, "There is a certain people dispersed and scattered among the peoples in all the provinces of your kingdom whose customs are different from those of all other people and who do not obey the king's laws; it is not in the king's best interest to tolerate them.
>
> 哈曼对亚哈随鲁王说:“有一种民,散居在王国各省的民中。他们的律例与万民的律例不同,也不守王的律例,所以容留他们与王无益。


##### 斯3:9
> If it pleases the king, let a decree be issued to destroy them, and I will put ten thousand talents of silver into the royal treasury for the men who carry out this business."
>
> 王若以为美,请下旨意灭绝他们;我就捐一万他连得银子,交给掌管国帑的人,纳入王的府库。”


##### 斯3:10
> So the king took his signet ring from his finger and gave it to Haman son of Hammedatha, the Agagite, the enemy of the Jews.
>
> 于是,王从自己手上摘下戒指,给犹大人的仇敌、亚甲族哈米大他的儿子哈曼。


##### 斯3:11
> "Keep the money," the king said to Haman, "and do with the people as you please."
>
> 王对哈曼说:“这银子仍赐给你,这民也交给你,你可以随意待他们。”


##### 斯3:12
> Then on the thirteenth day of the first month the royal secretaries were summoned. They wrote out in the script of each province and in the language of each people all Haman's orders to the king's satraps, the governors of the various provinces and the nobles of the various peoples. These were written in the name of King Xerxes himself and sealed with his own ring.
>
> 正月十三日,就召了王的书记来,照着哈曼一切所吩咐的,用各省的文字、各族的方言,奉亚哈随鲁王的名写旨意,传与总督和各省的省长,并各族的首领。又用王的戒指盖印,


##### 斯3:13
> Dispatches were sent by couriers to all the king's provinces with the order to destroy, kill and annihilate all the Jews--young and old, women and little children--on a single day, the thirteenth day of the twelfth month, the month of Adar, and to plunder their goods.
>
> 交给驿卒传到王的各省,吩咐将犹大人,无论老少妇女孩子,在一日之间,十二月,就是亚达月十三日,全然剪除,杀戮灭绝,并夺他们的财为掠物。


##### 斯3:14
> A copy of the text of the edict was to be issued as law in every province and made known to the people of every nationality so they would be ready for that day.
>
> 抄录这旨意,颁行各省,宣告各族,使他们预备等候那日。


##### 斯3:15
> Spurred on by the king's command, the couriers went out, and the edict was issued in the citadel of Susa. The king and Haman sat down to drink, but the city of Susa was bewildered.
>
> 驿卒奉王命急忙起行,旨意也传遍书珊城。王同哈曼坐下饮酒,书珊城的民,却都慌乱。


## 以斯帖记第4章
##### 斯4:1
> When Mordecai learned of all that had been done, he tore his clothes, put on sackcloth and ashes, and went out into the city, wailing loudly and bitterly.
>
> 末底改知道所做的这一切事,就撕裂衣服,穿麻衣,蒙灰尘,在城中行走,痛哭哀号。


##### 斯4:2
> But he went only as far as the king's gate, because no one clothed in sackcloth was allowed to enter it.
>
> 到了朝门前停住脚步,因为穿麻衣的不可进朝门。


##### 斯4:3
> In every province to which the edict and order of the king came, there was great mourning among the Jews, with fasting, weeping and wailing. Many lay in sackcloth and ashes.
>
> 王的谕旨所到的各省各处,犹大人大大悲哀,禁食哭泣哀号,穿麻衣躺在灰中的甚多。


##### 斯4:4
> When Esther's maids and eunuchs came and told her about Mordecai, she was in great distress. She sent clothes for him to put on instead of his sackcloth, but he would not accept them.
>
> 王后以斯帖的宫女和太监来把这事告诉以斯帖,她甚是忧愁,就送衣服给末底改穿,要他脱下麻衣,他却不受。


##### 斯4:5
> Then Esther summoned Hathach, one of the king's eunuchs assigned to attend her, and ordered him to find out what was troubling Mordecai and why.
>
> 以斯帖就把王所派伺候她的一个太监,名叫哈他革召来,吩咐他去见末底改,要知道这是什么事,是什么缘故。


##### 斯4:6
> So Hathach went out to Mordecai in the open square of the city in front of the king's gate.
>
> 于是,哈他革出到朝门前的宽阔处见末底改。


##### 斯4:7
> Mordecai told him everything that had happened to him, including the exact amount of money Haman had promised to pay into the royal treasury for the destruction of the Jews.
>
> 末底改将自己所遇的事,并哈曼为灭绝犹大人,应许捐入王库的银数,都告诉了他。


##### 斯4:8
> He also gave him a copy of the text of the edict for their annihilation, which had been published in Susa, to show to Esther and explain it to her, and he told him to urge her to go into the king's presence to beg for mercy and plead with him for her people.
>
> 又将所抄写传遍书珊城要灭绝犹大人的旨意交给哈他革,要给以斯帖看,又要给她说明,并嘱咐她进去见王,为本族的人在王面前恳切祈求。


##### 斯4:9
> Hathach went back and reported to Esther what Mordecai had said.
>
> 哈他革回来,将末底改的话告诉以斯帖。


##### 斯4:10
> Then she instructed him to say to Mordecai,
>
> 以斯帖就吩咐哈他革去见末底改说:“


##### 斯4:11
> "All the king's officials and the people of the royal provinces know that for any man or woman who approaches the king in the inner court without being summoned the king has but one law: that he be put to death. The only exception to this is for the king to extend the gold scepter to him and spare his life. But thirty days have passed since I was called to go to the king."
>
> 王的一切臣仆和各省的人民,都知道有一个定例:若不蒙召,擅入内院见王的,无论男女必被治死;除非王向他伸出金杖,不得存活。现在我没有蒙召进去见王已经三十日了。”


##### 斯4:12
> When Esther's words were reported to Mordecai,
>
> 人就把以斯帖这话告诉末底改。


##### 斯4:13
> he sent back this answer: "Do not think that because you are in the king's house you alone of all the Jews will escape.
>
> 末底改托人回覆以斯帖说:“你莫想在王宫里强过一切犹大人,得免这祸。


##### 斯4:14
> For if you remain silent at this time, relief and deliverance for the Jews will arise from another place, but you and your father's family will perish. And who knows but that you have come to royal position for such a time as this?"
>
> 此时你若闭口不言,犹大人必从别处得解脱、蒙拯救;你和你父家必至灭亡。焉知你得了王后的位分,不是为现今的机会吗?”


##### 斯4:15
> Then Esther sent this reply to Mordecai:
>
> 以斯帖就吩咐人回报末底改说:


##### 斯4:16
> "Go, gather together all the Jews who are in Susa, and fast for me. Do not eat or drink for three days, night or day. I and my maids will fast as you do. When this is done, I will go to the king, even though it is against the law. And if I perish, I perish."
>
> “你当去招聚书珊城所有的犹大人,为我禁食三昼三夜,不吃不喝;我和我的宫女也要这样禁食。然后我违例进去见王,我若死就死吧!”


##### 斯4:17
> So Mordecai went away and carried out all of Esther's instructions.
>
> 于是,末底改照以斯帖一切所吩咐的去行。


## 以斯帖记第5章
##### 斯5:1
> On the third day Esther put on her royal robes and stood in the inner court of the palace, in front of the king's hall. The king was sitting on his royal throne in the hall, facing the entrance.
>
> 第三日,以斯帖穿上朝服,进王宫的内院,对殿站立。王在殿里坐在宝座上,对着殿门。


##### 斯5:2
> When he saw Queen Esther standing in the court, he was pleased with her and held out to her the gold scepter that was in his hand. So Esther approached and touched the tip of the scepter.
>
> 王见王后以斯帖站在院内,就施恩于她,向她伸出手中的金杖;以斯帖便向前摸杖头。


##### 斯5:3
> Then the king asked, "What is it, Queen Esther? What is your request? Even up to half the kingdom, it will be given you."
>
> 王对她说:“王后以斯帖啊,你要什么?你求什么?就是国的一半也必赐给你。”


##### 斯5:4
> "If it pleases the king," replied Esther, "let the king, together with Haman, come today to a banquet I have prepared for him."
>
> 以斯帖说:“王若以为美,就请王带着哈曼今日赴我所预备的筵席。”


##### 斯5:5
> "Bring Haman at once," the king said, "so that we may do what Esther asks." So the king and Haman went to the banquet Esther had prepared.
>
> 王说:“叫哈曼速速照以斯帖的话去行。”于是,王带着哈曼赴以斯帖所预备的筵席。


##### 斯5:6
> As they were drinking wine, the king again asked Esther, "Now what is your petition? It will be given you. And what is your request? Even up to half the kingdom, it will be granted."
>
> 在酒席筵前,王又问以斯帖说:“你要什么?我必赐给你;你求什么?就是国的一半也必为你成就。”


##### 斯5:7
> Esther replied, "My petition and my request is this:
>
> 以斯帖回答说:“我有所要,我有所求。


##### 斯5:8
> If the king regards me with favor and if it pleases the king to grant my petition and fulfill my request, let the king and Haman come tomorrow to the banquet I will prepare for them. Then I will answer the king's question."
>
> 我若在王眼前蒙恩,王若愿意赐我所要的,准我所求的,就请王带着哈曼再赴我所要预备的筵席。明日我必照王所问的说明。”


##### 斯5:9
> Haman went out that day happy and in high spirits. But when he saw Mordecai at the king's gate and observed that he neither rose nor showed fear in his presence, he was filled with rage against Mordecai.
>
> 那日,哈曼心中快乐,欢欢喜喜地出来,但见末底改在朝门不站起来,连身也不动,就满心恼怒末底改。


##### 斯5:10
> Nevertheless, Haman restrained himself and went home. Calling together his friends and Zeresh, his wife,
>
> 哈曼暂且忍耐回家,叫人请他朋友和他妻子细利斯来。


##### 斯5:11
> Haman boasted to them about his vast wealth, his many sons, and all the ways the king had honored him and how he had elevated him above the other nobles and officials.
>
> 哈曼将他富厚的荣耀,众多的儿女,和王抬举他使他超乎首领臣仆之上,都述说给他们听。


##### 斯5:12
> "And that's not all," Haman added. "I'm the only person Queen Esther invited to accompany the king to the banquet she gave. And she has invited me along with the king tomorrow.
>
> 哈曼又说:“王后以斯帖预备筵席,除了我之外,不许别人随王赴席。明日王后又请我随王赴席;


##### 斯5:13
> But all this gives me no satisfaction as long as I see that Jew Mordecai sitting at the king's gate."
>
> 只是我见犹大人末底改坐在朝门,虽有这一切荣耀,也与我无益。”


##### 斯5:14
> His wife Zeresh and all his friends said to him, "Have a gallows built, seventy-five feet high, and ask the king in the morning to have Mordecai hanged on it. Then go with the king to the dinner and be happy." This suggestion delighted Haman, and he had the gallows built.
>
> 他的妻细利斯和他一切的朋友对他说:“不如立一个五丈高的木架,明日求王将末底改挂在其上,然后你可以欢欢喜喜地随王赴席。”哈曼以这话为美,就叫人做了木架。


## 以斯帖记第6章
##### 斯6:1
> That night the king could not sleep; so he ordered the book of the chronicles, the record of his reign, to be brought in and read to him.
>
> 那夜,王睡不着觉,就吩咐人取历史来,念给他听。


##### 斯6:2
> It was found recorded there that Mordecai had exposed Bigthana and Teresh, two of the king's officers who guarded the doorway, who had conspired to assassinate King Xerxes.
>
> 正遇见书上写着说,王的太监中有两个守门的辟探和提列,想要下手害亚哈随鲁王,末底改将这事告诉王后。


##### 斯6:3
> "What honor and recognition has Mordecai received for this?" the king asked. "Nothing has been done for him," his attendants answered.
>
> 王说:“末底改行了这事,赐他什么尊荣爵位没有?”伺候王的臣仆回答说:“没有赐他什么。”


##### 斯6:4
> The king said, "Who is in the court?" Now Haman had just entered the outer court of the palace to speak to the king about hanging Mordecai on the gallows he had erected for him.
>
> 王说:“谁在院子里?”(那时哈曼正进王宫的外院,要求王将末底改挂在他所预备的木架上。)


##### 斯6:5
> His attendants answered, "Haman is standing in the court." "Bring him in," the king ordered.
>
> 臣仆说:“哈曼站在院内。”王说:“叫他进来。”


##### 斯6:6
> When Haman entered, the king asked him, "What should be done for the man the king delights to honor?" Now Haman thought to himself, "Who is there that the king would rather honor than me?"
>
> 哈曼就进去。王问他说:“王所喜悦尊荣的人,当如何待他呢?”哈曼心里说:“王所喜悦尊荣的,不是我是谁呢?”


##### 斯6:7
> So he answered the king, "For the man the king delights to honor,
>
> 哈曼就回答说:“王所喜悦尊荣的人,


##### 斯6:8
> have them bring a royal robe the king has worn and a horse the king has ridden, one with a royal crest placed on its head.
>
> 当将王常穿的朝服和戴冠的御马,


##### 斯6:9
> Then let the robe and horse be entrusted to one of the king's most noble princes. Let them robe the man the king delights to honor, and lead him on the horse through the city streets, proclaiming before him, 'This is what is done for the man the king delights to honor!'"
>
> 都交给王极尊贵的一个大臣,命他将衣服给王所喜悦尊荣的人穿上,使他骑上马,走遍城里的街市,在他面前宣告说:‘王所喜悦尊荣的人,就如此待他。’”


##### 斯6:10
> "Go at once," the king commanded Haman. "Get the robe and the horse and do just as you have suggested for Mordecai the Jew, who sits at the king's gate. Do not neglect anything you have recommended."
>
> 王对哈曼说:“你速速将这衣服和马,照你所说的,向坐在朝门的犹大人末底改去行。凡你所说的,一样不可缺。”


##### 斯6:11
> So Haman got the robe and the horse. He robed Mordecai, and led him on horseback through the city streets, proclaiming before him, "This is what is done for the man the king delights to honor!"
>
> 于是哈曼将朝服给末底改穿上,使他骑上马走遍城里的街市,在他面前宣告说:“王所喜悦尊荣的人,就如此待他。”


##### 斯6:12
> Afterward Mordecai returned to the king's gate. But Haman rushed home, with his head covered in grief,
>
> 末底改仍回到朝门;哈曼却忧忧闷闷地蒙着头,急忙回家去了,


##### 斯6:13
> and told Zeresh his wife and all his friends everything that had happened to him. His advisers and his wife Zeresh said to him, "Since Mordecai, before whom your downfall has started, is of Jewish origin, you cannot stand against him--you will surely come to ruin!"
>
> 将所遇的一切事,详细说给他的妻细利斯和他的众朋友听。他的智慧人和他的妻细利斯对他说:“你在末底改面前始而败落,他如果是犹大人,你必不能胜他,终必在他面前败落。”


##### 斯6:14
> While they were still talking with him, the king's eunuchs arrived and hurried Haman away to the banquet Esther had prepared.
>
> 他们还与哈曼说话的时候,王的太监来催哈曼快去赴以斯帖所预备的筵席。


## 以斯帖记第7章
##### 斯7:1
> So the king and Haman went to dine with Queen Esther,
>
> 王带着哈曼来赴王后以斯帖的筵席。


##### 斯7:2
> and as they were drinking wine on that second day, the king again asked, "Queen Esther, what is your petition? It will be given you. What is your request? Even up to half the kingdom, it will be granted."
>
> 这第二次在酒席筵前,王又问以斯帖说:“王后以斯帖啊,你要什么?我必赐给你;你求什么?就是国的一半,也必为你成就。”


##### 斯7:3
> Then Queen Esther answered, "If I have found favor with you, O king, and if it pleases your majesty, grant me my life--this is my petition. And spare my people--this is my request.
>
> 王后以斯帖回答说:“我若在王眼前蒙恩,王若以为美,我所愿的,是愿王将我的性命赐给我;我所求的,是求王将我的本族赐给我。


##### 斯7:4
> For I and my people have been sold for destruction and slaughter and annihilation. If we had merely been sold as male and female slaves, I would have kept quiet, because no such distress would justify disturbing the king."
>
> 因我和我的本族被卖了,要剪除杀戮灭绝我们。我们若被卖为奴为婢,我也闭口不言,但王的损失,敌人万不能补足。”


##### 斯7:5
> King Xerxes asked Queen Esther, "Who is he? Where is the man who has dared to do such a thing?"
>
> 亚哈随鲁王问王后以斯帖说:“擅敢起意如此行的是谁?这人在哪里呢?”


##### 斯7:6
> Esther said, "The adversary and enemy is this vile Haman." Then Haman was terrified before the king and queen.
>
> 以斯帖说:“仇人敌人就是这恶人哈曼!”哈曼在王和王后面前就甚惊惶。


##### 斯7:7
> The king got up in a rage, left his wine and went out into the palace garden. But Haman, realizing that the king had already decided his fate, stayed behind to beg Queen Esther for his life.
>
> 王便大怒,起来离开酒席往御园去了。哈曼见王定意要加罪与他,就起来,求王后以斯帖救命。


##### 斯7:8
> Just as the king returned from the palace garden to the banquet hall, Haman was falling on the couch where Esther was reclining. The king exclaimed, "Will he even molest the queen while she is with me in the house?" As soon as the word left the king's mouth, they covered Haman's face.
>
> 王从御园回到酒席之处,见哈曼伏在以斯帖所靠的榻上。王说:“他竟敢在宫内、在我面前凌辱王后吗?”这话一出王口,人就蒙了哈曼的脸。


##### 斯7:9
> Then Harbona, one of the eunuchs attending the king, said, "A gallows seventy-five feet high stands by Haman's house. He had it made for Mordecai, who spoke up to help the king." The king said, "Hang him on it!"
>
> 伺候王的一个太监名叫哈波拿说:“哈曼为那救王有功的末底改做了五丈高的木架,现今立在哈曼家里!”王说:“把哈曼挂在其上!”


##### 斯7:10
> So they hanged Haman on the gallows he had prepared for Mordecai. Then the king's fury subsided.
>
> 于是,人将哈曼挂在他为末底改所预备的木架上。王的忿怒这才止息。


## 以斯帖记第8章
##### 斯8:1
> That same day King Xerxes gave Queen Esther the estate of Haman, the enemy of the Jews. And Mordecai came into the presence of the king, for Esther had told how he was related to her.
>
> 当日,亚哈随鲁王把犹大人仇敌哈曼的家产赐给王后以斯帖。末底改也来到王面前,因为以斯帖已经告诉王末底改是她的亲属。


##### 斯8:2
> The king took off his signet ring, which he had reclaimed from Haman, and presented it to Mordecai. And Esther appointed him over Haman's estate.
>
> 王摘下自己的戒指,就是从哈曼追回的,给了末底改。以斯帖派末底改管理哈曼的家产。


##### 斯8:3
> Esther again pleaded with the king, falling at his feet and weeping. She begged him to put an end to the evil plan of Haman the Agagite, which he had devised against the Jews.
>
> 以斯帖又俯伏在王脚前,流泪哀告,求他除掉亚甲族哈曼害犹大人的恶谋。


##### 斯8:4
> Then the king extended the gold scepter to Esther and she arose and stood before him.
>
> 王向以斯帖伸出金杖;以斯帖就起来,站在王前,


##### 斯8:5
> "If it pleases the king," she said, "and if he regards me with favor and thinks it the right thing to do, and if he is pleased with me, let an order be written overruling the dispatches that Haman son of Hammedatha, the Agagite, devised and wrote to destroy the Jews in all the king's provinces.
>
> 说:“亚甲族哈米大他的儿子哈曼设谋传旨,要杀灭在王各省的犹大人。现今王若愿意,我若在王眼前蒙恩,王若以为美,若喜悦我,请王另下旨意,废除哈曼所传的那旨意。


##### 斯8:6
> For how can I bear to see disaster fall on my people? How can I bear to see the destruction of my family?"
>
> 我何忍见我本族的人受害?何忍见我同宗的人被灭呢?”


##### 斯8:7
> King Xerxes replied to Queen Esther and to Mordecai the Jew, "Because Haman attacked the Jews, I have given his estate to Esther, and they have hanged him on the gallows.
>
> 亚哈随鲁王对王后以斯帖和犹大人末底改说:“因哈曼要下手害犹大人,我已将他的家产赐给以斯帖,人也将哈曼挂在木架上。


##### 斯8:8
> Now write another decree in the king's name in behalf of the Jews as seems best to you, and seal it with the king's signet ring--for no document written in the king's name and sealed with his ring can be revoked."
>
> 现在你们可以随意奉王的名写谕旨给犹大人,用王的戒指盖印,因为奉王名所写、用王戒指盖印的谕旨,人都不能废除。”


##### 斯8:9
> At once the royal secretaries were summoned--on the twenty-third day of the third month, the month of Sivan. They wrote out all Mordecai's orders to the Jews, and to the satraps, governors and nobles of the 127 provinces stretching from India to Cush. These orders were written in the script of each province and the language of each people and also to the Jews in their own script and language.
>
> 三月,就是西弯月,二十三日,将王的书记召来,按着末底改所吩咐的,用各省的文字、各族的方言,并犹大人的文字、方言写谕旨,传给那从印度直到古实一百二十七省的犹大人和总督、省长、首领。


##### 斯8:10
> Mordecai wrote in the name of King Xerxes, sealed the dispatches with the king's signet ring, and sent them by mounted couriers, who rode fast horses especially bred for the king.
>
> 末底改奉亚哈随鲁王的名写谕旨,用王的戒指盖印,交给骑御马、圈快马的驿卒,传到各处。


##### 斯8:11
> The king's edict granted the Jews in every city the right to assemble and protect themselves; to destroy, kill and annihilate any armed force of any nationality or province that might attack them and their women and children; and to plunder the property of their enemies.
>
> 谕旨中,王准各省各城的犹大人在一日之间,十二月,就是亚达月十三日,聚集保护性命,


##### 斯8:12
> The day appointed for the Jews to do this in all the provinces of King Xerxes was the thirteenth day of the twelfth month, the month of Adar.
>
> 剪除杀戮灭绝那要攻击犹大人的一切仇敌,和他们的妻子儿女,夺取他们的财为掠物。


##### 斯8:13
> A copy of the text of the edict was to be issued as law in every province and made known to the people of every nationality so that the Jews would be ready on that day to avenge themselves on their enemies.
>
> 抄录这谕旨,颁行各省,宣告各族,使犹大人预备等候那日,在仇敌身上报仇。


##### 斯8:14
> The couriers, riding the royal horses, raced out, spurred on by the king's command. And the edict was also issued in the citadel of Susa.
>
> 于是,骑快马的驿卒被王命催促,急忙起行。谕旨也传遍书珊城。


##### 斯8:15
> Mordecai left the king's presence wearing royal garments of blue and white, a large crown of gold and a purple robe of fine linen. And the city of Susa held a joyous celebration.
>
> 末底改穿着蓝色、白色的朝服,头戴大金冠冕,又穿紫色细麻布的外袍,从王面前出来。书珊城的人民都欢呼快乐。


##### 斯8:16
> For the Jews it was a time of happiness and joy, gladness and honor.
>
> 犹大人有光荣,欢喜快乐而得尊贵。


##### 斯8:17
> In every province and in every city, wherever the edict of the king went, there was joy and gladness among the Jews, with feasting and celebrating. And many people of other nationalities became Jews because fear of the Jews had seized them.
>
> 王的谕旨所到的各省各城,犹大人都欢喜快乐,设摆筵宴,以那日为吉日。那国的人民,有许多因惧怕犹大人,就入了犹大籍。


## 以斯帖记第9章
##### 斯9:1
> On the thirteenth day of the twelfth month, the month of Adar, the edict commanded by the king was to be carried out. On this day the enemies of the Jews had hoped to overpower them, but now the tables were turned and the Jews got the upper hand over those who hated them.
>
> 十二月,乃亚达月,十三日,王的谕旨将要举行,就是犹大人的仇敌盼望辖制他们的日子,犹大人反倒辖制恨他们的人。


##### 斯9:2
> The Jews assembled in their cities in all the provinces of King Xerxes to attack those seeking their destruction. No one could stand against them, because the people of all the other nationalities were afraid of them.
>
> 犹大人在亚哈随鲁王各省的城里聚集,下手击杀那要害他们的人。无人能敌挡他们,因为各族都惧怕他们。


##### 斯9:3
> And all the nobles of the provinces, the satraps, the governors and the king's administrators helped the Jews, because fear of Mordecai had seized them.
>
> 各省的首领、总督、省长和办理王事的人,因惧怕末底改,就都帮助犹大人。


##### 斯9:4
> Mordecai was prominent in the palace; his reputation spread throughout the provinces, and he became more and more powerful.
>
> 末底改在朝中为大,名声传遍各省,日渐昌盛。


##### 斯9:5
> The Jews struck down all their enemies with the sword, killing and destroying them, and they did what they pleased to those who hated them.
>
> 犹大人用刀击杀一切仇敌,任意杀灭恨他们的人。


##### 斯9:6
> In the citadel of Susa, the Jews killed and destroyed five hundred men.
>
> 在书珊城,犹大人杀灭了五百人。


##### 斯9:7
> They also killed Parshandatha, Dalphon, Aspatha,
>
> 又杀巴珊大他、达分、亚斯帕他、


##### 斯9:8
> Poratha, Adalia, Aridatha,
>
> 破拉他、亚大利雅、亚利大他、


##### 斯9:9
> Parmashta, Arisai, Aridai and Vaizatha,
>
> 帕玛斯他、亚利赛、亚利代、瓦耶撒他;


##### 斯9:10
> the ten sons of Haman son of Hammedatha, the enemy of the Jews. But they did not lay their hands on the plunder.
>
> 这十人都是哈米大他的孙子、犹大人仇敌哈曼的儿子。犹大人却没有下手夺取财物。


##### 斯9:11
> The number of those slain in the citadel of Susa was reported to the king that same day.
>
> 当日,将书珊城被杀的人数呈在王前。


##### 斯9:12
> The king said to Queen Esther, "The Jews have killed and destroyed five hundred men and the ten sons of Haman in the citadel of Susa. What have they done in the rest of the king's provinces? Now what is your petition? It will be given you. What is your request? It will also be granted."
>
> 王对王后以斯帖说:“犹大人在书珊城杀灭了五百人,又杀了哈曼的十个儿子,在王的各省不知如何呢?现在你要什么?我必赐给你;你还求什么?也必为你成就。”


##### 斯9:13
> "If it pleases the king," Esther answered, "give the Jews in Susa permission to carry out this day's edict tomorrow also, and let Haman's ten sons be hanged on gallows."
>
> 以斯帖说:“王若以为美,求你准书珊的犹大人,明日也照今日的旨意行,并将哈曼十个儿子的尸首挂在木架上。”


##### 斯9:14
> So the king commanded that this be done. An edict was issued in Susa, and they hanged the ten sons of Haman.
>
> 王便允准如此行。旨意传在书珊,人就把哈曼十个儿子的尸首挂起来了。


##### 斯9:15
> The Jews in Susa came together on the fourteenth day of the month of Adar, and they put to death in Susa three hundred men, but they did not lay their hands on the plunder.
>
> 亚达月十四日,书珊的犹大人又聚集在书珊,杀了三百人,却没有下手夺取财物。


##### 斯9:16
> Meanwhile, the remainder of the Jews who were in the king's provinces also assembled to protect themselves and get relief from their enemies. They killed seventy-five thousand of them but did not lay their hands on the plunder.
>
> 在王各省其余的犹大人,也都聚集保护性命,杀了恨他们的人七万五千,却没有下手夺取财物。这样,就脱离仇敌,得享平安。


##### 斯9:17
> This happened on the thirteenth day of the month of Adar, and on the fourteenth they rested and made it a day of feasting and joy.
>
> 亚达月十三日,行了这事;十四日安息,以这日为设筵欢乐的日子。


##### 斯9:18
> The Jews in Susa, however, had assembled on the thirteenth and fourteenth, and then on the fifteenth they rested and made it a day of feasting and joy.
>
> 但书珊的犹大人,这十三日、十四日聚集杀戮仇敌;十五日安息,以这日为设筵欢乐的日子。


##### 斯9:19
> That is why rural Jews--those living in villages--observe the fourteenth of the month of Adar as a day of joy and feasting, a day for giving presents to each other.
>
> 所以住无城墙乡村的犹大人,如今都以亚达月十四日为设筵欢乐的吉日,彼此馈送礼物。


##### 斯9:20
> Mordecai recorded these events, and he sent letters to all the Jews throughout the provinces of King Xerxes, near and far,
>
> 末底改记录这事,写信与亚哈随鲁王各省远近所有的犹大人,


##### 斯9:21
> to have them celebrate annually the fourteenth and fifteenth days of the month of Adar
>
> 嘱咐他们每年守亚达月十四、十五两日,


##### 斯9:22
> as the time when the Jews got relief from their enemies, and as the month when their sorrow was turned into joy and their mourning into a day of celebration. He wrote them to observe the days as days of feasting and joy and giving presents of food to one another and gifts to the poor.
>
> 以这月的两日为犹大人脱离仇敌得平安、转忧为喜、转悲为乐的吉日。在这两日设筵欢乐,彼此馈送礼物,周济穷人。


##### 斯9:23
> So the Jews agreed to continue the celebration they had begun, doing what Mordecai had written to them.
>
> 于是,犹大人按着末底改所写与他们的信,应承照初次所守的守为永例。


##### 斯9:24
> For Haman son of Hammedatha, the Agagite, the enemy of all the Jews, had plotted against the Jews to destroy them and had cast the pur  (that is, the lot) for their ruin and destruction.
>
> 是因犹大人的仇敌亚甲族哈米大他的儿子哈曼,设谋杀害犹大人,掣普珥,就是掣签,为要杀尽灭绝他们。


##### 斯9:25
> But when the plot came to the king's attention, he issued written orders that the evil scheme Haman had devised against the Jews should come back onto his own head, and that he and his sons should be hanged on the gallows.
>
> 这事报告于王,王便降旨使哈曼谋害犹大人的恶事,归到他自己的头上,并吩咐把他和他的众子都挂在木架上。


##### 斯9:26
> (Therefore these days were called Purim, from the word pur .) Because of everything written in this letter and because of what they had seen and what had happened to them,
>
> 照着普珥的名字,犹大人就称这两日为普珥日。他们因这信上的话,又因所看见、所遇见的事,


##### 斯9:27
> the Jews took it upon themselves to establish the custom that they and their descendants and all who join them should without fail observe these two days every year, in the way prescribed and at the time appointed.
>
> 就应承自己与后裔,并归附他们的人,每年按时必守这两日,永远不废。


##### 斯9:28
> These days should be remembered and observed in every generation by every family, and in every province and in every city. And these days of Purim should never cease to be celebrated by the Jews, nor should the memory of them die out among their descendants.
>
> 各省各城、家家户户、世世代代记念遵守这两日,使这普珥日在犹大人中不可废掉,在他们后裔中也不可忘记。


##### 斯9:29
> So Queen Esther, daughter of Abihail, along with Mordecai the Jew, wrote with full authority to confirm this second letter concerning Purim.
>
> 亚比孩的女儿王后以斯帖和犹大人末底改,以全权写第二封信,坚嘱犹大人守这普珥日。


##### 斯9:30
> And Mordecai sent letters to all the Jews in the 127 provinces of the kingdom of Xerxes--words of goodwill and assurance--
>
> 用和平诚实话写信给亚哈随鲁王国中一百二十七省所有的犹大人,


##### 斯9:31
> to establish these days of Purim at their designated times, as Mordecai the Jew and Queen Esther had decreed for them, and as they had established for themselves and their descendants in regard to their times of fasting and lamentation.
>
> 劝他们按时守这普珥日,禁食呼求,是照犹大人末底改和王后以斯帖所嘱咐的,也照犹大人为自己与后裔所应承的。


##### 斯9:32
> Esther's decree confirmed these regulations about Purim, and it was written down in the records.
>
> 以斯帖命定守普珥日,这事也记录在书上。


## 以斯帖记第10章
##### 斯10:1
> King Xerxes imposed tribute throughout the empire, to its distant shores.
>
> 亚哈随鲁王使旱地和海岛的人民都进贡。


##### 斯10:2
> And all his acts of power and might, together with a full account of the greatness of Mordecai to which the king had raised him, are they not written in the book of the annals of the kings of Media and Persia?
>
> 他以权柄能力所行的,并他抬举末底改使他高升的事,岂不都写在玛代和波斯王的历史上吗?


##### 斯10:3
> Mordecai the Jew was second in rank to King Xerxes, preeminent among the Jews, and held in high esteem by his many fellow Jews, because he worked for the good of his people and spoke up for the welfare of all the Jews.
>
> 犹大人末底改作亚哈随鲁王的宰相,在犹大人中为大,得他众弟兄的喜悦,为本族的人求好处,向他们说和平的话。

