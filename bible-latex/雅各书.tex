# 雅各书
<!-- TOC -->

- [雅各书](#雅各书)
    - [雅各书第1章](#雅各书第1章)
    - [雅各书第2章](#雅各书第2章)
    - [雅各书第3章](#雅各书第3章)
    - [雅各书第4章](#雅各书第4章)
    - [雅各书第5章](#雅各书第5章)

<!-- /TOC -->
## 雅各书第1章
##### 雅1:1
> James, a servant of God and of the Lord Jesus Christ, To the twelve tribes scattered among the nations: Greetings.
>
> 作 神和主耶稣基督仆人的雅各,请散住十二个支派之人的安。


##### 雅1:2
> Consider it pure joy, my brothers, whenever you face trials of many kinds,
>
> 我的弟兄们,你们落在百般试炼中,都要以为大喜乐;


##### 雅1:3
> because you know that the testing of your faith develops perseverance.
>
> 因为知道你们的信心经过试验,就生忍耐。


##### 雅1:4
> Perseverance must finish its work so that you may be mature and complete, not lacking anything.
>
> 但忍耐也当成功,使你们成全完备,毫无缺欠。


##### 雅1:5
> If any of you lacks wisdom, he should ask God, who gives generously to all without finding fault, and it will be given to him.
>
> 你们中间若有缺少智慧的,应当求那厚赐与众人、也不斥责人的 神,主就必赐给他。


##### 雅1:6
> But when he asks, he must believe and not doubt, because he who doubts is like a wave of the sea, blown and tossed by the wind.
>
> 只要凭着信心求,一点不疑惑;因为那疑惑的人,就像海中的波浪,被风吹动翻腾。


##### 雅1:7
> That man should not think he will receive anything from the Lord;
>
> 这样的人不要想从主那里得什么。


##### 雅1:8
> he is a double-minded man, unstable in all he does.
>
> 心怀二意的人,在他一切所行的路上都没有定见。


##### 雅1:9
> The brother in humble circumstances ought to take pride in his high position.
>
> 卑微的弟兄升高,就该喜乐;


##### 雅1:10
> But the one who is rich should take pride in his low position, because he will pass away like a wild flower.
>
> 富足的降卑,也该如此。因为他必要过去,如同草上的花一样,


##### 雅1:11
> For the sun rises with scorching heat and withers the plant; its blossom falls and its beauty is destroyed. In the same way, the rich man will fade away even while he goes about his business.
>
> 太阳出来,热风刮起,草就枯干,花也凋谢,美容就消没了;那富足的人在他所行的事上,也要这样衰残。


##### 雅1:12
> Blessed is the man who perseveres under trial, because when he has stood the test, he will receive the crown of life that God has promised to those who love him.
>
> 忍受试探的人是有福的,因为他经过试验以后,必得生命的冠冕,这是主应许给那些爱他之人的。


##### 雅1:13
> When tempted, no one should say, "God is tempting me." For God cannot be tempted by evil, nor does he tempt anyone;
>
> 人被试探,不可说:“我是被 神试探”,因为 神不能被恶试探,他也不试探人。


##### 雅1:14
> but each one is tempted when, by his own evil desire, he is dragged away and enticed.
>
> 但各人被试探,乃是被自己的私欲牵引、诱惑的。


##### 雅1:15
> Then, after desire has conceived, it gives birth to sin; and sin, when it is full-grown, gives birth to death.
>
> 私欲既怀了胎,就生出罪来;罪既长成,就生出死来。


##### 雅1:16
> Don't be deceived, my dear brothers.
>
> 我亲爱的弟兄们,不要看错了。


##### 雅1:17
> Every good and perfect gift is from above, coming down from the Father of the heavenly lights, who does not change like shifting shadows.
>
> 各样美善的恩赐和各样全备的赏赐都是从上头来的,从众光之父那里降下来的;在他并没有改变,也没有转动的影儿。


##### 雅1:18
> He chose to give us birth through the word of truth, that we might be a kind of firstfruits of all he created.
>
> 他按自己的旨意,用真道生了我们,叫我们在他所造的万物中,好像初熟的果子。


##### 雅1:19
> My dear brothers, take note of this: Everyone should be quick to listen, slow to speak and slow to become angry,
>
> 我亲爱的弟兄们,这是你们所知道的。但你们各人要快快的听,慢慢的说,慢慢的动怒,


##### 雅1:20
> for man's anger does not bring about the righteous life that God desires.
>
> 因为人的怒气并不成就 神的义。


##### 雅1:21
> Therefore, get rid of all moral filth and the evil that is so prevalent and humbly accept the word planted in you, which can save you.
>
> 所以,你们要脱去一切的污秽和盈余的邪恶,存温柔的心领受那所栽种的道,就是能救你们灵魂的道。


##### 雅1:22
> Do not merely listen to the word, and so deceive yourselves. Do what it says.
>
> 只是你们要行道,不要单单听道,自己欺哄自己。


##### 雅1:23
> Anyone who listens to the word but does not do what it says is like a man who looks at his face in a mirror
>
> 因为听道而不行道的,就像人对着镜子看自己本来的面目,


##### 雅1:24
> and, after looking at himself, goes away and immediately forgets what he looks like.
>
> 看见,走后,随即忘了他的相貌如何。


##### 雅1:25
> But the man who looks intently into the perfect law that gives freedom, and continues to do this, not forgetting what he has heard, but doing it--he will be blessed in what he does.
>
> 惟有详细察看那全备、使人自由之律法的,并且时常如此,这人既不是听了就忘,乃是实在行出来,就在他所行的事上必然得福。


##### 雅1:26
> If anyone considers himself religious and yet does not keep a tight rein on his tongue, he deceives himself and his religion is worthless.
>
> 若有人自以为虔诚,却不勒住他的舌头,反欺哄自己的心,这人的虔诚是虚的。


##### 雅1:27
> Religion that God our Father accepts as pure and faultless is this: to look after orphans and widows in their distress and to keep oneself from being polluted by the world.
>
> 在 神我们的父面前,那清洁没有玷污的虔诚,就是看顾在患难中的孤儿寡妇,并且保守自己不沾染世俗。


## 雅各书第2章
##### 雅2:1
> My brothers, as believers in our glorious Lord Jesus Christ, don't show favoritism.
>
> 我的弟兄们,你们信奉我们荣耀的主耶稣基督,便不可按着外貌待人。


##### 雅2:2
> Suppose a man comes into your meeting wearing a gold ring and fine clothes, and a poor man in shabby clothes also comes in.
>
> 若有一个人戴着金戒指,穿着华美衣服,进你们的会堂去,又有一个穷人,穿着肮脏衣服也进去;


##### 雅2:3
> If you show special attention to the man wearing fine clothes and say, "Here's a good seat for you," but say to the poor man, "You stand there" or "Sit on the floor by my feet,"
>
> 你们就重看那穿华美衣服的人,说:“请坐在这好位上”,又对那穷人说:“你站在那里”,或“坐在我脚凳下边”,


##### 雅2:4
> have you not discriminated among yourselves and become judges with evil thoughts?
>
> 这岂不是你们偏心待人,用恶意断定人吗?


##### 雅2:5
> Listen, my dear brothers: Has not God chosen those who are poor in the eyes of the world to be rich in faith and to inherit the kingdom he promised those who love him?
>
> 我亲爱的弟兄们,请听! 神岂不是拣选了世上的贫穷人,叫他们在信上富足,并承受他所应许给那些爱他之人的国吗?


##### 雅2:6
> But you have insulted the poor. Is it not the rich who are exploiting you? Are they not the ones who are dragging you into court?
>
> 你们反倒羞辱贫穷人!那富足人岂不是欺压你们,拉你们到公堂去吗?


##### 雅2:7
> Are they not the ones who are slandering the noble name of him to whom you belong?
>
> 他们不是亵渎你们所敬奉的尊名吗?(所敬奉或作被称)


##### 雅2:8
> If you really keep the royal law found in Scripture, "Love your neighbor as yourself," you are doing right.
>
> 经上记着说:“要爱人如己。”你们若全守这至尊的律法,才是好的;


##### 雅2:9
> But if you show favoritism, you sin and are convicted by the law as lawbreakers.
>
> 但你们若按外貌待人,便是犯罪,被律法定为犯法的。


##### 雅2:10
> For whoever keeps the whole law and yet stumbles at just one point is guilty of breaking all of it.
>
> 因为凡遵守全律法的,只在一条上跌倒,他就是犯了众条。


##### 雅2:11
> For he who said, "Do not commit adultery," also said, "Do not murder." If you do not commit adultery but do commit murder, you have become a lawbreaker.
>
> 原来那说“不可奸淫”的,也说“不可杀人”。你就是不奸淫,却杀人,仍是成了犯律法的。


##### 雅2:12
> Speak and act as those who are going to be judged by the law that gives freedom,
>
> 你们既然要按使人自由的律法受审判,就该照这律法说话行事。


##### 雅2:13
> because judgment without mercy will be shown to anyone who has not been merciful. Mercy triumphs over judgment!
>
> 因为那不怜悯人的,也要受无怜悯的审判,怜悯原是向审判夸胜。


##### 雅2:14
> What good is it, my brothers, if a man claims to have faith but has no deeds? Can such faith save him?
>
> 我的弟兄们,若有人说自己有信心,却没有行为,有什么益处呢?这信心能救他吗?


##### 雅2:15
> Suppose a brother or sister is without clothes and daily food.
>
> 若是弟兄或是姐妹,赤身露体,又缺了日用的饮食,


##### 雅2:16
> If one of you says to him, "Go, I wish you well; keep warm and well fed," but does nothing about his physical needs, what good is it?
>
> 你们中间有人对他们说,“平平安安地去吧!愿你们穿得暖吃得饱”,却不给他们身体所需用的,这有什么益处呢?


##### 雅2:17
> In the same way, faith by itself, if it is not accompanied by action, is dead.
>
> 这样,信心若没有行为就是死的。


##### 雅2:18
> But someone will say, "You have faith; I have deeds." Show me your faith without deeds, and I will show you my faith by what I do.
>
> 必有人说:“你有信心,我有行为;你将你没有行为的信心指给我看,我便借着我的行为,将我的信心指给你看。”


##### 雅2:19
> You believe that there is one God. Good! Even the demons believe that--and shudder.
>
> 你信 神只有一位,你信的不错;鬼魔也信,却是战惊。


##### 雅2:20
> You foolish man, do you want evidence that faith without deeds is useless?
>
> 虚浮的人哪,你愿意知道没有行为的信心是死的吗?


##### 雅2:21
> Was not our ancestor Abraham considered righteous for what he did when he offered his son Isaac on the altar?
>
> 我们的祖宗亚伯拉罕把他儿子以撒献在坛上,岂不是因行为称义吗?


##### 雅2:22
> You see that his faith and his actions were working together, and his faith was made complete by what he did.
>
> 可见信心是与他的行为并行,而且信心因着行为才得成全。


##### 雅2:23
> And the scripture was fulfilled that says, "Abraham believed God, and it was credited to him as righteousness," and he was called God's friend.
>
> 这就应验经上所说:“亚伯拉罕信 神,这就算为他的义。”他又得称为 神的朋友。


##### 雅2:24
> You see that a person is justified by what he does and not by faith alone.
>
> 这样看来,人称义是因着行为,不是单因着信。


##### 雅2:25
> In the same way, was not even Rahab the prostitute considered righteous for what she did when she gave lodging to the spies and sent them off in a different direction?
>
> 妓女喇合接待使者,又放他们从别的路上出去,不也是一样因行为称义吗?


##### 雅2:26
> As the body without the spirit is dead, so faith without deeds is dead.
>
> 身体没有灵魂是死的,信心没有行为也是死的。


## 雅各书第3章
##### 雅3:1
> Not many of you should presume to be teachers, my brothers, because you know that we who teach will be judged more strictly.
>
> 我的弟兄们,不要多人作师傅,因为晓得我们要受更重的判断。


##### 雅3:2
> We all stumble in many ways. If anyone is never at fault in what he says, he is a perfect man, able to keep his whole body in check.
>
> 原来我们在许多事上都有过失;若有人在话语上没有过失,他就是完全人,也能勒住自己的全身。


##### 雅3:3
> When we put bits into the mouths of horses to make them obey us, we can turn the whole animal.
>
> 我们若把嚼环放在马嘴里,叫它顺服,就能调动它的全身。


##### 雅3:4
> Or take ships as an example. Although they are so large and are driven by strong winds, they are steered by a very small rudder wherever the pilot wants to go.
>
> 看哪,船只虽然甚大,又被大风催逼,只用小小的舵,就随着掌舵的意思转动。


##### 雅3:5
> Likewise the tongue is a small part of the body, but it makes great boasts. Consider what a great forest is set on fire by a small spark.
>
> 这样,舌头在百体里也是最小的,却能说大话。看哪,最小的火能点着最大的树林;


##### 雅3:6
> The tongue also is a fire, a world of evil among the parts of the body. It corrupts the whole person, sets the whole course of his life on fire, and is itself set on fire by hell.
>
> 舌头就是火,在我们百体中,舌头是个罪恶的世界,能污秽全身,也能把生命的轮子点起来,并且是从地狱里点着的。


##### 雅3:7
> All kinds of animals, birds, reptiles and creatures of the sea are being tamed and have been tamed by man,
>
> 各类的走兽、飞禽、昆虫、水族,本来都可以制伏,也已经被人制伏了;


##### 雅3:8
> but no man can tame the tongue. It is a restless evil, full of deadly poison.
>
> 惟独舌头没有人能制伏,是不止息的恶物,满了害死人的毒气。


##### 雅3:9
> With the tongue we praise our Lord and Father, and with it we curse men, who have been made in God's likeness.
>
> 我们用舌头颂赞那为主、为父的,又用舌头咒诅那照着 神形像被造的人。


##### 雅3:10
> Out of the same mouth come praise and cursing. My brothers, this should not be.
>
> 颂赞和咒诅从一个口里出来,我的弟兄们,这是不应当的。


##### 雅3:11
> Can both fresh water and salt water flow from the same spring?
>
> 泉源从一个眼里能发出甜苦两样的水吗?


##### 雅3:12
> My brothers, can a fig tree bear olives, or a grapevine bear figs? Neither can a salt spring produce fresh water.
>
> 我的弟兄们,无花果树能生橄榄吗?葡萄树能结无花果吗?咸水里也不能发出甜水来。


##### 雅3:13
> Who is wise and understanding among you? Let him show it by his good life, by deeds done in the humility that comes from wisdom.
>
> 你们中间谁是有智慧、有见识的呢?他就当在智慧的温柔上显出他的善行来。


##### 雅3:14
> But if you harbor bitter envy and selfish ambition in your hearts, do not boast about it or deny the truth.
>
> 你们心里若怀着苦毒的嫉妒和纷争,就不可自夸,也不可说谎话抵挡真道。


##### 雅3:15
> Such "wisdom" does not come down from heaven but is earthly, unspiritual, of the devil.
>
> 这样的智慧不是从上头来的,乃是属地的,属情欲的、属鬼魔的。


##### 雅3:16
> For where you have envy and selfish ambition, there you find disorder and every evil practice.
>
> 在何处有嫉妒纷争,就在何处有扰乱和各样的坏事。


##### 雅3:17
> But the wisdom that comes from heaven is first of all pure; then peace-loving, considerate, submissive, full of mercy and good fruit, impartial and sincere.
>
> 惟独从上头来的智慧,先是清洁,后是和平,温良柔顺,满有怜悯,多结善果,没有偏见,没有假冒。


##### 雅3:18
> Peacemakers who sow in peace raise a harvest of righteousness.
>
> 并且使人和平的,是用和平所栽种的义果。


## 雅各书第4章
##### 雅4:1
> What causes fights and quarrels among you? Don't they come from your desires that battle within you?
>
> 你们中间的争战、斗殴,是从哪里来的呢?不是从你们百体中战斗之私欲来的吗?


##### 雅4:2
> You want something but don't get it. You kill and covet, but you cannot have what you want. You quarrel and fight. You do not have, because you do not ask God.
>
> 你们贪恋,还是得不着;你们杀害嫉妒,又斗殴争战,也不能得。你们得不着,是因为你们不求;


##### 雅4:3
> When you ask, you do not receive, because you ask with wrong motives, that you may spend what you get on your pleasures.
>
> 你们求也得不着,是因为你们妄求,要浪费在你们的宴乐中。


##### 雅4:4
> You adulterous people, don't you know that friendship with the world is hatred toward God? Anyone who chooses to be a friend of the world becomes an enemy of God.
>
> 你们这些淫乱的人哪(“淫乱的人”原文作“淫妇”),岂不知与世俗为友就是与 神为敌吗?所以凡想要与世俗为友的,就是与 神为敌了。


##### 雅4:5
> Or do you think Scripture says without reason that the spirit he caused to live in us envies intensely?
>
> 你们想经上所说是徒然的吗? 神所赐住在我们里面的灵,是恋爱至于嫉妒吗?


##### 雅4:6
> But he gives us more grace. That is why Scripture says: "God opposes the proud but gives grace to the humble."
>
> 但他赐更多的恩典,所以经上说:“ 神阻挡骄傲的人,赐恩给谦卑的人。”


##### 雅4:7
> Submit yourselves, then, to God. Resist the devil, and he will flee from you.
>
> 故此,你们要顺服 神。务要抵挡魔鬼,魔鬼就必离开你们逃跑了。


##### 雅4:8
> Come near to God and he will come near to you. Wash your hands, you sinners, and purify your hearts, you double-minded.
>
> 你们亲近 神, 神就必亲近你们。有罪的人哪,要洁净你们的手;心怀二意的人哪,要清洁你们的心。


##### 雅4:9
> Grieve, mourn and wail. Change your laughter to mourning and your joy to gloom.
>
> 你们要愁苦、悲哀、哭泣,将喜笑变作悲哀,欢乐变作愁闷。


##### 雅4:10
> Humble yourselves before the Lord, and he will lift you up.
>
> 务要在主面前自卑,主就必叫你们升高。


##### 雅4:11
> Brothers, do not slander one another. Anyone who speaks against his brother or judges him speaks against the law and judges it. When you judge the law, you are not keeping it, but sitting in judgment on it.
>
> 弟兄们,你们不可彼此批评。人若批评弟兄,论断弟兄,就是批评律法,论断律法。你若论断律法,就不是遵行律法,乃是判断人的。


##### 雅4:12
> There is only one Lawgiver and Judge, the one who is able to save and destroy. But you--who are you to judge your neighbor?
>
> 设立律法和判断人的,只有一位,就是那能救人也能灭人的。你是谁,竟敢论断别人呢?


##### 雅4:13
> Now listen, you who say, "Today or tomorrow we will go to this or that city, spend a year there, carry on business and make money."
>
> 嗐,你们有话说,今天明天我们要往某城里去,在那里住一年,作买卖得利。


##### 雅4:14
> Why, you do not even know what will happen tomorrow. What is your life? You are a mist that appears for a little while and then vanishes.
>
> 其实明天如何,你们还不知道。你们的生命是什么呢?你们原来是一片云雾,出现少时就不见了。


##### 雅4:15
> Instead, you ought to say, "If it is the Lord's will, we will live and do this or that."
>
> 你们只当说:“主若愿意,我们就可以活着,也可以做这事,或做那事。”


##### 雅4:16
> As it is, you boast and brag. All such boasting is evil.
>
> 现今你们竟以张狂夸口;凡这样夸口都是恶的。


##### 雅4:17
> Anyone, then, who knows the good he ought to do and doesn't do it, sins.
>
> 人若知道行善,却不去行,这就是他的罪了。


## 雅各书第5章
##### 雅5:1
> Now listen, you rich people, weep and wail because of the misery that is coming upon you.
>
> 嗐,你们这些富足人哪,应当哭泣,号啕,因为将有苦难临到你们身上。


##### 雅5:2
> Your wealth has rotted, and moths have eaten your clothes.
>
> 你们的财物坏了,衣服也被虫子咬了。


##### 雅5:3
> Your gold and silver are corroded. Their corrosion will testify against you and eat your flesh like fire. You have hoarded wealth in the last days.
>
> 你们的金银都长了锈;那锈要证明你们的不是,又要吃你们的肉,如同火烧。你们在这末世只知积攒钱财。


##### 雅5:4
> Look! The wages you failed to pay the workmen who mowed your fields are crying out against you. The cries of the harvesters have reached the ears of the Lord Almighty.
>
> 工人给你们收割庄稼,你们亏欠他们的工钱;这工钱有声音呼叫,并且那收割之人的冤声已经入了万军之主的耳了。


##### 雅5:5
> You have lived on earth in luxury and self-indulgence. You have fattened yourselves in the day of slaughter.
>
> 你们在世上享美福,好宴乐,当宰杀的日子竟娇养你们的心。


##### 雅5:6
> You have condemned and murdered innocent men, who were not opposing you.
>
> 你们定了义人的罪,把他杀害,他也不抵挡你们。


##### 雅5:7
> Be patient, then, brothers, until the Lord's coming. See how the farmer waits for the land to yield its valuable crop and how patient he is for the autumn and spring rains.
>
> 弟兄们哪,你们要忍耐,直到主来。看哪,农夫忍耐等候地里宝贵的出产,直到得了秋雨春雨。


##### 雅5:8
> You too, be patient and stand firm, because the Lord's coming is near.
>
> 你们也当忍耐,坚固你们的心,因为主来的日子近了。


##### 雅5:9
> Don't grumble against each other, brothers, or you will be judged. The Judge is standing at the door!
>
> 弟兄们,你们不要彼此埋怨,免得受审判。看哪,审判的主站在门前了!


##### 雅5:10
> Brothers, as an example of patience in the face of suffering, take the prophets who spoke in the name of the Lord.
>
> 弟兄们,你们要把那先前奉主名说话的众先知,当作能受苦能忍耐的榜样。


##### 雅5:11
> As you know, we consider blessed those who have persevered. You have heard of Job's perseverance and have seen what the Lord finally brought about. The Lord is full of compassion and mercy.
>
> 那先前忍耐的人,我们称他们是有福的。你们听见过约伯的忍耐,也知道主给他的结局,明显主是满心怜悯,大有慈悲。


##### 雅5:12
> Above all, my brothers, do not swear--not by heaven or by earth or by anything else. Let your "Yes" be yes, and your "No," no, or you will be condemned.
>
> 我的弟兄们,最要紧的是不可起誓。不可指着天起誓,也不可指着地起誓,无论何誓都不可起。你们说话,是就说是,不是就说不是,免得你们落在审判之下。


##### 雅5:13
> Is any one of you in trouble? He should pray. Is anyone happy? Let him sing songs of praise.
>
> 你们中间有受苦的呢,他就该祷告;有喜乐的呢,他就该歌颂。


##### 雅5:14
> Is any one of you sick? He should call the elders of the church to pray over him and anoint him with oil in the name of the Lord.
>
> 你们中间有病了的呢,他就该请教会的长老来,他们可以奉主的名用油抹他,为他祷告。


##### 雅5:15
> And the prayer offered in faith will make the sick person well; the Lord will raise him up. If he has sinned, he will be forgiven.
>
> 出于信心的祈祷要救那病人,主必叫他起来;他若犯了罪,也必蒙赦免。


##### 雅5:16
> Therefore confess your sins to each other and pray for each other so that you may be healed. The prayer of a righteous man is powerful and effective.
>
> 所以你们要彼此认罪,互相代求,使你们可以得医治。义人祈祷所发的力量是大有功效的。


##### 雅5:17
> Elijah was a man just like us. He prayed earnestly that it would not rain, and it did not rain on the land for three and a half years.
>
> 以利亚与我们是一样性情的人,他恳切祷告,求不要下雨,雨就三年零六个月不下在地上。


##### 雅5:18
> Again he prayed, and the heavens gave rain, and the earth produced its crops.
>
> 他又祷告,天就降下雨来,地也生出土产。


##### 雅5:19
> My brothers, if one of you should wander from the truth and someone should bring him back,
>
> 我的弟兄们,你们中间若有失迷真道的,有人使他回转;


##### 雅5:20
> remember this: Whoever turns a sinner from the error of his way will save him from death and cover over a multitude of sins.
>
> 这人该知道叫一个罪人从迷路上转回,便是救一个灵魂不死,并且遮盖许多的罪。

