\documentclass[12pt, a4paper]{ctexart}
%%%%%%%%%%%%%%%%%%%%%%%%%%%%%    包库引用   %%%%%%%%%%%%%%%%%%%%%%%%%%%%%
\usepackage{setspace}%调整行距
\usepackage{geometry}%调整页边距
\usepackage{titletoc} % 设置目录格式
\usepackage{titlesec} % 设置标题格式
\usepackage{fancyhdr} % 设置页眉页脚
%%%%%%%%%%%%%%%%%%%%%%%%%%%%%设置中文标题编号%%%%%%%%%%%%%%%%%%%%%%%%%%%%%
% 设置中文标号
\CTEXsetup[name={,、},number={\chinese{section}},format={\centering\zihao{-4}},beforeskip={24bp},afterskip={18bp}]{section} % 一、
\CTEXsetup[name={(,)},number={\chinese{subsection}},format={\raggedright\zihao{-4}},beforeskip={12bp},afterskip={12bp}]{subsection} % (1)
\CTEXsetup[name={,.},number={\arabic{subsubsection}},format={\raggedright\zihao{-4}},beforeskip={6bp},afterskip={6bp}]{subsubsection} % 1.
% 修改标题字体字号
%%%%%%%%%%%%%%%%%%%%%%%%%%%%% 调整页眉页脚 %%%%%%%%%%%%%%%%%%%%%%%%%%%%%
\pagestyle{fancy}
\fancyhf{} % 清空所有位置的默认设置
\fancyhead{} % 右奇、左偶页页眉显示页码
\renewcommand{\headrulewidth}{0pt} % 设置页眉横线宽度为0
\fancyfoot[C]{\thepage} % 所有页的页脚中心显示页码
%%%%%%%%%%%%%%%%%%%%%%%%%%%%%     调整     %%%%%%%%%%%%%%%%%%%%%%%%%%%%%
%\onehalfspacing
\geometry{left=1.5cm,right=1.5cm,top=2.5cm,bottom=2.5cm} % 调整页边距
%%%%%%%%%%%%%%%%%%%%%%%%%%%%%   基本信息   %%%%%%%%%%%%%%%%%%%%%%%%%%%%%
\title{《圣经以色列史》7-12章阅读报告}
\newcommand{\classname}{旧约导论}
\newcommand{\teacher}{林老师}
\author{张以赛亚(zhangzundong@gmail.com)}
\date{\today}
%%%%%%%%%%%%%%%%%%%%%%%%%%%%%     开始     %%%%%%%%%%%%%%%%%%%%%%%%%%%%%
\begin{document}
%----------------封面--------------------------------------------------
\makeatletter
\begin{titlepage}
  \begin{center}
    更新神学院
    \\[7cm]
    \@title
    \vfill
    神学硕士课程
    \\\classname
    \\指导教师:\teacher
    \\[3cm]
    \@author
    \\\@date
  \end{center}
\end{titlepage}
\makeatother
%----------------目录--------------------------------------------------
\pagenumbering{roman}
\tableofcontents
\titlecontents{section}[2em]{\songti}
{\contentslabel{2.5em}}{\hspace{-2.5em}}
{\titlerule*[0.5pt]{\hspace{-2.5em}}\thecontentspage}

\titlecontents{subsection}[4em]{\songti}
{\contentslabel{2.5em}}{\hspace{-2.5em}}
{\titlerule*[0.5pt]{\hspace{-2.5em}}\thecontentspage}

\titlecontents{subsubsection}[5em]{\songti}
{\contentslabel{1em}}{\hspace{-2.5em}}
{\titlerule*[0.5pt]{\hspace{-2.5em}}\thecontentspage}
\newpage
%----------------页码计数----------------------------------------------
\pagenumbering{arabic}
%----------------章节计数----------------------------------------------
\setcounter{section}{6} % 将章节计数器设置为6,下一章就会从7开始
%----------------正文--------------------------------------------------
\section{定居迦南时期}
\subsection{引言}
古以色列的诞生、在迦南地的定居,以及演变成有组织的王国,是早期以色列史上最激动人心的篇章,同时也是最具争议性的篇章。有学者认为族长时期和出埃及的历史已成定案,因此无需讨论。然而,我们当然不赞同此见解。另一方面,学者们对以色列在迦南的出现却辩论不休。

\subsection{学术研究概述}
经过近一个世纪对以色列起源的深入研究,学者在每项建议上都有分歧。因此,可以预期本章的探讨将会相当艰难。我们按照常做法,先综述学术界的研究情况。关于以色列在迦南出现,学者提出了多个模式。我们介绍过基本的模式之后,便会考察现存的证据,包括文献和物证。

\subsection{以色列定居迦南地的史料}
文字的资料有来自圣经,也有来自圣经以外的文本。圣经资料中,约书亚记和士师记最受注意。经外资料有驰名的米聶他石碑和亚瑪拿书信,物证则包括主要遗址的考古发现和区域性表面勘测的结果。

\subsection{以色列出现在迦南:学术理论模式综述}
\subsubsection{征服模式}
与奥尔布赖特和他的美国门生,以及亚丁和跟随他的以色列学者密切相关,认真看待圣经一致传达的印象:以色列人用军事行动征服迦南地。
\subsubsection{和平渗透模式}
由阿尔特在1925年提出,以色列进入迦南并不是突发或军事行动,而是以渐进、大体上和平的状态进行。
\subsubsection{农民起义模式}
由门登霍尔提出,主张以色列在迦南出现,主要是通过内部的社会文化转变而成。
\subsubsection{其他内源模式}
认为早期以色列是迦南地现存居民中产生的,如德弗的“崩溃模式”和芬克尔斯坦的“循环模式”。

\subsection{圣经文本阅读}
\subsubsection{《约书亚记》}
从摩西死后开始,讲述耶和华吩咐摩西的助手约书亚带领民罗过约旦河,进入应许地。书中强调神赐给以色列那地的应许。
\subsubsection{《士师记》}
描述了以色列人未能忠心事奉耶和华,导致社会和宗教上的混乱,以及士师们如何被兴起拯救以色列人脱离外邦的压迫。

\subsection{经外文本阅读}
\subsubsection{米聶他石碑}
提到“以色列成为废堆”,是圣经以外最早提到“以色列”的文本,对了解以色列史极为重要。
\subsubsection{亚瑪拿书信}
提及“阿皮魯”,可能与征服迦南的“希伯來人”有关,这些书信反映了迦南城邦统治者与埃及的往来。

\subsection{考古学发现}
\subsubsection{耶利哥、艾、夏瑣、拉億}
考古发现与圣经记载有吻合之处,如耶利哥的城墙倒塌和焚毁证据,支持了圣经中关于约书亚征服耶利哥的记载。
\subsubsection{基遍、示羅、以巴路山}
提供了关于以色列人定居迦南的额外信息,如基遍的挖掘发现了原始希伯来文刻上“基遍”的名字的瓶子。

\subsection{结语}
本章讨论了以色列在迦南兴起的历史,结合了圣经文本、经外文本和考古学发现。虽然存在争议和未解之谜,但现有的证据并没有推翻圣经的基本记载。我们认为,要撰写以色列在迦南崛兴的历史,不必采取与圣经记载截然不同的方向。


\section{王国早期}

\subsection{以色列社会的转变}
撒母耳记上下详细记载了以色列社会从士师时代向王国时代过渡的重大变化,这些变化对政治和宗教产生了深远的影响。士师记结束时提到,以色列中没有王,各人都行自己认为对的事。撒母耳记上开始时,情况并未有太大改变,但王国的诞生已指日可待。

\subsection{撒母耳记的预言}
撒母耳记上二章1至10节的哈拿之歌预言了一位将临的王,并特别提到耶和华将赐力量给他所立的王。在以色列第一位王出现之前,撒母耳作为先知的角色至关重要。

\subsection{撒母耳的出生与成长}
撒母耳记上前几章集中叙述了撒母耳的出生和成长,他被耶和华确立为先知。撒母耳的事业很快与以色列第一位王扫罗的经历联系在一起,随后与大卫的经历相连,大卫建立了以色列第一个且最长久的王朝。

\subsection{文学与历史的价值}
撒母耳记作为文学作品,其叙事受到高度赞扬。但其历史价值如何?撒母耳记的叙事是否只是生动的故事?还是也是生动的历史?我们可以合理地将其视为历史编纂吗?

\subsection{考古学的证据}
一些圣经学者曾公开表示大卫和所罗门可能并非真实存在。但随着考古学的发展,如但城的挖掘工作,发现了提及“大卫家”的铭文,为大卫和所罗门的历史真实性提供了额外的支持。

\subsection{以色列王国早期资料来源}
描述以色列联合王国时代的圣经书卷,除了撒母耳记上下,还有历代志上下。这些书卷提供了不同的视角和重点,帮助我们理解王国时期的情况。

\subsection{结语}
本章讨论了以色列过渡到君王制度时期的圣经见证的可信度。撒母耳记上下和历代志上下都是具有各自目的和观点的历史编纂作品。尽管存在争议,但圣经文本的历史价值并未因此而受到削弱。考古学的发现,如耶路撒冷的考古挖掘,为大卫统治时期提供了背景资料,尽管对于文物的解释存在争议。最终,我们发现圣经的见证在历史编纂中占有重要地位。

\section{王国后期:所罗门时期}

第九章《王国后期:所罗门时期》提供了对所罗门王统治时期的深入探讨,以下是对每个主题的详细叙述和原文内容引用:

\subsection{所罗门的统治}
所罗门继承了大卫的王位,成为以色列的国王,开启了以色列历史上的一个繁荣时期。《列王纪上》第三章描述了所罗门如何向上帝求智慧,以公正治理国家。上帝不仅赐给他智慧,还赐给他财富和荣耀,使他超越了以往的君王。所罗门的统治不仅体现在他的智慧和公正上,还体现在他对国家的管理和对外关系上。

根据《列王纪上》第四章的记载,所罗门的统治版图从“大河”(幼发拉底河)延伸至非利士地,直到埃及的边界。具体来说,他的统治范围从提弗萨(位于叙利亚的阿勒颇地区,幼发拉底河岸)一直到迦萨(非利士地最南端的海岸城市)。这表明所罗门的王国覆盖了整个迦南地区,包括今天的以色列、约旦、黎巴嫩的一部分以及叙利亚的部分地区。

所罗门王的版图大小在《列王纪上》中并没有给出具体的地理尺寸,但通过这些描述可以推断,他的王国在当时是一个强大的地区性帝国,控制了重要的贸易路线,并在中东地区拥有显著的影响力。

\subsection{资料来源和年代顺序问题}
《列王纪》和《历代志》是研究所罗门时期历史的主要圣经资料来源。然而,这些文本的年代顺序并不总是清晰,给历史学家带来了挑战。例如,《列王纪上》第六章和第十一章提供了所罗门建筑项目和统治时期的信息,但具体的年代顺序需要通过考古发现和历史研究来进一步确认。

《列王纪》和《历代志》是两卷圣经文本,它们记录了以色列王国的历史,包括君王的统治、重要的事件和宗教的发展。这两卷书提供了关于以色列历史的宝贵资料,但它们在年代顺序的记录上存在一些问题和挑战,理解这些问题的关键在于以下几个方面:

\subsubsection{相对年代与绝对年代}
《列王纪》提供的年代顺序主要是相对的,即一个王的统治开始和结束是相对于前一个王的统治结束来确定的。这种记录方式有助于理解王权的更迭,但并不直接提供具体的公元年份。

要确定一个更精确的绝对年代顺序,需要依赖外部的历史资料,如亚述和巴比伦的编年史,以及考古发现。

\subsubsection{年代记录的不一致性}
有时候,《列王纪》和《历代志》中对于同一事件或统治时期的记录存在差异。这些差异可能是由于不同的资料来源、编纂者的意图或文本传承过程中的变化造成的。

\subsubsection{编纂目的的影响}
《列王纪》和《历代志》的编纂目的不仅仅是记录历史事实,它们还旨在传达特定的宗教和政治信息。因此,年代记录的选择性可能受到了编纂者意图的影响。

\subsubsection{历史与文学的交织}
这些圣经文本是历史和文学交织的产物。年代记录可能被文学化,以符合叙事的需要,这可能导致年代顺序的模糊或夸张。

\subsubsection{考古学的贡献}
考古学发现可以为理解年代顺序提供额外的线索。例如,通过考古发掘出的文物和建筑遗迹,可以为圣经中提到的某些王和事件提供年代上的参考。

\subsubsection{历史编纂的复杂性}
古代历史编纂是一个复杂的过程,涉及到资料的选择、编排和解释。《列王纪》和《历代志》中的年代顺序问题反映了这一过程的复杂性。

\subsubsection{现代学术研究}
现代学者通过对比圣经文本与其他古代近东的历史记录,以及利用放射性碳定年等科学技术,尝试解决年代顺序的问题。

总的来说,理解《列王纪》和《历代志》中的年代顺序问题需要综合考虑文本的编纂背景、历史资料的局限性、考古学的发现以及现代学术研究的成果。这要求读者在阅读这些圣经文本时,既要欣赏其文学和宗教价值,也要意识到它们作为历史资料的局限性。


\subsection{所罗门的早期事件}
所罗门登基初期,采取了一系列措施来巩固自己的权力。《列王纪上》第二章描述了所罗门如何消除潜在的政治威胁,确保了王位的稳定。此外,所罗门与埃及的联姻,通过与法老的女儿结婚,加强了与周边大国的外交关系。

\subsection{所罗门的行政和成就}
所罗门的行政改革和建筑项目体现了他的组织能力和对国家的远见。《列王纪上》第九章详细记载了所罗门如何组织劳动力,进行圣殿和王宫的建设。这些建筑不仅展示了所罗门的财富和权力,也成为了以色列民族的象征。

所罗门圣殿是古代以色列历史上最重要的建筑之一,其建筑细节在《列王纪上》第6章和第7章中有详细的描述。以下是一些关键的建筑细节:

\subsubsection{结构布局}
圣殿由三个主要部分组成:圣所(Holy Place)、至圣所(Holy of Holies)和入口的前廊(Portico)。

\subsubsection{尺寸}
圣殿的内部尺寸约为20立方米(长约6.5米,宽约6.5米,高约6米)。整个建筑坐落在一个高约1米的平台上。

\subsubsection{材料}
圣殿的建造使用了大量珍贵的材料,包括石头、木材和金属。墙壁和地板是用石头建造的,而内部则大量使用了香柏木和黄杨木。

\subsubsection{装饰}
圣殿内部装饰华丽,墙壁和地板覆盖着细麻布,墙壁上还贴有薄金。至圣所内有两个基路伯雕像,用橄榄木雕刻并覆盖金箔,它们展开翅膀,遮盖着约柜(Ark of the Covenant)。

\subsubsection{约柜的放置}
至圣所内放置着约柜,这是存放十诫法版的地方。

\subsubsection{灯具和器具}
圣殿内有七个分支的金灯台(Menorah)和金香坛(Golden Altar),以及用于献祭的金桌子。

\subsubsection{入口前廊}
圣殿入口处有一个大型前廊,也称为“乌陵和土明的房间”,这里存放着祭司用来确定上帝旨意的乌陵和土明。

\subsubsection{窗户和门}
圣殿的窗户是用细工木格子制成的,门是用橄榄木制成,表面覆盖金箔。

\subsubsection{建筑时间}
根据《列王纪上》的记载,所罗门圣殿的建造工作始于所罗门王第四年,历时约七年完成。

\subsubsection{建筑师}
所罗门聘请了一位名叫希兰的泰尔(今黎巴嫩)建筑师来设计和监督圣殿的建造。

这些建筑细节不仅展示了所罗门时期的建筑技术和艺术水平,也反映了当时的宗教信仰和社会文化。所罗门圣殿成为了以色列人宗教生活的中心,直到它在公元前586年被巴比伦人摧毁。


\subsection{所罗门的宗教政策}
所罗门在宗教方面的政策体现了他对耶和华的崇拜。《列王纪上》第八章描述了所罗门如何为耶和华建立圣殿,并在圣殿竣工时举行盛大的奉献仪式。然而,所罗门晚年可能受到外族妻子的影响,转向了其他神祇的崇拜,这一点在《列王纪上》第十一章有所提及。

\subsection{所罗门时期的社会和经济状况}
所罗门时期的社会和经济状况反映了国家的繁荣。《列王纪上》第四章提到了所罗门的贸易和财富,显示了以色列在当时的地区经济中占据了重要地位。所罗门的统治带来了经济的增长和社会的稳定。

\subsection{王国的分裂迹象}
尽管所罗门的统治带来了繁荣,但《列王纪上》第十一章也提到了王国内部的紧张关系,这些紧张关系最终导致了王国的分裂。所罗门去世后,以色列王国分裂为北方的以色列国和南方的犹大国。

根据《列王纪上》的记载,王国分裂的迹象和原因主要包括以下几点:

\subsubsection{所罗门的晚年政策}
所罗门在位晚期,为了维持国家的繁荣和建筑项目,加重了税收和劳役,这使得民众负担加重,不满情绪逐渐积累。

\subsubsection{经济负担}
所罗门大兴土木,包括建造圣殿、王宫和多个城邑,这些项目需要大量的人力和物力,给百姓带来了沉重的经济压力。

\subsubsection{宗教问题}
所罗门晚年可能在宗教上出现了偏差,他娶了许多外邦女子为妻,这些妻子将自己的偶像带入以色列,可能导致了宗教上的混乱和不纯。

\subsubsection{耶罗波安的崛起}
耶罗波安作为以色列支派的代表,对所罗门的统治感到不满,他的崛起反映了王国内部的分裂倾向。耶罗波安在示罗遇见亚希雅先知,先知预言了王国的分裂。

\subsubsection{民众的不满}
《列王纪上》第十二章提到,以色列民众感受到的生活压力与在埃及为奴时相似,这种不满情绪为王国的分裂埋下了伏笔。

\subsubsection{所罗门去世后的权力真空}
所罗门去世后,没有一位强有力的继承人能够维持国家的统一,这导致了权力的真空和各支派间的争斗。

\subsubsection{南北王国的差异}
北方支派与南方支派在经济、政治和文化上存在差异,这些差异在所罗门死后加剧了分裂的趋势。

\subsubsection{外部势力的影响}
周边国家如埃及和亚述的动态也可能对以色列王国的稳定产生了影响,特别是在王国内部出现弱点时。

这些迹象表明,王国的分裂是多方面因素共同作用的结果,不仅仅是一个单一事件或决策所导致的。《列王纪上》第十二章至第十四章详细记载了王国分裂的过程,以及耶罗波安成为北方以色列王国的第一位王,而罗波安(所罗门的儿子)则继续统治南方的犹大王国。


\subsection{结语}
第九章通过对所罗门时期的深入分析,展示了以色列王国在其鼎盛时期的复杂性和多样性,以及所罗门个人在政治、经济和宗教方面的重要影响。这一时期的以色列历史不仅对当时的社会产生了深远的影响,也为后世留下了丰富的文化遗产。通过对圣经文本的认真研究和考古发现的综合考量,我们可以更全面地理解所罗门时期的以色列王国。

\end{document}