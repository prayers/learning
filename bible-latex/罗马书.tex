 
%%%%%%%%%%%%%%%%%%%%%%%%%%%%%%%%%%%%%%%%%%%%%%%%%%%%%%%%%%%%%%%%
%  文章模板:A4 纸,小五字,单列(可根据要求改双列 twocolumn)
%%%%%%%%%%%%%%%%%%%%%%%%%%%%%%%%%%%%%%%%%%%%%%%%%%%%%%%%%%%%%%%%
\documentclass[a4paper,11pt,onecolumn,twoside]{ctexart}
%%%%%%%%%%%%%%%%%%%%%%%%%%%%%%%%%%%%%%%%%%%%%%%%%%%%%%%%%%%%%%%%
%  packages
%    这部分声明需要用到的包
%%%%%%%%%%%%%%%%%%%%%%%%%%%%%%%%%%%%%%%%%%%%%%%%%%%%%%%%%%%%%%%%
\usepackage{fancyhdr}
\usepackage{amsmath,amsfonts,amssymb,graphicx}    % EPS 图片支持
\usepackage{subfigure}   % 使用子图形
\usepackage{indentfirst} % 中文段落首行缩进
\usepackage{bm}          % 公式中的粗体字符(用命令\boldsymbol)
\usepackage{multicol}    % 正文双栏
\usepackage{indentfirst} % 中文首段缩进
\usepackage{picins}      % 图片嵌入段落宏包 比如照片
\usepackage{abstract}    % 2栏文档,一栏摘要及关键字宏包
%%%%%%%%%%%%%%%%%%%%%%%%%%%%%%%%%%%%%%%%%%%%%%%%%%%%%%%%%%%%%%%%
%  lengths
%    下面的命令重定义页面边距,使其符合中文刊物习惯。
%%%%%%%%%%%%%%%%%%%%%%%%%%%%%%%%%%%%%%%%%%%%%%%%%%%%%%%%%%%%%%%%
\addtolength{\topmargin}{-54pt}
\setlength{\oddsidemargin}{-0.9cm}  % 3.17cm - 1 inch
\setlength{\evensidemargin}{\oddsidemargin}
\setlength{\textwidth}{17.00cm}
\setlength{\textheight}{24.00cm}    % 24.62
%%%%%%%%%%%%%%%%%%%%%%%%%%%%%%%%%%%%%%%%%%%%%%%%%%%%%%%%%%%%%%%%
%  定义标题格式,包括title,author,affiliation,email等。
%  在任何用到中文的地方,用\begin{CJK} ... \end{CJK}将其括起来。
%%%%%%%%%%%%%%%%%%%%%%%%%%%%%%%%%%%%%%%%%%%%%%%%%%%%%%%%%%%%%%%%

\renewcommand{\baselinestretch}{1.1} %定义行间距
\parindent 22pt %重新定义缩进长度

%%%%%%%%%%%%%%%%%%%%%%%%%%%%%%%%%%%%%%%%%%%%%%%%%%%%%%%%%%%%%%%%
% 标题,作者,通信地址定义
%%%%%%%%%%%%%%%%%%%%%%%%%%%%%%%%%%%%%%%%%%%%%%%%%%%%%%%%%%%%%%%%

\title{
	\huge{罗~~马~~书}
	\thanks{Note: 2020-12-02}
}
\author{Paul}
\date{}  % 这一行用来去掉默认的日期显示
%%%%%%%%%%%%%%%%%%%%%%%%%%%%%%%%%%%%%%%%%%%%%%%%%%%%%%%%%%%%%%%%
% 正文两栏环境不允许float环境,比如 figure, table。所以重新定义
% figure,使之可以浮动到你想要的位置。table也同样,把figure改为
% table就可以。
%%%%%%%%%%%%%%%%%%%%%%%%%%%%%%%%%%%%%%%%%%%%%%%%%%%%%%%%%%%%%%%%
\newenvironment{figurehere}
  {\def\@captype{figure}}
  {}
\makeatother
%%%%%%%%%%%%%%%%%%%%%%%%%%%%%%%%%%%%%%%%%%%%%%%%%%%%%%%%%%%%%%%%
%  文章正文
%%%%%%%%%%%%%%%%%%%%%%%%%%%%%%%%%%%%%%%%%%%%%%%%%%%%%%%%%%%%%%%%
\begin{document}
%%%%%%%%%%%%%%%%%%%%%%%%%%%%%%%%%%%%%%%%%%%%%%%%%%%%%%%%%%%%%%%%
%  自定义命令
%%%%%%%%%%%%%%%%%%%%%%%%%%%%%%%%%%%%%%%%%%%%%%%%%%%%%%%%%%%%%%%%
% 此行使文献引用以上标形式显示
\newcommand{\supercite}[1]{\textsuperscript{\cite{#1}}}
%%%%%%%%%%%%%%%%%%%%%%%%%%%%%%%%%%%%%%%%%%%%%%%%%%%%%%%%%%%%%%%%
%  显示title,并设页码为空(按杂志社要求)
%%%%%%%%%%%%%%%%%%%%%%%%%%%%%%%%%%%%%%%%%%%%%%%%%%%%%%%%%%%%%%%%
\maketitle
%%%%%%%%%%%%%%%%%%%%%%%%%%%%%%%%%%%%%%%%%%%%%%%%%%%%%%%%%%%%%%%%
%%%%%%%%%%%%%%%%%%%%%%%%%%%%%%%%%%%%%%%%%%%%%%%%%%%%%%%%%%%%%%%%
%  中文摘要
%  调整摘要、关键词,中图分类号的页边距
%  中英文同时调整
%%%%%%%%%%%%%%%%%%%%%%%%%%%%%%%%%%%%%%%%%%%%%%%%%%%%%%%%%%%%%%%%
\setlength{\oddsidemargin}{ 1cm}  % 3.17cm - 1 inch
\setlength{\evensidemargin}{\oddsidemargin}
\setlength{\textwidth}{13.50cm}
\vspace{-.8cm}
\begin{center}
	\parbox{\textwidth}{\textbf{摘要}\quad  《罗马书》在圣经中列在书信的最前面。马丁路得称《罗马书》为福音摘要;又说基督教只要有《约翰福音》和《罗马书》就不致消灭,仍必发扬光大。他也勉励信徒读这书;他说人可尽量研读《罗马书》,研读得越多,越能发现它的宝藏。加尔文也见证说:任何人若通晓此书,便是找到了一条明白整本圣经的通道。\\事实上,本书是教会历史上最具影响力的圣经书卷。奥古斯丁因读到本书第十三章而悔改归主;马丁路得因藉本书而领悟到“因信称义”的真理,乃掀起了宗教改革;约翰韦斯利因听见别人朗读马丁路得的《罗马书注释》而体会到得救的确据。\\总而言之,本书乃是经中的大经,无论就它所论到的题目之大,所引用的旧约圣经之多,所叙事物的范围之广,以及神所豫定的救恩之丰,均非其他经书所可比拟。\\
		\textbf{关键词}\quad 书信,因信称义
}
\end{center}

%%%%%%%%%%%%%%%%%%%%%%%%%%%%%%%%%%%%%%%%%%%%%%%%%%%%%%%%%%%%%%%%
%  英文摘要
%%%%%%%%%%%%%%%%%%%%%%%%%%%%%%%%%%%%%%%%%%%%%%%%%%%%%%%%%%%%%%%%
\vspace{.1cm}
\begin{center}
	\parbox{\textwidth}{{\large{\textbf{Romans}}}\\
	\vspace{-0.5cm}
	\begin{center}
		\textbf{Paul}\\[2pt]
	\end{center}
	{\small{\textbf{Abstract}\quad The Book of Romans is so important that it is listed as the first one among all the epistles of the Bible. Martin Luther called the Book of Romans was the abstract of the gospel. He also said that as long as the Christianity had the Gospel of John and the Book of Romans, it would not be destroyed and would be still manifested. He also encouraged believers to read and even intensively read the book. He said, the more one read the Book of Romans, the more its treasure one would find. Calvin also witnessed: “if anyone understands this book, he has found the way of the whole Bible”.\\
			In fact, among all the books of the Bible, this book is the most influential one in the history of the church. Augustus repented after reading the 13th chapter of this book. Martin Luther understood the truth “justification by faith” through this book and set off the revolution of religion. John Wesley received the assurance of being saved after he had heard Martin Luther’s commentary of the Book of Romans that was read aloud by others.\\
			In a word, this book is the scriptures of scriptures, which surpasses other books in the greatness of the theme, the frequency of quotations of the Old Testament, the width of the sphere of things it mentions, as well as the fullness of the salvation that God had predestined.\\
	\textbf{Key Words}\quad epistle, Justification by Faith}}
	}
\end{center}
%%%%%%%%%%%%%%%%%%%%%%%%%%%%%%%%%%%%%%%%%%%%%%%%%%%%%%%%%%%%%%%%
%  正文由此开始-------------------------
%%%%%%%%%%%%%%%%%%%%%%%%%%%%%%%%%%%%%%%%%%%%%%%%%%%%%%%%%%%%%%%%
%%%%%%%%%%%%%%%%%%%%%%%%%%%%%%%%%%%%%%%%%%%%%%%%%%%%%%%%%%%%%%%%
%  恢复正文页边距
%%%%%%%%%%%%%%%%%%%%%%%%%%%%%%%%%%%%%%%%%%%%%%%%%%%%%%%%%%%%%%%%
\setlength{\oddsidemargin}{-.5cm}  % 3.17cm - 1 inch
\setlength{\evensidemargin}{\oddsidemargin}
\setlength{\textwidth}{17.00cm}
%%%%%%%%%%%%%%%%%%%%%%%%%%%%%%%%%%%%%%%%%%%%%%%%%%%%%%%%%%%%%%%%
%  分栏开始
\begin{multicols}{2}
%%%%%%%%%%%%%%%%%%%%%%%%%%%%%%%%%%%%%%%%%%%%%%%%%%%%%%%%%%%%%%%%
%  目录
\tableofcontents
%%%%%%%%%%%%%%%%%%%%%%%%%%%%%%%%%%%%%%%%%%%%%%%%%%%%%%%%%%%%%%%%
\section{引言}
罗马书介绍
\section{罗马书第1章}

 罗1:1
 Paul, a servant of Christ Jesus, called to be an apostle and set apart for the gospel of God--

 耶稣基督的仆人保罗,奉召为使徒,特派传 神的福音。


 罗1:2
 the gospel he promised beforehand through his prophets in the Holy Scriptures

 这福音是 神从前借众先知在圣经上所应许的。


 罗1:3
 regarding his Son, who as to his human nature was a descendant of David,

 论到他儿子我主耶稣基督,按肉体说,是从大卫后裔生的;


 罗1:4
 and who through the Spirit of holiness was declared with power to be the Son of God by his resurrection from the dead: Jesus Christ our Lord.

 按圣善的灵说,因从死里复活,以大能显明是 神的儿子。


 罗1:5
 Through him and for his name's sake, we received grace and apostleship to call people from among all the Gentiles to the obedience that comes from faith.

 我们从他受了恩惠并使徒的职分,在万国之中叫人为他的名信服真道;


 罗1:6
 And you also are among those who are called to belong to Jesus Christ.

 其中也有你们这蒙召属耶稣基督的人。


 罗1:7
 To all in Rome who are loved by God and called to be saints: Grace and peace to you from God our Father and from the Lord Jesus Christ.

 我写信给你们在罗马为 神所爱、奉召作圣徒的众人。愿恩惠、平安从我们的父 神并主耶稣基督归与你们!


 罗1:8
 First, I thank my God through Jesus Christ for all of you, because your faith is being reported all over the world.

 第一,我靠着耶稣基督,为你们众人感谢我的 神,因你们的信德传遍了天下。


 罗1:9
 God, whom I serve with my whole heart in preaching the gospel of his Son, is my witness how constantly I remember you

 我在他儿子福音上,用心灵所事奉的 神,可以见证我怎样不住地提到你们。


 罗1:10
 in my prayers at all times; and I pray that now at last by God's will the way may be opened for me to come to you.

 在祷告之间常常恳求,或者照 神的旨意,终能得平坦的道路往你们那里去。


 罗1:11
 I long to see you so that I may impart to you some spiritual gift to make you strong--

 因为我切切地想见你们,要把些属灵的恩赐分给你们,使你们可以坚固。


 罗1:12
 that is, that you and I may be mutually encouraged by each other's faith.

 这样,我在你们中间,因你与我彼此的信心,就可以同得安慰。


 罗1:13
 I do not want you to be unaware, brothers, that I planned many times to come to you (but have been prevented from doing so until now) in order that I might have a harvest among you, just as I have had among the other Gentiles.

 弟兄们,我不愿意你们不知道,我屡次定意往你们那里去,要在你们中间得些果子,如同在其余的外邦人中一样;只是到如今仍有阻隔。


 罗1:14
 I am obligated both to Greeks and non-Greeks, both to the wise and the foolish.

 无论是希利尼人、化外人、聪明人、愚拙人,我都欠他们的债。


 罗1:15
 That is why I am so eager to preach the gospel also to you who are at Rome.

 所以情愿尽我的力量,将福音也传给你们在罗马的人。


 罗1:16
 I am not ashamed of the gospel, because it is the power of God for the salvation of everyone who believes: first for the Jew, then for the Gentile.

 我不以福音为耻;这福音本是 神的大能,要救一切相信的,先是犹太人,后是希利尼人。


 罗1:17
 For in the gospel a righteousness from God is revealed, a righteousness that is by faith from first to last, just as it is written: "The righteous will live by faith."

 因为 神的义正在这福音上显明出来;这义是本于信,以至于信。如经上所记:“义人必因信得生。”


 罗1:18
 The wrath of God is being revealed from heaven against all the godlessness and wickedness of men who suppress the truth by their wickedness,

 原来, 神的忿怒,从天上显明在一切不虔不义的人身上,就是那些行不义阻挡真理的人。


 罗1:19
 since what may be known about God is plain to them, because God has made it plain to them.

 神的事情,人所能知道的,原显明在人心里,因为 神已经给他们显明。


 罗1:20
 For since the creation of the world God's invisible qualities--his eternal power and divine nature--have been clearly seen, being understood from what has been made, so that men are without excuse.

 自从造天地以来, 神的永能和 神性是明明可知的,虽是眼不能见,但借着所造之物就可以晓得,叫人无可推诿。


 罗1:21
 For although they knew God, they neither glorified him as God nor gave thanks to him, but their thinking became futile and their foolish hearts were darkened.

 因为他们虽然知道 神,却不当作 神荣耀他,也不感谢他。他们的思念变为虚妄,无知的心就昏暗了。


 罗1:22
 Although they claimed to be wise, they became fools

 自称为聪明,反成了愚拙;


 罗1:23
 and exchanged the glory of the immortal God for images made to look like mortal man and birds and animals and reptiles.

 将不能朽坏之神的荣耀变为偶像,仿佛必朽坏的人和飞禽、走兽、昆虫的样式。


 罗1:24
 Therefore God gave them over in the sinful desires of their hearts to sexual impurity for the degrading of their bodies with one another.

 所以, 神任凭他们逞着心里的情欲行污秽的事,以致彼此玷辱自己的身体。


 罗1:25
 They exchanged the truth of God for a lie, and worshiped and served created things rather than the Creator--who is forever praised. Amen.

 他们将 神的真实变为虚谎,去敬拜事奉受造之物,不敬奉那造物的主。主乃是可称颂的,直到永远。阿们!


 罗1:26
 Because of this, God gave them over to shameful lusts. Even their women exchanged natural relations for unnatural ones.

 因此, 神任凭他们放纵可羞耻的情欲。他们的女人把顺性的用处变为逆性的用处;


 罗1:27
 In the same way the men also abandoned natural relations with women and were inflamed with lust for one another. Men committed indecent acts with other men, and received in themselves the due penalty for their perversion.

 男人也是如此,弃了女人顺性的用处,欲火攻心,彼此贪恋,男和男行可羞耻的事,就在自己身上受这妄为当得的报应。


 罗1:28
 Furthermore, since they did not think it worthwhile to retain the knowledge of God, he gave them over to a depraved mind, to do what ought not to be done.

 他们既然故意不认识 神, 神就任凭他们存邪僻的心,行那些不合理的事;


 罗1:29
 They have become filled with every kind of wickedness, evil, greed and depravity. They are full of envy, murder, strife, deceit and malice. They are gossips,

 装满了各样不义、邪恶、贪婪、恶毒(或作“阴毒”),满心是嫉妒、凶杀、争竞、诡诈、毒恨,


 罗1:30
 slanderers, God-haters, insolent, arrogant and boastful; they invent ways of doing evil; they disobey their parents;

 又是谗毁的、背后说人的、怨恨 神的(或作“被 神所憎恶的”)、侮慢人的、狂傲的、自夸的、捏造恶事的、违背父母的、


 罗1:31
 they are senseless, faithless, heartless, ruthless.

 无知的、背约的、无亲情的、不怜悯人的。


 罗1:32
 Although they know God's righteous decree that those who do such things deserve death, they not only continue to do these very things but also approve of those who practice them.

 他们虽知道 神判定行这样事的人是当死的,然而他们不但自己去行,还喜欢别人去行。


\section{ 罗马书第2章}
 罗2:1
 You, therefore, have no excuse, you who pass judgment on someone else, for at whatever point you judge the other, you are condemning yourself, because you who pass judgment do the same things.

 你这论断人的,无论你是谁,也无可推诿。你在什么事上论断人,就在什么事上定自己的罪,因你这论断人的,自己所行却和别人一样。


 罗2:2
 Now we know that God's judgment against those who do such things is based on truth.

 我们知道这样行的人, 神必照真理审判他。


 罗2:3
 So when you, a mere man, pass judgment on them and yet do the same things, do you think you will escape God's judgment?

 你这人哪,你论断行这样事的人,自己所行的却和别人一样!你以为能逃脱 神的审判吗?


 罗2:4
 Or do you show contempt for the riches of his kindness, tolerance and patience, not realizing that God's kindness leads you toward repentance?

 还是你藐视他丰富的恩慈、宽容、忍耐,不晓得他的恩慈是领你悔改呢?


 罗2:5
 But because of your stubbornness and your unrepentant heart, you are storing up wrath against yourself for the day of God's wrath, when his righteous judgment will be revealed.

 你竟任着你刚硬不悔改的心,为自己积蓄忿怒,以致 神震怒,显他公义审判的日子来到。


 罗2:6
 God "will give to each person according to what he has done."

 他必照各人的行为报应各人。


 罗2:7
 To those who by persistence in doing good seek glory, honor and immortality, he will give eternal life.

 凡恒心行善,寻求荣耀、尊贵和不能朽坏之福的,就以永生报应他们;


 罗2:8
 But for those who are self-seeking and who reject the truth and follow evil, there will be wrath and anger.

 惟有结党不顺从真理,反顺从不义的,就以忿怒、恼恨报应他们。


 罗2:9
 There will be trouble and distress for every human being who does evil: first for the Jew, then for the Gentile;

 将患难、困苦加给一切作恶的人,先是犹太人,后是希利尼人;


 罗2:10
 but glory, honor and peace for everyone who does good: first for the Jew, then for the Gentile.

 却将荣耀、尊贵、平安加给一切行善的人,先是犹太人,后是希利尼人。


 罗2:11
 For God does not show favoritism.

 因为 神不偏待人。


 罗2:12
 All who sin apart from the law will also perish apart from the law, and all who sin under the law will be judged by the law.

 凡没有律法犯了罪的,也必不按律法灭亡;凡在律法以下犯了罪的,也必按律法受审判。


 罗2:13
 For it is not those who hear the law who are righteous in God's sight, but it is those who obey the law who will be declared righteous.

 (原来,在 神面前不是听律法的为义,乃是行律法的称义。


 罗2:14
 (Indeed, when Gentiles, who do not have the law, do by nature things required by the law, they are a law for themselves, even though they do not have the law,

 没有律法的外邦人,若顺着本性行律法上的事,他们虽然没有律法,自己就是自己的律法。


 罗2:15
 since they show that the requirements of the law are written on their hearts, their consciences also bearing witness, and their thoughts now accusing, now even defending them.)

 这是显出律法的功用刻在他们心里,他们是非之心同作见证,并且他们的思念互相较量,或以为是,或以为非。)


 罗2:16
 This will take place on the day when God will judge men's secrets through Jesus Christ, as my gospel declares.

 就在 神借耶稣基督审判人隐秘事的日子,照着我的福音所言。


 罗2:17
 Now you, if you call yourself a Jew; if you rely on the law and brag about your relationship to God;

 你称为犹太人,又倚靠律法,且指着 神夸口;


 罗2:18
 if you know his will and approve of what is superior because you are instructed by the law;

 既从律法中受了教训,就晓得 神的旨意,也能分别是非(或作“也喜爱那美好的事”);


 罗2:19
 if you are convinced that you are a guide for the blind, a light for those who are in the dark,

 又深信自己是给瞎子领路的,是黑暗中人的光,


 罗2:20
 an instructor of the foolish, a teacher of infants, because you have in the law the embodiment of knowledge and truth--

 是蠢笨人的师傅,是小孩子的先生,在律法上有知识和真理的模范。


 罗2:21
 you, then, who teach others, do you not teach yourself? You who preach against stealing, do you steal?

 你既是教导别人,还不教导自己吗?你讲说人不可偷窃,自己还偷窃吗?


 罗2:22
 You who say that people should not commit adultery, do you commit adultery? You who abhor idols, do you rob temples?

 你说人不可奸淫,自己还奸淫吗?你厌恶偶像,自己还偷窃庙中之物吗?


 罗2:23
 You who brag about the law, do you dishonor God by breaking the law?

 你指着律法夸口,自己倒犯律法玷辱 神吗?


 罗2:24
 As it is written: "God's name is blasphemed among the Gentiles because of you."

 神的名在外邦人中,因你们受了亵渎,正如经上所记的。


 罗2:25
 Circumcision has value if you observe the law, but if you break the law, you have become as though you had not been circumcised.

 你若是行律法的,割礼固然于你有益;若是犯律法的,你的割礼就算不得割礼。


 罗2:26
 If those who are not circumcised keep the law's requirements, will they not be regarded as though they were circumcised?

 所以那未受割礼的,若遵守律法的条例,他虽然未受割礼,岂不算是有割礼吗?


 罗2:27
 The one who is not circumcised physically and yet obeys the law will condemn you who, even though you have the written code and circumcision, are a lawbreaker.

 而且那本来未受割礼的,若能全守律法,岂不是要审判你这有仪文和割礼竟犯律法的人吗?


 罗2:28
 A man is not a Jew if he is only one outwardly, nor is circumcision merely outward and physical.

 因为外面作犹太人的,不是真犹太人;外面肉身的割礼,也不是真割礼。


 罗2:29
 No, a man is a Jew if he is one inwardly; and circumcision is circumcision of the heart, by the Spirit, not by the written code. Such a man's praise is not from men, but from God.

 惟有里面作的,才是真犹太人;真割礼也是心里的,在乎灵,不在乎仪文。这人的称赞不是从人来的,乃是从 神来的。


 \section{罗马书第3章}
 罗3:1
 What advantage, then, is there in being a Jew, or what value is there in circumcision?

 这样说来,犹太人有什么长处,割礼有什么益处呢?


 罗3:2
 Much in every way! First of all, they have been entrusted with the very words of God.

 凡事大有好处,第一是 神的圣言交托他们。


 罗3:3
 What if some did not have faith? Will their lack of faith nullify God's faithfulness?

 即便有不信的,这有何妨呢?难道他们的不信,就废掉 神的信吗?


 罗3:4
 Not at all! Let God be true, and every man a liar. As it is written: "So that you may be proved right when you speak and prevail when you judge."

 断乎不能!不如说, 神是真实的,人都是虚谎的。如经上所记:“你责备人的时候,显为公义;被人议论的时候,可以得胜。”


 罗3:5
 But if our unrighteousness brings out God's righteousness more clearly, what shall we say? That God is unjust in bringing his wrath on us? (I am using a human argument.)

 我且照着人的常话说,我们的不义若显出 神的义来,我们可以怎么说呢? 神降怒,是他不义吗?


 罗3:6
 Certainly not! If that were so, how could God judge the world?

 断乎不是!若是这样, 神怎能审判世界呢?


 罗3:7
 Someone might argue, "If my falsehood enhances God's truthfulness and so increases his glory, why am I still condemned as a sinner?"

 若 神的真实,因我的虚谎越发显出他的荣耀,为什么我还受审判,好像罪人呢?


 罗3:8
 Why not say--as we are being slanderously reported as saying and as some claim that we say--"Let us do evil that good may result"? Their condemnation is deserved.

 为什么不说,我们可以作恶以成善呢?这是毁谤我们的人说我们有这话。这等人定罪是该当的。


 罗3:9
 What shall we conclude then? Are we any better? Not at all! We have already made the charge that Jews and Gentiles alike are all under sin.

 这却怎么样呢?我们比他们强吗?决不是的!因我们已经证明:犹太人和希利尼人都在罪恶之下。


 罗3:10
 As it is written: "There is no one righteous, not even one;

 就如经上所记:“没有义人,连一个也没有!


 罗3:11
 there is no one who understands, no one who seeks God.

 没有明白的,没有寻求 神的;


 罗3:12
 All have turned away, they have together become worthless; there is no one who does good, not even one."

 都是偏离正路,一同变为无用。没有行善的,连一个也没有!”


 罗3:13
 "Their throats are open graves; their tongues practice deceit." "The poison of vipers is on their lips."

 “他们的喉咙是敞开的坟墓;他们用舌头弄诡诈。”“嘴唇里有虺蛇的毒气。”


 罗3:14
 "Their mouths are full of cursing and bitterness."

 “满口是咒骂苦毒。”


 罗3:15
 "Their feet are swift to shed blood;

 “杀人流血,他们的脚飞跑,


 罗3:16
 ruin and misery mark their ways,

 所经过的路,便行残害暴虐的事;


 罗3:17
 and the way of peace they do not know."

 平安的路,他们未曾知道。”


 罗3:18
 "There is no fear of God before their eyes."

 “他们眼中不怕 神。”


 罗3:19
 Now we know that whatever the law says, it says to those who are under the law, so that every mouth may be silenced and the whole world held accountable to God.

 我们晓得律法上的话,都是对律法以下之人说的,好塞住各人的口,叫普世的人都伏在 神审判之下。


 罗3:20
 Therefore no one will be declared righteous in his sight by observing the law; rather, through the law we become conscious of sin.

 所以凡有血气的,没有一个因行律法能在 神面前称义,因为律法本是叫人知罪。


 罗3:21
 But now a righteousness from God, apart from law, has been made known, to which the Law and the Prophets testify.

 但如今, 神的义在律法以外已经显明出来,有律法和先知为证:


 罗3:22
 This righteousness from God comes through faith in Jesus Christ to all who believe. There is no difference,

 就是 神的义,因信耶稣基督加给一切相信的人,并没有分别。


 罗3:23
 for all have sinned and fall short of the glory of God,

 因为世人都犯了罪,亏缺了 神的荣耀;


 罗3:24
 and are justified freely by his grace through the redemption that came by Christ Jesus.

 如今却蒙 神的恩典,因基督耶稣的救赎,就白白地称义。


 罗3:25
 God presented him as a sacrifice of atonement, through faith in his blood. He did this to demonstrate his justice, because in his forbearance he had left the sins committed beforehand unpunished--

 神设立耶稣作挽回祭,是凭着耶稣的血,借着人的信,要显明 神的义。因为他用忍耐的心,宽容人先时所犯的罪,


 罗3:26
 he did it to demonstrate his justice at the present time, so as to be just and the one who justifies those who have faith in Jesus.

 好在今时显明他的义,使人知道他自己为义,也称信耶稣的人为义。


 罗3:27
 Where, then, is boasting? It is excluded. On what principle? On that of observing the law? No, but on that of faith.

 既是这样,哪里能夸口呢?没有可夸的了!用何法没有的呢?是用立功之法吗?不是,乃用信主之法。


 罗3:28
 For we maintain that a man is justified by faith apart from observing the law.

 所以(有古卷作“因为”)我们看定了:人称义是因着信,不在乎遵行律法。


 罗3:29
 Is God the God of Jews only? Is he not the God of Gentiles too? Yes, of Gentiles too,

 难道 神只作犹太人的 神吗?不也是作外邦人的 神吗?是的,也作外邦人的 神。


 罗3:30
 since there is only one God, who will justify the circumcised by faith and the uncircumcised through that same faith.

 神既是一位,他就要因信称那受割礼的为义,也要因信称那未受割礼的为义。


 罗3:31
 Do we, then, nullify the law by this faith? Not at all! Rather, we uphold the law.

 这样,我们因信废了律法么?断乎不是!更是坚固律法。


 \section{罗马书第4章}
 罗4:1
 What then shall we say that Abraham, our forefather, discovered in this matter?

 如此说来,我们的祖宗亚伯拉罕凭着肉体得了什么呢?


 罗4:2
 If, in fact, Abraham was justified by works, he had something to boast about--but not before God.

 倘若亚伯拉罕是因行为称义,就有可夸的;只是在 神面前并无可夸。


 罗4:3
 What does the Scripture say? "Abraham believed God, and it was credited to him as righteousness."

 经上说什么呢?说:“亚伯拉罕信 神,这就算为他的义。”


 罗4:4
 Now when a man works, his wages are not credited to him as a gift, but as an obligation.

 做工的得工价,不算恩典,乃是该得的;


 罗4:5
 However, to the man who does not work but trusts God who justifies the wicked, his faith is credited as righteousness.

 惟有不做工的,只信称罪人为义的 神,他的信就算为义。


 罗4:6
 David says the same thing when he speaks of the blessedness of the man to whom God credits righteousness apart from works:

 正如大卫称那在行为以外蒙 神算为义的人是有福的。


 罗4:7
 "Blessed are they whose transgressions are forgiven, whose sins are covered.

 他说:“得赦免其过、遮盖其罪的,这人是有福的!


 罗4:8
 Blessed is the man whose sin the Lord will never count against him."

 主不算为有罪的,这人是有福的!”


 罗4:9
 Is this blessedness only for the circumcised, or also for the uncircumcised? We have been saying that Abraham's faith was credited to him as righteousness.

 如此看来,这福是单加给那受割礼的人吗?不也是加给那未受割礼的人吗?因我们所说,亚伯拉罕的信,就算为他的义。


 罗4:10
 Under what circumstances was it credited? Was it after he was circumcised, or before? It was not after, but before!

 是怎么算的呢?是在他受割礼的时候呢?是在他未受割礼的时候呢?不是在受割礼的时候,乃是在未受割礼的时候。


 罗4:11
 And he received the sign of circumcision, a seal of the righteousness that he had by faith while he was still uncircumcised. So then, he is the father of all who believe but have not been circumcised, in order that righteousness might be credited to them.

 并且他受了割礼的记号,作他未受割礼的时候因信称义的印证,叫他作一切未受割礼而信之人的父,使他们也算为义;


 罗4:12
 And he is also the father of the circumcised who not only are circumcised but who also walk in the footsteps of the faith that our father Abraham had before he was circumcised.

 又作受割礼之人的父,就是那些不但受割礼,并且按我们的祖宗亚伯拉罕,未受割礼而信之踪迹去行的人。


 罗4:13
 It was not through law that Abraham and his offspring received the promise that he would be heir of the world, but through the righteousness that comes by faith.

 因为 神应许亚伯拉罕和他后裔必得承受世界,不是因律法,乃是因信而得的义。


 罗4:14
 For if those who live by law are heirs, faith has no value and the promise is worthless,

 若是属乎律法的人才得为后嗣,信就归于虚空,应许也就废弃了。


 罗4:15
 because law brings wrath. And where there is no law there is no transgression.

 因为律法是惹动忿怒的。(或作叫人受刑的)哪里没有律法,哪里就没有过犯。


 罗4:16
 Therefore, the promise comes by faith, so that it may be by grace and may be guaranteed to all Abraham's offspring--not only to those who are of the law but also to those who are of the faith of Abraham. He is the father of us all.

 所以人得为后嗣是本乎信,因此就属乎恩,叫应许定然归给一切后裔,不但归给那属乎律法的,也归给那效法亚伯拉罕之信的。


 罗4:17
 As it is written: "I have made you a father of many nations." He is our father in the sight of God, in whom he believed--the God who gives life to the dead and calls things that are not as though they were.

 亚伯拉罕所信的,是那叫死人复活、使无变为有的 神,他在主面前作我们世人的父。如经上所记:“我已经立你作多国的父。”


 罗4:18
 Against all hope, Abraham in hope believed and so became the father of many nations, just as it had been said to him, "So shall your offspring be."

 他在无可指望的时候,因信仍有指望,就得以作多国的父,正如先前所说:“你的后裔将要如此。”


 罗4:19
 Without weakening in his faith, he faced the fact that his body was as good as dead--since he was about a hundred years old--and that Sarah's womb was also dead.

 他将近百岁的时候,虽然想到自己的身体如同已死,撒拉的生育已经断绝,他的信心还是不软弱。


 罗4:20
 Yet he did not waver through unbelief regarding the promise of God, but was strengthened in his faith and gave glory to God,

 并且仰望 神的应许,总没有因不信,心里起疑惑,反倒因信,心里得坚固,将荣耀归给 神。


 罗4:21
 being fully persuaded that God had power to do what he had promised.

 且满心相信 神所应许的必能作成。


 罗4:22
 This is why "it was credited to him as righteousness."

 所以这就算为他的义。


 罗4:23
 The words "it was credited to him" were written not for him alone,

 “算为他义”的这句话,不是单为他写的,


 罗4:24
 but also for us, to whom God will credit righteousness--for us who believe in him who raised Jesus our Lord from the dead.

 也是为我们将来得算为义之人写的,就是我们这信 神使我们的主耶稣从死里复活的人。


 罗4:25
 He was delivered over to death for our sins and was raised to life for our justification.

 耶稣被交给人,是为我们的过犯;复活,是为叫我们称义(或作“耶稣是为我们的过犯交付了,是为我们称义复活了”)。


 \section{罗马书第5章}
 罗5:1
 Therefore, since we have been justified through faith, we have peace with God through our Lord Jesus Christ,

 我们既因信称义,就藉着我们的主耶稣基督得与 神相和。


 罗5:2
 through whom we have gained access by faith into this grace in which we now stand. And we rejoice in the hope of the glory of God.

 我们又藉着他,因信得进入现在所站的这恩典中,并且欢欢喜喜盼望 神的荣耀。


 罗5:3
 Not only so, but we also rejoice in our sufferings, because we know that suffering produces perseverance;

 不但如此,就是在患难中也是欢欢喜喜的。因为知道患难生忍耐,


 罗5:4
 perseverance, character; and character, hope.

 忍耐生老练,老练生盼望;


 罗5:5
 And hope does not disappoint us, because God has poured out his love into our hearts by the Holy Spirit, whom he has given us.

 盼望不至于羞耻。因为所赐给我们的圣灵将 神的爱浇灌在我们心里。


 罗5:6
 You see, at just the right time, when we were still powerless, Christ died for the ungodly.

 因我们还软弱的时候,基督就按所定的日期为罪人死。


 罗5:7
 Very rarely will anyone die for a righteous man, though for a good man someone might possibly dare to die.

 为义人死,是少有的;为仁人死,或者有敢作的;


 罗5:8
 But God demonstrates his own love for us in this: While we were still sinners, Christ died for us.

 惟有基督在我们还作罪人的时候为我们死, 神的爱就在此向我们显明了。


 罗5:9
 Since we have now been justified by his blood, how much more shall we be saved from God's wrath through him!

 现在我们既靠着他的血称义,就更要藉着他免去 神的忿怒。


 罗5:10
 For if, when we were God's enemies, we were reconciled to him through the death of his Son, how much more, having been reconciled, shall we be saved through his life!

 因为我们作仇敌的时候,且藉着 神儿子的死得与 神和好;既已和好,就更要因他的生得救了。


 罗5:11
 Not only is this so, but we also rejoice in God through our Lord Jesus Christ, through whom we have now received reconciliation.

 不但如此,我们既藉着我主耶稣基督得与 神和好,也就藉着他以 神为乐。


 罗5:12
 Therefore, just as sin entered the world through one man, and death through sin, and in this way death came to all men, because all sinned--

 这就如罪是从一人入了世界,死又是从罪来的,于是死就临到众人,因为众人都犯了罪。


 罗5:13
 for before the law was given, sin was in the world. But sin is not taken into account when there is no law.

 没有律法之先,罪已经在世上;但没有律法,罪也不算罪。


 罗5:14
 Nevertheless, death reigned from the time of Adam to the time of Moses, even over those who did not sin by breaking a command, as did Adam, who was a pattern of the one to come.

 然而从亚当到摩西,死就作了王,连那些不与亚当犯一样罪过的,也在他的权下。亚当乃是那以后要来之人的预像。


 罗5:15
 But the gift is not like the trespass. For if the many died by the trespass of the one man, how much more did God's grace and the gift that came by the grace of the one man, Jesus Christ, overflow to the many!

 只是过犯不如恩赐。若因一人的过犯,众人都死了,何况 神的恩典,与那因耶稣基督一人恩典中的赏赐,岂不更加倍地临到众人吗?


 罗5:16
 Again, the gift of God is not like the result of the one man's sin: The judgment followed one sin and brought condemnation, but the gift followed many trespasses and brought justification.

 因一人犯罪就定罪,也不如恩赐;原来审判是由一人而定罪,恩赐乃是由许多过犯而称义。


 罗5:17
 For if, by the trespass of the one man, death reigned through that one man, how much more will those who receive God's abundant provision of grace and of the gift of righteousness reign in life through the one man, Jesus Christ.

 若因一人的过犯,死就因这一人作了王,何况那些受洪恩又蒙所赐之义的,岂不更要因耶稣基督一人在生命中作王吗?


 罗5:18
 Consequently, just as the result of one trespass was condemnation for all men, so also the result of one act of righteousness was justification that brings life for all men.

 如此说来,因一次的过犯,众人都被定罪;照样,因一次的义行,众人也就被称义得生命了。


 罗5:19
 For just as through the disobedience of the one man the many were made sinners, so also through the obedience of the one man the many will be made righteous.

 因一人的悖逆,众人成为罪人;照样,因一人的顺从,众人也成为义了。


 罗5:20
 The law was added so that the trespass might increase. But where sin increased, grace increased all the more,

 律法本是外添的,叫过犯显多;只是罪在哪里显多,恩典就更显多了。


 罗5:21
 so that, just as sin reigned in death, so also grace might reign through righteousness to bring eternal life through Jesus Christ our Lord.

 就如罪作王叫人死;照样,恩典也藉着义作王,叫人因我们的主耶稣基督得永生。


\section{ 罗马书第6章}
 \textbf{罗6:1}
 What shall we say, then? Shall we go on sinning so that grace may increase?

 这样,怎么说呢?我们可以仍在罪中,叫恩典显多吗?


\textbf{ 罗6:2}
 By no means! We died to sin; how can we live in it any longer?

 断乎不可!我们在罪上死了的人岂可仍在罪中活着呢?


 \textbf{罗6:3}
 Or don't you know that all of us who were baptized into Christ Jesus were baptized into his death?

 岂不知我们这受洗归入基督耶稣的人,是受洗归入他的死吗?


 \textbf{罗6:4}
 We were therefore buried with him through baptism into death in order that, just as Christ was raised from the dead through the glory of the Father, we too may live a new life.

 所以我们藉着洗礼归入死,和他一同埋葬,原是叫我们一举一动有新生的样式,像基督藉着父的荣耀从死里复活一样。


 \textbf{罗6:5}
 If we have been united with him like this in his death, we will certainly also be united with him in his resurrection.

 我们若在他死的形状上与他联合,也要在他复活的形状上与他联合。


 \textbf{罗6:6}
 For we know that our old self was crucified with him so that the body of sin might be done away with, that we should no longer be slaves to sin--

 因为知道我们的旧人和他同钉十字架,使罪身灭绝,叫我们不再作罪的奴仆,


 \textbf{罗6:7}
 because anyone who has died has been freed from sin.

 因为已死的人是脱离了罪。


 \textbf{罗6:8}
 Now if we died with Christ, we believe that we will also live with him.

 我们若是与基督同死,就信必与他同活,


 \textbf{罗6:9}
 For we know that since Christ was raised from the dead, he cannot die again; death no longer has mastery over him.

 因为知道基督既从死里复活,就不再死,死也不再作他的主了。


 \textbf{罗6:10}
 The death he died, he died to sin once for all; but the life he lives, he lives to God.

 他死是向罪死了,只有一次;他活是向 神活着。


 \textbf{罗6:11}
 In the same way, count yourselves dead to sin but alive to God in Christ Jesus.

 这样,你们向罪也当看自己是死的;向 神在基督耶稣里,却当看自己是活的。


 \textbf{罗6:12}
 Therefore do not let sin reign in your mortal body so that you obey its evil desires.

 所以,不要容罪在你们必死的身上作王,使你们顺从身子的私欲。


 \textbf{罗6:13}
 Do not offer the parts of your body to sin, as instruments of wickedness, but rather offer yourselves to God, as those who have been brought from death to life; and offer the parts of your body to him as instruments of righteousness.

 也不要将你们的肢体献给罪作不义的器具;倒要像从死里复活的人,将自己献给 神,并将肢体作义的器具献给 神。


 \textbf{罗6:14}
 For sin shall not be your master, because you are not under law, but under grace.

 罪必不能作你们的主,因你们不在律法之下,乃在恩典之下。


 \textbf{罗6:15}
 What then? Shall we sin because we are not under law but under grace? By no means!

 这却怎么样呢?我们在恩典之下,不在律法之下,就可以犯罪吗?断乎不可!


 \textbf{罗6:16}
 Don't you know that when you offer yourselves to someone to obey him as slaves, you are slaves to the one whom you obey--whether you are slaves to sin, which leads to death, or to obedience, which leads to righteousness?

 岂不晓得你们献上自己作奴仆,顺从谁,就作谁的奴仆吗?或作罪的奴仆,以至于死;或作顺命的奴仆,以至成义。


 \textbf{罗6:17}
 But thanks be to God that, though you used to be slaves to sin, you wholeheartedly obeyed the form of teaching to which you were entrusted.

 感谢 神!因为你们从前虽然作罪的奴仆,现今却从心里顺服了所传给你们道理的模范。


 \textbf{罗6:18}
 You have been set free from sin and have become slaves to righteousness.

 你们既从罪里得了释放,就作了义的奴仆。


 \textbf{罗6:19}
 I put this in human terms because you are weak in your natural selves. Just as you used to offer the parts of your body in slavery to impurity and to ever-increasing wickedness, so now offer them in slavery to righteousness leading to holiness.

 我因你们肉体的软弱,就照人的常话对你们说:你们从前怎样将肢体献给不洁不法作奴仆,以至于不法;现今也要照样将肢体献给义作奴仆,以至于成圣。


 \textbf{罗6:20}
 When you were slaves to sin, you were free from the control of righteousness.

 因为你们作罪之奴仆的时候,就不被义约束了。


 \textbf{罗6:21}
 What benefit did you reap at that time from the things you are now ashamed of? Those things result in death!

 你们现今所看为羞耻的事,当日有什么果子呢?那些事的结局就是死!


 \textbf{罗6:22}
 But now that you have been set free from sin and have become slaves to God, the benefit you reap leads to holiness, and the result is eternal life.

 但现今你们既从罪里得了释放,作了 神的奴仆,就有成圣的果子,那结局就是永生!


 \textbf{罗6:23}
 For the wages of sin is death, but the gift of God is eternal life in Christ Jesus our Lord.

 因为罪的工价乃是死;惟有 神的恩赐,在我们的主基督耶稣里,乃是永生。


\section{ 罗马书第7章}
 罗7:1
 Do you not know, brothers--for I am speaking to men who know the law--that the law has authority over a man only as long as he lives?

 弟兄们,我现在对明白律法的人说:你们岂不晓得律法管人是在活着的时候吗?


 罗7:2
 For example, by law a married woman is bound to her husband as long as he is alive, but if her husband dies, she is released from the law of marriage.

 就如女人有了丈夫,丈夫还活着,就被律法约束;丈夫若死了,就脱离了丈夫的律法。


 罗7:3
 So then, if she marries another man while her husband is still alive, she is called an adulteress. But if her husband dies, she is released from that law and is not an adulteress, even though she marries another man.

 所以丈夫活着,她若归于别人,便叫淫妇;丈夫若死了,她就脱离了丈夫的律法,虽然归于别人,也不是淫妇。


 罗7:4
 So, my brothers, you also died to the law through the body of Christ, that you might belong to another, to him who was raised from the dead, in order that we might bear fruit to God.

 我的弟兄们,这样说来,你们藉着基督的身体,在律法上也是死了,叫你们归于别人,就是归于那从死里复活的,叫我们结果子给 神。


 罗7:5
 For when we were controlled by the sinful nature, the sinful passions aroused by the law were at work in our bodies, so that we bore fruit for death.

 因为我们属肉体的时候,那因律法而生的恶欲就在我们肢体中发动,以致结成死亡的果子。


 罗7:6
 But now, by dying to what once bound us, we have been released from the law so that we serve in the new way of the Spirit, and not in the old way of the written code.

 但我们既然在捆我们的律法上死了,现今就脱离了律法,叫我们服事主,要按着心灵的新样,不按着仪文的旧样(心灵或作圣灵)。


 罗7:7
 What shall we say, then? Is the law sin? Certainly not! Indeed I would not have known what sin was except through the law. For I would not have known what coveting really was if the law had not said, "Do not covet."

 这样,我们可说什么呢?律法是罪吗?断乎不是!只是非因律法,我就不知何为罪。非律法说,“不可起贪心”,我就不知何为贪心。


 罗7:8
 But sin, seizing the opportunity afforded by the commandment, produced in me every kind of covetous desire. For apart from law, sin is dead.

 然而罪趁着机会,就藉着诫命叫诸般的贪心在我里头发动,因为没有律法,罪是死的。


 罗7:9
 Once I was alive apart from law; but when the commandment came, sin sprang to life and I died.

 我以前没有律法,是活着的;但是诫命来到,罪又活了,我就死了。


 罗7:10
 I found that the very commandment that was intended to bring life actually brought death.

 那本来叫人活的诫命,反倒叫我死,


 罗7:11
 For sin, seizing the opportunity afforded by the commandment, deceived me, and through the commandment put me to death.

 因为罪趁着机会,就藉着诫命引诱我,并且杀了我。


 罗7:12
 So then, the law is holy, and the commandment is holy, righteous and good.

 这样看来,律法是圣洁的,诫命也是圣洁、公义、良善的。


 罗7:13
 Did that which is good, then, become death to me? By no means! But in order that sin might be recognized as sin, it produced death in me through what was good, so that through the commandment sin might become utterly sinful.

 既然如此,那良善的是叫我死吗?断乎不是!叫我死的乃是罪。但罪藉着那良善的叫我死,就显出真是罪,叫罪因着诫命更显出是恶极了。


 罗7:14
 We know that the law is spiritual; but I am unspiritual, sold as a slave to sin.

 我们原晓得律法是属乎灵的,但我是属乎肉体的,是已经卖给罪了。


 罗7:15
 I do not understand what I do. For what I want to do I do not do, but what I hate I do.

 因为我所做的,我自己不明白;我所愿意的,我并不做;我所恨恶的,我倒去做。


 罗7:16
 And if I do what I do not want to do, I agree that the law is good.

 若我所做的,是我所不愿意的,我就应承律法是善的。


 罗7:17
 As it is, it is no longer I myself who do it, but it is sin living in me.

 既是这样,就不是我做的,乃是住在我里头的罪做的。


 罗7:18
 I know that nothing good lives in me, that is, in my sinful nature. For I have the desire to do what is good, but I cannot carry it out.

 我也知道在我里头,就是我肉体之中,没有良善。因为立志为善由得我,只是行出来由不得我。


 罗7:19
 For what I do is not the good I want to do; no, the evil I do not want to do--this I keep on doing.

 故此,我所愿意的善,我反不做;我所不愿意的恶,我倒去做。


 罗7:20
 Now if I do what I do not want to do, it is no longer I who do it, but it is sin living in me that does it.

 若我去做所不愿意做的,就不是我做的,乃是住在我里头的罪做的。


 罗7:21
 So I find this law at work: When I want to do good, evil is right there with me.

 我觉得有个律,就是我愿意为善的时候,便有恶与我同在。


 罗7:22
 For in my inner being I delight in God's law;

 因为按着我里面的意思(原文作“人”),我是喜欢 神的律;


 罗7:23
 but I see another law at work in the members of my body, waging war against the law of my mind and making me a prisoner of the law of sin at work within my members.

 但我觉得肢体中另有个律和我心中的律交战,把我掳去,叫我附从那肢体中犯罪的律。


 罗7:24
 What a wretched man I am! Who will rescue me from this body of death?

 我真是苦啊!谁能救我脱离这取死的身体呢?


 罗7:25
 Thanks be to God--through Jesus Christ our Lord! So then, I myself in my mind am a slave to God's law, but in the sinful nature a slave to the law of sin.

 感谢 神!靠着我们的主耶稣基督就能脱离了。这样看来,我以内心顺服 神的律,我肉体却顺服罪的律了。


\section{ 罗马书第8章}
 罗8:1
 Therefore, there is now no condemnation for those who are in Christ Jesus,

 如今,那些在基督耶稣里的就不定罪了。


 罗8:2
 because through Christ Jesus the law of the Spirit of life set me free from the law of sin and death.

 因为赐生命圣灵的律在基督耶稣里释放了我,使我脱离罪和死的律了。


 罗8:3
 For what the law was powerless to do in that it was weakened by the sinful nature, God did by sending his own Son in the likeness of sinful man to be a sin offering. And so he condemned sin in sinful man,

 律法既因肉体软弱,有所不能行的, 神就差遣自己的儿子成为罪身的形状,作了赎罪祭,在肉体中定了罪案,


 罗8:4
 in order that the righteous requirements of the law might be fully met in us, who do not live according to the sinful nature but according to the Spirit.

 使律法的义成就在我们这不随从肉体,只随从圣灵的人身上。


 罗8:5
 Those who live according to the sinful nature have their minds set on what that nature desires; but those who live in accordance with the Spirit have their minds set on what the Spirit desires.

 因为随从肉体的人,体贴肉体的事;随从圣灵的人,体贴圣灵的事。


 罗8:6
 The mind of sinful man is death, but the mind controlled by the Spirit is life and peace;

 体贴肉体的就是死;体贴圣灵的乃是生命平安。


 罗8:7
 the sinful mind is hostile to God. It does not submit to God's law, nor can it do so.

 原来体贴肉体的,就是与 神为仇,因为不服 神的律法,也是不能服。


 罗8:8
 Those controlled by the sinful nature cannot please God.

 而且属肉体的人不能得 神的喜欢。


 罗8:9
 You, however, are controlled not by the sinful nature but by the Spirit, if the Spirit of God lives in you. And if anyone does not have the Spirit of Christ, he does not belong to Christ.

 如果 神的灵住在你们心里,你们就不属肉体,乃属圣灵了。人若没有基督的灵,就不是属基督的。


 罗8:10
 But if Christ is in you, your body is dead because of sin, yet your spirit is alive because of righteousness.

 基督若在你们心里,身体就因罪而死,心灵却因义而活。


 罗8:11
 And if the Spirit of him who raised Jesus from the dead is living in you, he who raised Christ from the dead will also give life to your mortal bodies through his Spirit, who lives in you.

 然而叫耶稣从死里复活者的灵,若住在你们心里,那叫基督耶稣从死里复活的,也必藉着住在你们心里的圣灵,使你们必死的身体又活过来。


 罗8:12
 Therefore, brothers, we have an obligation--but it is not to the sinful nature, to live according to it.

 弟兄们,这样看来,我们并不是欠肉体的债,去顺从肉体活着。


 罗8:13
 For if you live according to the sinful nature, you will die; but if by the Spirit you put to death the misdeeds of the body, you will live,

 你们若顺从肉体活着,必要死;若靠着圣灵治死身体的恶行,必要活着。


 罗8:14
 because those who are led by the Spirit of God are sons of God.

 因为凡被 神的灵引导的,都是 神的儿子。


 罗8:15
 For you did not receive a spirit that makes you a slave again to fear, but you received the Spirit of sonship. And by him we cry, "Abba, Father."

 你们所受的不是奴仆的心,仍旧害怕;所受的乃是儿子的心,因此我们呼叫:“阿爸,父!”


 罗8:16
 The Spirit himself testifies with our spirit that we are God's children.

 圣灵与我们的心同证我们是 神的儿女;


 罗8:17
 Now if we are children, then we are heirs--heirs of God and co-heirs with Christ, if indeed we share in his sufferings in order that we may also share in his glory.

 既是儿女,便是后嗣,就是 神的后嗣,和基督同作后嗣。如果我们和他一同受苦,也必和他一同得荣耀。


 罗8:18
 I consider that our present sufferings are not worth comparing with the glory that will be revealed in us.

 我想,现在的苦楚若比起将来要显于我们的荣耀,就不足介意了。


 罗8:19
 The creation waits in eager expectation for the sons of God to be revealed.

 受造之物切望等候 神的众子显出来。


 罗8:20
 For the creation was subjected to frustration, not by its own choice, but by the will of the one who subjected it, in hope

 因为受造之物服在虚空之下,不是自己愿意,乃是因那叫他如此的。


 罗8:21
 that the creation itself will be liberated from its bondage to decay and brought into the glorious freedom of the children of God.

 但受造之物仍然指望脱离败坏的辖制,得享 神儿女自由的荣耀(“享”原文作“入”)。


 罗8:22
 We know that the whole creation has been groaning as in the pains of childbirth right up to the present time.

 我们知道一切受造之物一同叹息、劳苦,直到如今。


 罗8:23
 Not only so, but we ourselves, who have the firstfruits of the Spirit, groan inwardly as we wait eagerly for our adoption as sons, the redemption of our bodies.

 不但如此,就是我们这有圣灵初结果子的,也是自己心里叹息,等候得着儿子的名分,乃是我们的身体得赎。


 罗8:24
 For in this hope we were saved. But hope that is seen is no hope at all. Who hopes for what he already has?

 我们得救是在乎盼望;只是所见的盼望不是盼望,谁还盼望他所见的呢?(有古卷作“人所看见的何必再盼望呢?”)


 罗8:25
 But if we hope for what we do not yet have, we wait for it patiently.

 但我们若盼望那所不见的,就必忍耐等候。


 罗8:26
 In the same way, the Spirit helps us in our weakness. We do not know what we ought to pray for, but the Spirit himself intercedes for us with groans that words cannot express.

 况且,我们的软弱有圣灵帮助,我们本不晓得当怎样祷告,只是圣灵亲自用说不出来的叹息替我们祷告。


 罗8:27
 And he who searches our hearts knows the mind of the Spirit, because the Spirit intercedes for the saints in accordance with God's will.

 鉴察人心的,晓得圣灵的意思,因为圣灵照着 神的旨意替圣徒祈求。


 罗8:28
 And we know that in all things God works for the good of those who love him, who have been called according to his purpose.

 我们晓得万事都互相效力,叫爱 神的人得益处,就是按他旨意被召的人。


 罗8:29
 For those God foreknew he also predestined to be conformed to the likeness of his Son, that he might be the firstborn among many brothers.

 因为他预先所知道的人,就预先定下效法他儿子的模样,使他儿子在许多弟兄中作长子。


 罗8:30
 And those he predestined, he also called; those he called, he also justified; those he justified, he also glorified.

 预先所定下的人又召他们来;所召来的人又称他们为义;所称为义的人又叫他们得荣耀。


 罗8:31
 What, then, shall we say in response to this? If God is for us, who can be against us?

 既是这样,还有什么说的呢? 神若帮助我们,谁能敌挡我们呢?


 罗8:32
 He who did not spare his own Son, but gave him up for us all--how will he not also, along with him, graciously give us all things?

 神既不爱惜自己的儿子为我们众人舍了,岂不也把万物和他一同白白地赐给我们吗?


 罗8:33
 Who will bring any charge against those whom God has chosen? It is God who justifies.

 谁能控告 神所拣选的人呢?有 神称他们为义了(或作“是称他们为义的 神吗?”)。


 罗8:34
 Who is he that condemns? Christ Jesus, who died--more than that, who was raised to life--is at the right hand of God and is also interceding for us.

 谁能定他们的罪呢?有基督耶稣已经死了,而且从死里复活,现今在 神的右边,也替我们祈求。(“有基督云云”,或作“是已经死了,而且从死里复活,现今在 神的右边,也替我们祈求的基督耶稣吗?”)


 罗8:35
 Who shall separate us from the love of Christ? Shall trouble or hardship or persecution or famine or nakedness or danger or sword?

 谁能使我们与基督的爱隔绝呢?难道是患难吗?是困苦吗?是逼迫吗?是饥饿吗?是赤身露体吗?是危险吗?是刀剑吗?


 罗8:36
 As it is written: "For your sake we face death all day long; we are considered as sheep to be slaughtered."

 如经上所记:“我们为你的缘故终日被杀,人看我们如将宰的羊。”


 罗8:37
 No, in all these things we are more than conquerors through him who loved us.

 然而,靠着爱我们的主,在这一切的事上已经得胜有余了。


 罗8:38
 For I am convinced that neither death nor life, neither angels nor demons, neither the present nor the future, nor any powers,

 因为我深信无论是死、是生,是天使、是掌权的,是有能的,是现在的事、是将来的事,


 罗8:39
 neither height nor depth, nor anything else in all creation, will be able to separate us from the love of God that is in Christ Jesus our Lord.

 是高处的、是低处的,是别的受造之物,都不能叫我们与 神的爱隔绝;这爱是在我们的主基督耶稣里的。


\section{ 罗马书第9章}
 罗9:1
 I speak the truth in Christ--I am not lying, my conscience confirms it in the Holy Spirit--

 我在基督里说真话,并不谎言,有我良心被圣灵感动,给我作见证。


 罗9:2
 I have great sorrow and unceasing anguish in my heart.

 我是大有忧愁,心里时常伤痛;


 罗9:3
 For I could wish that I myself were cursed and cut off from Christ for the sake of my brothers, those of my own race,

 为我弟兄、我骨肉之亲,就是自己被咒诅,与基督分离,我也愿意。


 罗9:4
 the people of Israel. Theirs is the adoption as sons; theirs the divine glory, the covenants, the receiving of the law, the temple worship and the promises.

 他们是以色列人,那儿子的名分、荣耀、诸约、律法、礼仪、应许,都是他们的。


 罗9:5
 Theirs are the patriarchs, and from them is traced the human ancestry of Christ, who is God over all, forever praised! Amen.

 列祖就是他们的祖宗,按肉体说,基督也是从他们出来的,他是在万有之上,永远可称颂的 神。阿们!


 罗9:6
 It is not as though God's word had failed. For not all who are descended from Israel are Israel.

 这不是说 神的话落了空,因为从以色列生的,不都是以色列人;


 罗9:7
 Nor because they are his descendants are they all Abraham's children. On the contrary, "It is through Isaac that your offspring will be reckoned."

 也不因为是亚伯拉罕的后裔,就都作他的儿女;惟独“从以撒生的,才要称为你的后裔。”


 罗9:8
 In other words, it is not the natural children who are God's children, but it is the children of the promise who are regarded as Abraham's offspring.

 这就是说,肉身所生的儿女不是 神的儿女;惟独那应许的儿女才算是后裔。


 罗9:9
 For this was how the promise was stated: "At the appointed time I will return, and Sarah will have a son."

 因为所应许的话是这样说:“到明年这时候我要来,撒拉必生一个儿子。”


 罗9:10
 Not only that, but Rebekah's children had one and the same father, our father Isaac.

 不但如此,还有利百加,既从一个人,就是从我们的祖宗以撒怀了孕,


 罗9:11
 Yet, before the twins were born or had done anything good or bad--in order that God's purpose in election might stand:

 (双子还没有生下来,善恶还没有作出来,只因要显明 神拣选人的旨意,不在乎人的行为,乃在乎召人的主。)


 罗9:12
 not by works but by him who calls--she was told, "The older will serve the younger."

 神就对利百加说:“将来大的要服事小的。”


 罗9:13
 Just as it is written: "Jacob I loved, but Esau I hated."

 正如经上所记:“雅各是我所爱的,以扫是我所恶的。”


 罗9:14
 What then shall we say? Is God unjust? Not at all!

 这样,我们可说什么呢?难道 神有什么不公平吗?断乎没有!


 罗9:15
 For he says to Moses, "I will have mercy on whom I have mercy, and I will have compassion on whom I have compassion."

 因他对摩西说:“我要怜悯谁,就怜悯谁;要恩待谁,就恩待谁。”


 罗9:16
 It does not, therefore, depend on man's desire or effort, but on God's mercy.

 据此看来,这不在乎那定意的,也不在乎那奔跑的,只在乎发怜悯的 神。


 罗9:17
 For the Scripture says to Pharaoh: "I raised you up for this very purpose, that I might display my power in you and that my name might be proclaimed in all the earth."

 因为经上有话向法老说:“我将你兴起来,特要在你身上彰显我的权能,并要使我的名传遍天下。”


 罗9:18
 Therefore God has mercy on whom he wants to have mercy, and he hardens whom he wants to harden.

 如此看来, 神要怜悯谁,就怜悯谁;要叫谁刚硬,就叫谁刚硬。


 罗9:19
 One of you will say to me: "Then why does God still blame us? For who resists his will?"

 这样,你必对我说:“他为什么还指责人呢?有谁抗拒他的旨意呢?”


 罗9:20
 But who are you, O man, to talk back to God? "Shall what is formed say to him who formed it, 'Why did you make me like this?'"

 你这个人哪,你是谁,竟敢向 神强嘴呢?受造之物岂能对造他的说:“你为什么这样造我呢?”


 罗9:21
 Does not the potter have the right to make out of the same lump of clay some pottery for noble purposes and some for common use?

 窑匠难道没有权柄从一团泥里拿一块做成贵重的器皿,又拿一块做成卑贱的器皿吗?


 罗9:22
 What if God, choosing to show his wrath and make his power known, bore with great patience the objects of his wrath--prepared for destruction?

 倘若 神要显明他的忿怒,彰显他的权能,就多多忍耐宽容那可怒、预备遭毁灭的器皿;


 罗9:23
 What if he did this to make the riches of his glory known to the objects of his mercy, whom he prepared in advance for glory--

 又要将他丰盛的荣耀彰显在那蒙怜悯、早预备得荣耀的器皿上。


 罗9:24
 even us, whom he also called, not only from the Jews but also from the Gentiles?

 这器皿就是我们被 神所召的,不但是从犹太人中,也是从外邦人中。这有什么不可呢?


 罗9:25
 As he says in Hosea: "I will call them 'my people' who are not my people; and I will call her 'my loved one' who is not my loved one,"

 就像 神在何西阿书上说:“那本来不是我子民的,我要称为‘我的子民’;本来不是蒙爱的,我要称为‘蒙爱的’。


 罗9:26
 and, "It will happen that in the very place where it was said to them, 'You are not my people,' they will be called 'sons of the living God.'"

 从前在什么地方对他们说,‘你们不是我的子民’,将来就在那里称他们为‘永生 神的儿子’。”


 罗9:27
 Isaiah cries out concerning Israel: "Though the number of the Israelites be like the sand by the sea, only the remnant will be saved.

 以赛亚指着以色列人喊着说:“以色列人虽多如海沙,得救的不过是剩下的余数。


 罗9:28
 For the Lord will carry out his sentence on earth with speed and finality."

 因为主要在世上施行他的话,叫他的话都成全,速速地完结。”


 罗9:29
 It is just as Isaiah said previously: "Unless the Lord Almighty had left us descendants, we would have become like Sodom, we would have been like Gomorrah."

 又如以赛亚先前说过:“若不是万军之主给我们存留余种,我们早已像所多玛、蛾摩拉的样子了。”


 罗9:30
 What then shall we say? That the Gentiles, who did not pursue righteousness, have obtained it, a righteousness that is by faith;

 这样,我们可说什么呢?那本来不追求义的外邦人反得了义,就是因信而得的义;


 罗9:31
 but Israel, who pursued a law of righteousness, has not attained it.

 但以色列人追求律法的义,反得不着律法的义。


 罗9:32
 Why not? Because they pursued it not by faith but as if it were by works. They stumbled over the "stumbling stone."

 这是什么缘故呢?是因为他们不凭着信心求,只凭着行为求,他们正跌在那绊脚石上。


 罗9:33
 As it is written: "See, I lay in Zion a stone that causes men to stumble and a rock that makes them fall, and the one who trusts in him will never be put to shame."

 就如经上所记:“我在锡安放一块绊脚的石头,跌人的磐石;信靠他的人必不至于羞愧。”


 \section{罗马书第10章}
 罗10:1
 Brothers, my heart's desire and prayer to God for the Israelites is that they may be saved.

 弟兄们,我心里所愿的,向 神所求的,是要以色列人得救。


 罗10:2
 For I can testify about them that they are zealous for God, but their zeal is not based on knowledge.

 我可以证明他们向 神有热心,但不是按着真知识。


 罗10:3
 Since they did not know the righteousness that comes from God and sought to establish their own, they did not submit to God's righteousness.

 因为不知道 神的义,想要立自己的义,就不服 神的义了。


 罗10:4
 Christ is the end of the law so that there may be righteousness for everyone who believes.

 律法的总结就是基督,使凡信他的都得着义。


 罗10:5
 Moses describes in this way the righteousness that is by the law: "The man who does these things will live by them."

 摩西写着说:“人若行那出于律法的义,就必因此活着。”


 罗10:6
 But the righteousness that is by faith says: "Do not say in your heart, 'Who will ascend into heaven?'" (that is, to bring Christ down)

 惟有出于信心的义如此说:“你不要心里说:‘谁要升到天上去呢?’(就是要领下基督来。)


 罗10:7
 "or 'Who will descend into the deep?'" (that is, to bring Christ up from the dead).

 ‘谁要下到阴间去呢?’(就是要领基督从死里上来。)”


 罗10:8
 But what does it say? "The word is near you; it is in your mouth and in your heart," that is, the word of faith we are proclaiming:

 他到底怎么说呢?他说:“这道离你不远,正在你口里,在你心里。”就是我们所传信主的道。


 罗10:9
 That if you confess with your mouth, "Jesus is Lord," and believe in your heart that God raised him from the dead, you will be saved.

 你若口里认耶稣为主,心里信 神叫他从死里复活,就必得救。


 罗10:10
 For it is with your heart that you believe and are justified, and it is with your mouth that you confess and are saved.

 因为人心里相信,就可以称义;口里承认,就可以得救。


 罗10:11
 As the Scripture says, "Anyone who trusts in him will never be put to shame."

 经上说:“凡信他的人,必不至于羞愧。”


 罗10:12
 For there is no difference between Jew and Gentile--the same Lord is Lord of all and richly blesses all who call on him,

 犹太人和希利尼人并没有分别;因为众人同有一位主,他也厚待一切求告他的人。


 罗10:13
 for, "Everyone who calls on the name of the Lord will be saved."

 因为“凡求告主名的,就必得救。”


 罗10:14
 How, then, can they call on the one they have not believed in? And how can they believe in the one of whom they have not heard? And how can they hear without someone preaching to them?

 然而人未曾信他,怎能求他呢?未曾听见他,怎能信他呢?没有传道的,怎能听见呢?


 罗10:15
 And how can they preach unless they are sent? As it is written, "How beautiful are the feet of those who bring good news!"

 若没有奉差遣,怎能传道呢?如经上所记:“报福音传喜信的人,他们的脚踪何等佳美!”


 罗10:16
 But not all the Israelites accepted the good news. For Isaiah says, "Lord, who has believed our message?"

 只是人没有都听从福音。因为以赛亚说:“主啊,我们所传的有谁信呢?”


 罗10:17
 Consequently, faith comes from hearing the message, and the message is heard through the word of Christ.

 可见信道是从听道来的,听道是从基督的话来的。


 罗10:18
 But I ask: Did they not hear? Of course they did: "Their voice has gone out into all the earth, their words to the ends of the world."

 但我说,人没有听见吗?诚然听见了。“他们的声音传遍天下,他们的言语传到地极。”


 罗10:19
 Again I ask: Did Israel not understand? First, Moses says, "I will make you envious by those who are not a nation; I will make you angry by a nation that has no understanding."

 我再说,以色列人不知道吗?先有摩西说:“我要用那不成子民的,惹动你们的愤恨;我要用那无知的民,触动你们的怒气。”


 罗10:20
 And Isaiah boldly says, "I was found by those who did not seek me; I revealed myself to those who did not ask for me."

 又有以赛亚放胆说:“没有寻找我的,我叫他们遇见;没有访问我的,我向他们显现。”


 罗10:21
 But concerning Israel he says, "All day long I have held out my hands to a disobedient and obstinate people."

 至于以色列人,他说:“我整天伸手招呼那悖逆顶嘴的百姓。”


 \section{罗马书第11章}
 罗11:1
 I ask then: Did God reject his people? By no means! I am an Israelite myself, a descendant of Abraham, from the tribe of Benjamin.

 我且说, 神弃绝了他的百姓吗?断乎没有!因为我也是以色列人,亚伯拉罕的后裔,属便雅悯支派的。


 罗11:2
 God did not reject his people, whom he foreknew. Don't you know what the Scripture says in the passage about Elijah--how he appealed to God against Israel:

 神并没有弃绝他预先所知道的百姓。你们岂不晓得经上论到以利亚是怎么说的呢?他在 神面前怎样控告以色列人说:


 罗11:3
 "Lord, they have killed your prophets and torn down your altars; I am the only one left, and they are trying to kill me"?

 “主啊,他们杀了你的先知,拆了你的祭坛,只剩下我一个人,他们还要寻索我的命。”


 罗11:4
 And what was God's answer to him? "I have reserved for myself seven thousand who have not bowed the knee to Baal."

 神的回话是怎么说的呢?他说:“我为自己留下七千人,是未曾向巴力屈膝的。”


 罗11:5
 So too, at the present time there is a remnant chosen by grace.

 如今也是这样,照着拣选的恩典,还有所留的余数。


 罗11:6
 And if by grace, then it is no longer by works; if it were, grace would no longer be grace.

 既是出于恩典,就不在乎行为,不然,恩典就不是恩典了。


 罗11:7
 What then? What Israel sought so earnestly it did not obtain, but the elect did. The others were hardened,

 这是怎么样呢?以色列人所求的,他们没有得着,惟有蒙拣选的人得着了,其余的就成了顽梗不化的。


 罗11:8
 as it is written: "God gave them a spirit of stupor, eyes so that they could not see and ears so that they could not hear, to this very day."

 如经上所记:“ 神给他们昏迷的心,眼睛不能看见,耳朵不能听见,直到今日。”


 罗11:9
 And David says: "May their table become a snare and a trap, a stumbling block and a retribution for them.

 大卫也说:“愿他们的筵席变为网罗,变为机槛,变为绊脚石,作他们的报应;


 罗11:10
 May their eyes be darkened so they cannot see, and their backs be bent forever."

 愿他们的眼睛昏蒙,不得看见;愿你时常弯下他们的腰。”


 罗11:11
 Again I ask: Did they stumble so as to fall beyond recovery? Not at all! Rather, because of their transgression, salvation has come to the Gentiles to make Israel envious.

 我且说,他们失脚是要他们跌倒吗?断乎不是!反倒因他们的过失,救恩便临到外邦人,要激动他们发愤。


 罗11:12
 But if their transgression means riches for the world, and their loss means riches for the Gentiles, how much greater riches will their fullness bring!

 若他们的过失,为天下的富足,他们的缺乏,为外邦人的富足,何况他们的丰满呢?。


 罗11:13
 I am talking to you Gentiles. Inasmuch as I am the apostle to the Gentiles, I make much of my ministry

 我对你们外邦人说这话,因我是外邦人的使徒,所以敬重我的职分(“敬重”原文作“荣耀”)。


 罗11:14
 in the hope that I may somehow arouse my own people to envy and save some of them.

 或者可以激动我骨肉之亲发愤,好救他们一些人。


 罗11:15
 For if their rejection is the reconciliation of the world, what will their acceptance be but life from the dead?

 若他们被丢弃,天下就得与 神和好,他们被收纳,岂不是死而复生吗?


 罗11:16
 If the part of the dough offered as firstfruits is holy, then the whole batch is holy; if the root is holy, so are the branches.

 所献的新面若是圣洁,全团也就圣洁了;树根若是圣洁,树枝也就圣洁了。


 罗11:17
 If some of the branches have been broken off, and you, though a wild olive shoot, have been grafted in among the others and now share in the nourishing sap from the olive root,

 若有几根枝子被折下来,你这野橄榄得接在其中,一同得着橄榄根的肥汁,


 罗11:18
 do not boast over those branches. If you do, consider this: You do not support the root, but the root supports you.

 你就不可向旧枝子夸口;若是夸口,当知道不是你托着根,乃是根托着你。


 罗11:19
 You will say then, "Branches were broken off so that I could be grafted in."

 你若说:“那枝子被折下来是特为叫我接上。”


 罗11:20
 Granted. But they were broken off because of unbelief, and you stand by faith. Do not be arrogant, but be afraid.

 不错!他们因为不信,所以被折下来;你因为信,所以立得住;你不可自高,反要惧怕。


 罗11:21
 For if God did not spare the natural branches, he will not spare you either.

 神既不爱惜原来的枝子,也必不爱惜你。


 罗11:22
 Consider therefore the kindness and sternness of God: sternness to those who fell, but kindness to you, provided that you continue in his kindness. Otherwise, you also will be cut off.

 可见 神的恩慈和严厉,向那跌倒的人是严厉的,向你是有恩慈的。只要你长久在他的恩慈里,不然,你也要被砍下来。


 罗11:23
 And if they do not persist in unbelief, they will be grafted in, for God is able to graft them in again.

 而且他们若不是长久不信,仍要被接上,因为 神能够把他们从新接上。


 罗11:24
 After all, if you were cut out of an olive tree that is wild by nature, and contrary to nature were grafted into a cultivated olive tree, how much more readily will these, the natural branches, be grafted into their own olive tree!

 你是从那天生的野橄榄上砍下来的,尚且逆着性得接在好橄榄上,何况这本树的枝子要接在本树上呢!


 罗11:25
 I do not want you to be ignorant of this mystery, brothers, so that you may not be conceited: Israel has experienced a hardening in part until the full number of the Gentiles has come in.

 弟兄们,我不愿意你们不知道这奥秘(恐怕你们自以为聪明):就是以色列人有几分是硬心的,等到外邦人的数目添满了,


 罗11:26
 And so all Israel will be saved, as it is written: "The deliverer will come from Zion; he will turn godlessness away from Jacob.

 于是以色列全家都要得救。如经上所记:“必有一位救主从锡安出来,要消除雅各家的一切罪恶。”


 罗11:27
 And this is my covenant with them when I take away their sins."

 又说:“我除去他们罪的时候,这就是我与他们所立的约。”


 罗11:28
 As far as the gospel is concerned, they are enemies on your account; but as far as election is concerned, they are loved on account of the patriarchs,

 就着福音说,他们为你们的缘故是仇敌;就着拣选说,他们为列祖的缘故是蒙爱的。


 罗11:29
 for God's gifts and his call are irrevocable.

 因为 神的恩赐和选召是没有后悔的。


 罗11:30
 Just as you who were at one time disobedient to God have now received mercy as a result of their disobedience,

 你们从前不顺服 神,如今因他们的不顺服,你们倒蒙了怜恤。


 罗11:31
 so they too have now become disobedient in order that they too may now receive mercy as a result of God's mercy to you.

 这样,他们也是不顺服,叫他们因着施给你们的怜恤,现在也就蒙怜恤。


 罗11:32
 For God has bound all men over to disobedience so that he may have mercy on them all.

 因为 神将众人都圈在不顺服之中,特意要怜恤众人。


 罗11:33
 Oh, the depth of the riches of the wisdom and knowledge of God! How unsearchable his judgments, and his paths beyond tracing out!

 深哉, 神丰富的智慧和知识!他的判断何其难测!他的踪迹何其难寻!


 罗11:34
 "Who has known the mind of the Lord? Or who has been his counselor?"

 “谁知道主的心?谁作过他的谋士呢?”


 罗11:35
 "Who has ever given to God, that God should repay him?"

 “谁是先给了他,使他后来偿还呢?”


 罗11:36
 For from him and through him and to him are all things. To him be the glory forever! Amen.

 因为万有都是本于他,倚靠他,归于他。愿荣耀归给他,直到永远。阿们!


 \section{罗马书第12章}
 罗12:1
 Therefore, I urge you, brothers, in view of God's mercy, to offer your bodies as living sacrifices, holy and pleasing to God--this is your spiritual act of worship.

 所以弟兄们,我以 神的慈悲劝你们,将身体献上,当作活祭,是圣洁的,是 神所喜悦的;你们如此事奉,乃是理所当然的。


 罗12:2
 Do not conform any longer to the pattern of this world, but be transformed by the renewing of your mind. Then you will be able to test and approve what God's will is--his good, pleasing and perfect will.

 不要效法这个世界,只要心意更新而变化,叫你们察验何为 神的善良、纯全、可喜悦的旨意。


 罗12:3
 For by the grace given me I say to every one of you: Do not think of yourself more highly than you ought, but rather think of yourself with sober judgment, in accordance with the measure of faith God has given you.

 我凭着所赐我的恩,对你们各人说:不要看自己过于所当看的,要照着 神所分给各人信心的大小,看得合乎中道。


 罗12:4
 Just as each of us has one body with many members, and these members do not all have the same function,

 正如我们一个身子上有好些肢体,肢体也不都是一样的用处。


 罗12:5
 so in Christ we who are many form one body, and each member belongs to all the others.

 我们这许多人,在基督里成为一身,互相联络作肢体,也是如此。


 罗12:6
 We have different gifts, according to the grace given us. If a man's gift is prophesying, let him use it in proportion to his faith.

 按我们所得的恩赐,各有不同:或说预言,就当照着信心的程度说预言;


 罗12:7
 If it is serving, let him serve; if it is teaching, let him teach;

 或作执事,就当专一执事;或作教导的,就当专一教导;


 罗12:8
 if it is encouraging, let him encourage; if it is contributing to the needs of others, let him give generously; if it is leadership, let him govern diligently; if it is showing mercy, let him do it cheerfully.

 或作劝化的,就当专一劝化;施舍的,就当诚实;治理的,就当殷勤;怜悯人的,就当甘心。


 罗12:9
 Love must be sincere. Hate what is evil; cling to what is good.

 爱人不可虚假;恶要厌恶,善要亲近。


 罗12:10
 Be devoted to one another in brotherly love. Honor one another above yourselves.

 爱弟兄,要彼此亲热;恭敬人,要彼此推让。


 罗12:11
 Never be lacking in zeal, but keep your spiritual fervor, serving the Lord.

 殷勤不可懒惰。要心里火热,常常服事主。


 罗12:12
 Be joyful in hope, patient in affliction, faithful in prayer.

 在指望中要喜乐,在患难中要忍耐;祷告要恒切。


 罗12:13
 Share with God's people who are in need. Practice hospitality.

 圣徒缺乏要帮补,客要一味地款待。


 罗12:14
 Bless those who persecute you; bless and do not curse.

 逼迫你们的,要给他们祝福;只要祝福,不可咒诅。


 罗12:15
 Rejoice with those who rejoice; mourn with those who mourn.

 与喜乐的人要同乐;与哀哭的人要同哭。


 罗12:16
 Live in harmony with one another. Do not be proud, but be willing to associate with people of low position. Do not be conceited.

 要彼此同心,不要志气高大,倒要俯就卑微的人(“人”或作“事”)。不要自以为聪明。


 罗12:17
 Do not repay anyone evil for evil. Be careful to do what is right in the eyes of everybody.

 不要以恶报恶。众人以为美的事,要留心去做。


 罗12:18
 If it is possible, as far as it depends on you, live at peace with everyone.

 若是能行,总要尽力与众人和睦。


 罗12:19
 Do not take revenge, my friends, but leave room for God's wrath, for it is written: "It is mine to avenge; I will repay," says the Lord.

 亲爱的弟兄,不要自己伸冤,宁可让步,听凭主怒(或作“让人发怒”)。因为经上记着:“主说:‘伸冤在我,我必报应。’”


 罗12:20
 On the contrary: "If your enemy is hungry, feed him; if he is thirsty, give him something to drink. In doing this, you will heap burning coals on his head."

 所以,“你的仇敌若饿了,就给他吃;若渴了,就给他喝。因为你这样行,就是把炭火堆在他的头上。”


 罗12:21
 Do not be overcome by evil, but overcome evil with good.

 你不可为恶所胜,反要以善胜恶。


\section{ 罗马书第13章}
 罗13:1
 Everyone must submit himself to the governing authorities, for there is no authority except that which God has established. The authorities that exist have been established by God.

 在上有权柄的,人人当顺服他,因为没有权柄不是出于 神的,凡掌权的都是 神所命的。


 罗13:2
 Consequently, he who rebels against the authority is rebelling against what God has instituted, and those who do so will bring judgment on themselves.

 所以抗拒掌权的,就是抗拒 神的命;抗拒的必自取刑罚。


 罗13:3
 For rulers hold no terror for those who do right, but for those who do wrong. Do you want to be free from fear of the one in authority? Then do what is right and he will commend you.

 作官的原不是叫行善的惧怕,乃是叫作恶的惧怕。你愿意不惧怕掌权的吗?你只要行善,就可得他的称赞,


 罗13:4
 For he is God's servant to do you good. But if you do wrong, be afraid, for he does not bear the sword for nothing. He is God's servant, an agent of wrath to bring punishment on the wrongdoer.

 因为他是 神的用人,是与你有益的。你若作恶,却当惧怕,因为他不是空空地佩剑。他是 神的用人,是伸冤的,刑罚那作恶的。


 罗13:5
 Therefore, it is necessary to submit to the authorities, not only because of possible punishment but also because of conscience.

 所以你们必须顺服,不但是因为刑罚,也是因为良心。


 罗13:6
 This is also why you pay taxes, for the authorities are God's servants, who give their full time to governing.

 你们纳粮也为这个缘故,因他们是 神的差役,常常特管这事。


 罗13:7
 Give everyone what you owe him: If you owe taxes, pay taxes; if revenue, then revenue; if respect, then respect; if honor, then honor.

 凡人所当得的,就给他;当得粮的,给他纳粮;当得税的,给他上税;当惧怕的,惧怕他;当恭敬的,恭敬他。


 罗13:8
 Let no debt remain outstanding, except the continuing debt to love one another, for he who loves his fellowman has fulfilled the law.

 凡事都不可亏欠人,惟有彼此相爱,要常以为亏欠,因为爱人的就完全了律法。


 罗13:9
 The commandments, "Do not commit adultery," "Do not murder," "Do not steal," "Do not covet," and whatever other commandment there may be, are summed up in this one rule: "Love your neighbor as yourself."

 像那“不可奸淫”,“不可杀人”,“不可偷盗”,“不可贪婪”,或有别的诫命,都包在“爱人如己”这一句话之内了。


 罗13:10
 Love does no harm to its neighbor. Therefore love is the fulfillment of the law.

 爱是不加害与人的,所以爱就完全了律法。


 罗13:11
 And do this, understanding the present time. The hour has come for you to wake up from your slumber, because our salvation is nearer now than when we first believed.

 再者,你们晓得现今就是该趁早睡醒的时候,因为我们得救,现今比初信的时候更近了。


 罗13:12
 The night is nearly over; the day is almost here. So let us put aside the deeds of darkness and put on the armor of light.

 黑夜已深,白昼将近;我们就当脱去暗昧的行为,带上光明的兵器。


 罗13:13
 Let us behave decently, as in the daytime, not in orgies and drunkenness, not in sexual immorality and debauchery, not in dissension and jealousy.

 行事为人要端正,好像行在白昼;不可荒宴醉酒,不可好色邪荡,不可争竞嫉妒。


 罗13:14
 Rather, clothe yourselves with the Lord Jesus Christ, and do not think about how to gratify the desires of the sinful nature.

 总要披戴主耶稣基督,不要为肉体安排去放纵私欲。


\section{ 罗马书第14章}
 罗14:1
 Accept him whose faith is weak, without passing judgment on disputable matters.

 信心软弱的,你们要接纳,但不要辩论所疑惑的事。


 罗14:2
 One man's faith allows him to eat everything, but another man, whose faith is weak, eats only vegetables.

 有人信百物都可吃,但那软弱的,只吃蔬菜。


 罗14:3
 The man who eats everything must not look down on him who does not, and the man who does not eat everything must not condemn the man who does, for God has accepted him.

 吃的人不可轻看不吃的人,不吃的人不可论断吃的人,因为 神已经收纳他了。


 罗14:4
 Who are you to judge someone else's servant? To his own master he stands or falls. And he will stand, for the Lord is able to make him stand.

 你是谁,竟论断别人的仆人呢?他或站住,或跌倒,自有他的主人在;而且他也必要站住,因为主能使他站住。


 罗14:5
 One man considers one day more sacred than another; another man considers every day alike. Each one should be fully convinced in his own mind.

 有人看这日比那日强,有人看日日都是一样,只是各人心里要意见坚定。


 罗14:6
 He who regards one day as special, does so to the Lord. He who eats meat, eats to the Lord, for he gives thanks to God; and he who abstains, does so to the Lord and gives thanks to God.

 守日的人,是为主守的;吃的人,是为主吃的,因他感谢 神;不吃的人,是为主不吃的,也感谢 神。


 罗14:7
 For none of us lives to himself alone and none of us dies to himself alone.

 我们没有一个人为自己活,也没有一个人为自己死。


 罗14:8
 If we live, we live to the Lord; and if we die, we die to the Lord. So, whether we live or die, we belong to the Lord.

 我们若活着,是为主而活;若死了,是为主而死。所以我们或活或死,总是主的人。


 罗14:9
 For this very reason, Christ died and returned to life so that he might be the Lord of both the dead and the living.

 因此基督死了,又活了,为要作死人并活人的主。


 罗14:10
 You, then, why do you judge your brother? Or why do you look down on your brother? For we will all stand before God's judgment seat.

 你这个人,为什么论断弟兄呢?又为什么轻看弟兄呢?因我们都要站在 神的台前。


 罗14:11
 It is written: "'As surely as I live,' says the Lord, 'every knee will bow before me; every tongue will confess to God.'"

 经上写着:“主说:‘我凭着我的永生起誓,万膝必向我跪拜,万口必向我承认。’”


 罗14:12
 So then, each of us will give an account of himself to God.

 这样看来,我们各人必要将自己的事在 神面前说明。


 罗14:13
 Therefore let us stop passing judgment on one another. Instead, make up your mind not to put any stumbling block or obstacle in your brother's way.

 所以我们不可再彼此论断,宁可定意,谁也不给弟兄放下绊脚跌人之物。


 罗14:14
 As one who is in the Lord Jesus, I am fully convinced that no food is unclean in itself. But if anyone regards something as unclean, then for him it is unclean.

 我凭着主耶稣确知深信,凡物本来没有不洁净的;惟独人以为不洁净的,在他就不洁净了。


 罗14:15
 If your brother is distressed because of what you eat, you are no longer acting in love. Do not by your eating destroy your brother for whom Christ died.

 你若因食物叫弟兄忧愁,就不是按着爱人的道理行。基督已经替他死,你不可因你的食物叫他败坏。


 罗14:16
 Do not allow what you consider good to be spoken of as evil.

 不可叫你的善被人毁谤,


 罗14:17
 For the kingdom of God is not a matter of eating and drinking, but of righteousness, peace and joy in the Holy Spirit,

 因为 神的国不在乎吃喝,只在乎公义、和平并圣灵中的喜乐。


 罗14:18
 because anyone who serves Christ in this way is pleasing to God and approved by men.

 在这几样上服事基督的,就为 神所喜悦,又为人所称许。


 罗14:19
 Let us therefore make every effort to do what leads to peace and to mutual edification.

 所以,我们务要追求和睦的事与彼此建立德行的事。


 罗14:20
 Do not destroy the work of God for the sake of food. All food is clean, but it is wrong for a man to eat anything that causes someone else to stumble.

 不可因食物毁坏 神的工程。凡物固然洁净,但有人因食物叫人跌倒,就是他的罪了。


 罗14:21
 It is better not to eat meat or drink wine or to do anything else that will cause your brother to fall.

 无论是吃肉,是喝酒,是什么别的事,叫弟兄跌倒,一概不做才好。


 罗14:22
 So whatever you believe about these things keep between yourself and God. Blessed is the man who does not condemn himself by what he approves.

 你有信心,就当在 神面前守着。人在自己以为可行的事上能不自责,就有福了。


 罗14:23
 But the man who has doubts is condemned if he eats, because his eating is not from faith; and everything that does not come from faith is sin.

 若有疑心而吃的,就必有罪。因为他吃,不是出于信心;凡不出于信心的都是罪。


\section{ 罗马书第15章}
 罗15:1
 We who are strong ought to bear with the failings of the weak and not to please ourselves.

 我们坚固的人应该担代不坚固人的软弱,不求自己的喜悦。


 罗15:2
 Each of us should please his neighbor for his good, to build him up.

 我们各人务要叫邻舍喜悦,使他得益处,建立德行。


 罗15:3
 For even Christ did not please himself but, as it is written: "The insults of those who insult you have fallen on me."

 因为基督也不求自己的喜悦,如经上所记:“辱骂你人的辱骂都落在我身上。”


 罗15:4
 For everything that was written in the past was written to teach us, so that through endurance and the encouragement of the Scriptures we might have hope.

 从前所写的圣经,都是为教训我们写的,叫我们因圣经所生的忍耐和安慰,可以得着盼望。


 罗15:5
 May the God who gives endurance and encouragement give you a spirit of unity among yourselves as you follow Christ Jesus,

 但愿赐忍耐、安慰的 神,叫你们彼此同心,效法基督耶稣,


 罗15:6
 so that with one heart and mouth you may glorify the God and Father of our Lord Jesus Christ.

 一心一口荣耀 神、我们主耶稣基督的父。


 罗15:7
 Accept one another, then, just as Christ accepted you, in order to bring praise to God.

 所以你们要彼此接纳,如同基督接纳你们一样,使荣耀归与 神。


 罗15:8
 For I tell you that Christ has become a servant of the Jews on behalf of God's truth, to confirm the promises made to the patriarchs

 我说,基督是为 神真理作了受割礼人的执事,要证实所应许列祖的话,


 罗15:9
 so that the Gentiles may glorify God for his mercy, as it is written: "Therefore I will praise you among the Gentiles; I will sing hymns to your name."

 并叫外邦人因他的怜悯荣耀 神。如经上所记:“因此我要在外邦中称赞你,歌颂你的名。”


 罗15:10
 Again, it says, "Rejoice, O Gentiles, with his people."

 又说:“你们外邦人当与主的百姓一同欢乐。”


 罗15:11
 And again, "Praise the Lord, all you Gentiles, and sing praises to him, all you peoples."

 又说:“外邦啊,你们当赞美主!万民哪,你们都当颂赞他!”


 罗15:12
 And again, Isaiah says, "The Root of Jesse will spring up, one who will arise to rule over the nations; the Gentiles will hope in him."

 又有以赛亚说:“将来有耶西的根,就是那兴起来要治理外邦的,外邦人要仰望他。”


 罗15:13
 May the God of hope fill you with all joy and peace as you trust in him, so that you may overflow with hope by the power of the Holy Spirit.

 但愿使人有盼望的 神,因信将诸般的喜乐平安充满你们的心,使你们藉着圣灵的能力大有盼望。


 罗15:14
 I myself am convinced, my brothers, that you yourselves are full of goodness, complete in knowledge and competent to instruct one another.

 弟兄们,我自己也深信你们是满有良善,充足了诸般的知识,也能彼此劝戒。


 罗15:15
 I have written you quite boldly on some points, as if to remind you of them again, because of the grace God gave me

 但我稍微放胆写信给你们,是要提醒你们的记性,特因 神所给我的恩典,


 罗15:16
 to be a minister of Christ Jesus to the Gentiles with the priestly duty of proclaiming the gospel of God, so that the Gentiles might become an offering acceptable to God, sanctified by the Holy Spirit.

 使我为外邦人作基督耶稣的仆役,作 神福音的祭司,叫所献上的外邦人,因着圣灵成为圣洁,可蒙悦纳。


 罗15:17
 Therefore I glory in Christ Jesus in my service to God.

 所以论到 神的事,我在基督耶稣里有可夸的。


 罗15:18
 I will not venture to speak of anything except what Christ has accomplished through me in leading the Gentiles to obey God by what I have said and done--

 除了基督藉我做的那些事,我什么都不敢提,只提他藉我言语作为,用神迹奇事的能力,并圣灵的能力,使外邦人顺服。


 罗15:19
 by the power of signs and miracles, through the power of the Spirit. So from Jerusalem all the way around to Illyricum, I have fully proclaimed the gospel of Christ.

 甚至我从耶路撒冷直转到以利哩古,到处传了基督的福音。


 罗15:20
 It has always been my ambition to preach the gospel where Christ was not known, so that I would not be building on someone else's foundation.

 我立了志向,不在基督的名被称过的地方传福音,免得建造在别人的根基上。


 罗15:21
 Rather, as it is written: "Those who were not told about him will see, and those who have not heard will understand."

 就如经上所记:“未曾闻知他信息的,将要看见;未曾听过的,将要明白。”


 罗15:22
 This is why I have often been hindered from coming to you.

 我因多次被拦阻,总不得到你们那里去。


 罗15:23
 But now that there is no more place for me to work in these regions, and since I have been longing for many years to see you,

 但如今在这里再没有可传的地方,而且这好几年,我切心想望到西班牙去的时候,可以到你们那里,


 罗15:24
 I plan to do so when I go to Spain. I hope to visit you while passing through and to have you assist me on my journey there, after I have enjoyed your company for a while.

 盼望从你们那里经过,得见你们,先与你们彼此交往,心里稍微满足,然后蒙你们送行。


 罗15:25
 Now, however, I am on my way to Jerusalem in the service of the saints there.

 但现在,我往耶路撒冷去供给圣徒。


 罗15:26
 For Macedonia and Achaia were pleased to make a contribution for the poor among the saints in Jerusalem.

 因为马其顿和亚该亚人乐意凑出捐项给耶路撒冷圣徒中的穷人。


 罗15:27
 They were pleased to do it, and indeed they owe it to them. For if the Gentiles have shared in the Jews' spiritual blessings, they owe it to the Jews to share with them their material blessings.

 这固然是他们乐意的,其实也算是所欠的债。因外邦人既然在他们属灵的好处上有分,就当把养身之物供给他们。


 罗15:28
 So after I have completed this task and have made sure that they have received this fruit, I will go to Spain and visit you on the way.

 等我办完了这事,把这善果向他们交付明白,我就要路过你们那里,往西班牙去。


 罗15:29
 I know that when I come to you, I will come in the full measure of the blessing of Christ.

 我也晓得去的时候,必带着基督丰盛的恩典而去。


 罗15:30
 I urge you, brothers, by our Lord Jesus Christ and by the love of the Spirit, to join me in my struggle by praying to God for me.

 弟兄们,我藉着我们主耶稣基督,又藉着圣灵的爱,劝你们与我一同竭力,为我祈求 神,


 罗15:31
 Pray that I may be rescued from the unbelievers in Judea and that my service in Jerusalem may be acceptable to the saints there,

 叫我脱离在犹太不顺从的人;也叫我为耶路撒冷所办的捐项可蒙圣徒悦纳;


 罗15:32
 so that by God's will I may come to you with joy and together with you be refreshed.

 并叫我顺着 神的旨意,欢欢喜喜地到你们那里,与你们同得安息。


 罗15:33
 The God of peace be with you all. Amen.

 愿赐平安的 神常和你们众人同在。阿们!


\section{ 罗马书第16章}
 罗16:1
 I commend to you our sister Phoebe, a servant of the church in Cenchrea.

 我对你们举荐我们的姊妹非比,她是坚革哩教会中的女执事。


 罗16:2
 I ask you to receive her in the Lord in a way worthy of the saints and to give her any help she may need from you, for she has been a great help to many people, including me.

 请你们为主接待她,合乎圣徒的体统。她在何事上要你们帮助,你们就帮助她,因她素来帮助许多人,也帮助了我。


 罗16:3
 Greet Priscilla and Aquila, my fellow workers in Christ Jesus.

 问百基拉和亚居拉安。他们在基督耶稣里与我同工,


 罗16:4
 They risked their lives for me. Not only I but all the churches of the Gentiles are grateful to them.

 也为我的命将自己的颈项置之度外。不但我感谢他们,就是外邦的众教会也感谢他们。


 罗16:5
 Greet also the church that meets at their house. Greet my dear friend Epenetus, who was the first convert to Christ in the province of Asia.

 又问在他们家中的教会安。问我所亲爱的以拜尼土安,他在亚西亚是归基督初结的果子。


 罗16:6
 Greet Mary, who worked very hard for you.

 又问马利亚安,她为你们多受劳苦。


 罗16:7
 Greet Andronicus and Junias, my relatives who have been in prison with me. They are outstanding among the apostles, and they were in Christ before I was.

 又问我亲属与我一同坐监的安多尼古和犹尼亚安,他们在使徒中是有名望的,也是比我先在基督里。


 罗16:8
 Greet Ampliatus, whom I love in the Lord.

 又问我在主里面所亲爱的暗伯利安。


 罗16:9
 Greet Urbanus, our fellow worker in Christ, and my dear friend Stachys.

 又问在基督里与我们同工的耳巴奴并我所亲爱的士大古安。


 罗16:10
 Greet Apelles, tested and approved in Christ. Greet those who belong to the household of Aristobulus.

 又问在基督里经过试验的亚比利安。问亚利多布家鶪的人安。


 罗16:11
 Greet Herodion, my relative. Greet those in the household of Narcissus who are in the Lord.

 又问我亲属希罗天安。问拿其数家在主里的人安。


 罗16:12
 Greet Tryphena and Tryphosa, those women who work hard in the Lord. Greet my dear friend Persis, another woman who has worked very hard in the Lord.

 又问为主劳苦的土非拿氏和土富撒氏安。问可亲爱为主多受劳苦的彼息氏安。


 罗16:13
 Greet Rufus, chosen in the Lord, and his mother, who has been a mother to me, too.

 又问在主蒙拣选的鲁孚和他母亲安;他的母亲就是我的母亲。


 罗16:14
 Greet Asyncritus, Phlegon, Hermes, Patrobas, Hermas and the brothers with them.

 又问亚逊其土、弗勒干、黑米、八罗巴、黑马并与他们在一处的弟兄们安。


 罗16:15
 Greet Philologus, Julia, Nereus and his sister, and Olympas and all the saints with them.

 又问非罗罗古和犹利亚、尼利亚和他姊妹、同阿林巴并与他们在一处的众圣徒安。


 罗16:16
 Greet one another with a holy kiss. All the churches of Christ send greetings.

 你们亲嘴问安,彼此务要圣洁。基督的众教会都问你们安。


 罗16:17
 I urge you, brothers, to watch out for those who cause divisions and put obstacles in your way that are contrary to the teaching you have learned. Keep away from them.

 弟兄们,那些离间你们、叫你们跌倒、背乎所学之道的人,我劝你们要留意躲避他们。


 罗16:18
 For such people are not serving our Lord Christ, but their own appetites. By smooth talk and flattery they deceive the minds of naive people.

 因为这样的人不服事我们的主基督,只服事自己的肚腹,用花言巧语诱惑那些老实人的心。


 罗16:19
 Everyone has heard about your obedience, so I am full of joy over you; but I want you to be wise about what is good, and innocent about what is evil.

 你们的顺服已经传于众人,所以我为你们欢喜,但我愿意你们在善上聪明,在恶上愚拙。


 罗16:20
 The God of peace will soon crush Satan under your feet. The grace of our Lord Jesus be with you.

 赐平安的 神,快要将撒但践踏在你们脚下。愿我主耶稣基督的恩常和你们同在!


 罗16:21
 Timothy, my fellow worker, sends his greetings to you, as do Lucius, Jason and Sosipater, my relatives.

 与我同工的提摩太和我的亲属路求、耶孙、所西巴德问你们安。


 罗16:22
 I, Tertius, who wrote down this letter, greet you in the Lord.

 我这代笔写信的德丢,在主里面问你们安。


 罗16:23
 Gaius, whose hospitality I and the whole church here enjoy, sends you his greetings. Erastus, who is the city's director of public works, and our brother Quartus send you their greetings.

 那接待我,也接待全教会的该犹问你们安。


 罗16:24
 [May the grace of our Lord Jesus Christ be with all of you. Amen.]

 城内管银库的以拉都和兄弟括土问你们安。


 罗16:25
 Now to him who is able to establish you by my gospel and the proclamation of Jesus Christ, according to the revelation of the mystery hidden for long ages past,

 惟有 神能照我所传的福音,和所讲的耶稣基督,并照永古隐藏不言的奥秘,坚固你们的心。


 罗16:26
 but now revealed and made known through the prophetic writings by the command of the eternal God, so that all nations might believe and obey him--

 这奥秘如今显明出来,而且按着永生 神的命,藉众先知的书指示万国的民,使他们信服真道。


 罗16:27
 to the only wise God be glory forever through Jesus Christ! Amen.

 愿荣耀,因耶稣基督,归与独一全智的 神,直到永远。阿们。

%%%%%%%%%%%%%%%%%%%%%%%%%%%%%%%%%%%%%%%%%%%%%%%%%%%%%%%%%%%%%%%%
%  分栏结束
%%%%%%%%%%%%%%%%%%%%%%%%%%%%%%%%%%%%%%%%%%%%%%%%%%%%%%%%%%%%%%%%
\end{multicols}
%%%%%%%%%%%%%%%%%%%%%%%%%%%%%%%%%%%%%%%%%%%%%%%%%%%%%%%%%%%%%%%%
%  文章结束
%%%%%%%%%%%%%%%%%%%%%%%%%%%%%%%%%%%%%%%%%%%%%%%%%%%%%%%%%%%%%%%%
\clearpage

\end{document}