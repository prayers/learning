# 传道书
<!-- TOC -->

- [传道书](#传道书)
    - [传道书第1章](#传道书第1章)
    - [传道书第2章](#传道书第2章)
    - [传道书第3章](#传道书第3章)
    - [传道书第4章](#传道书第4章)
    - [传道书第5章](#传道书第5章)
    - [传道书第6章](#传道书第6章)
    - [传道书第7章](#传道书第7章)
    - [传道书第8章](#传道书第8章)
    - [传道书第9章](#传道书第9章)
    - [传道书第10章](#传道书第10章)
    - [传道书第11章](#传道书第11章)
    - [传道书第12章](#传道书第12章)

<!-- /TOC -->
## 传道书第1章
##### 传1:1
> The words of the Teacher, son of David, king in Jerusalem:
>
> 在耶路撒冷作王,大卫的儿子,传道者的言语。


##### 传1:2
> "Meaningless! Meaningless!" says the Teacher. "Utterly meaningless! Everything is meaningless."
>
> 传道者说:“虚空的虚空,虚空的虚空,凡事都是虚空。”


##### 传1:3
> What does man gain from all his labor at which he toils under the sun?
>
> 人一切的劳碌,就是他在日光之下的劳碌,有什么益处呢?


##### 传1:4
> Generations come and generations go, but the earth remains forever.
>
> 一代过去,一代又来,地却永远长存。


##### 传1:5
> The sun rises and the sun sets, and hurries back to where it rises.
>
> 日头出来,日头落下,急归所出之地。


##### 传1:6
> The wind blows to the south and turns to the north; round and round it goes, ever returning on its course.
>
> 风往南刮,又向北转,不住地旋转,而且返回转行原道。


##### 传1:7
> All streams flow into the sea, yet the sea is never full. To the place the streams come from, there they return again.
>
> 江河都往海里流,海却不满;江河从何处流,仍归还何处。


##### 传1:8
> All things are wearisome, more than one can say. The eye never has enough of seeing, nor the ear its fill of hearing.
>
> 万事令人厌烦(或作“万物满有困乏”),人不能说尽。眼看,看不饱;耳听,听不足。


##### 传1:9
> What has been will be again, what has been done will be done again; there is nothing new under the sun.
>
> 已有的事,后必再有;已行的事,后必再行。日光之下,并无新事。


##### 传1:10
> Is there anything of which one can say, "Look! This is something new"? It was here already, long ago; it was here before our time.
>
> 岂有一件事人能指着说:“这是新的”?哪知,在我们以前的世代早已有了。


##### 传1:11
> There is no remembrance of men of old, and even those who are yet to come will not be remembered by those who follow.
>
> 已过的世代,无人记念;将来的世代,后来的人也不记念。


##### 传1:12
> I, the Teacher, was king over Israel in Jerusalem.
>
> 我传道者在耶路撒冷作过以色列的王。


##### 传1:13
> I devoted myself to study and to explore by wisdom all that is done under heaven. What a heavy burden God has laid on men!
>
> 我专心用智慧寻求查究天下所做的一切事,乃知 神叫世人所经练的是极重的劳苦。


##### 传1:14
> I have seen all the things that are done under the sun; all of them are meaningless, a chasing after the wind.
>
> 我见日光之下所做的一切事,都是虚空,都是捕风。


##### 传1:15
> What is twisted cannot be straightened; what is lacking cannot be counted.
>
> 弯曲的不能变直;缺少的不能足数。


##### 传1:16
> I thought to myself, "Look, I have grown and increased in wisdom more than anyone who has ruled over Jerusalem before me; I have experienced much of wisdom and knowledge."
>
> 我心里议论说:“我得了大智慧,胜过我以前在耶路撒冷的众人,而且我心中多经历智慧和知识的事。”


##### 传1:17
> Then I applied myself to the understanding of wisdom, and also of madness and folly, but I learned that this, too, is a chasing after the wind.
>
> 我又专心察明智慧、狂妄和愚昧,乃知这也是捕风。


##### 传1:18
> For with much wisdom comes much sorrow; the more knowledge, the more grief.
>
> 因为多有智慧,就多有愁烦;加增知识的,就加增忧伤。


## 传道书第2章
##### 传2:1
> I thought in my heart, "Come now, I will test you with pleasure to find out what is good." But that also proved to be meaningless.
>
> 我心里说:“来吧!我以喜乐试试你,你好享福。”谁知,这也是虚空。


##### 传2:2
> "Laughter," I said, "is foolish. And what does pleasure accomplish?"
>
> 我指嬉笑说:“这是狂妄”;论喜乐说:“有何功效呢?”


##### 传2:3
> I tried cheering myself with wine, and embracing folly--my mind still guiding me with wisdom. I wanted to see what was worthwhile for men to do under heaven during the few days of their lives.
>
> 我心里察究,如何用酒使我肉体舒畅,我心却仍以智慧引导我;又如何持住愚昧,等我看明世人,在天下一生当行何事为美。


##### 传2:4
> I undertook great projects: I built houses for myself and planted vineyards.
>
> 我为自己动大工程,建造房屋,栽种葡萄园;


##### 传2:5
> I made gardens and parks and planted all kinds of fruit trees in them.
>
> 修造园囿,在其中栽种各样果木树;


##### 传2:6
> I made reservoirs to water groves of flourishing trees.
>
> 挖造水池,用以浇灌嫩小的树木。


##### 传2:7
> I bought male and female slaves and had other slaves who were born in my house. I also owned more herds and flocks than anyone in Jerusalem before me.
>
> 我买了仆婢,也有生在家中的仆婢;又有许多牛群羊群,胜过以前在耶路撒冷众人所有的。


##### 传2:8
> I amassed silver and gold for myself, and the treasure of kings and provinces. I acquired men and women singers, and a harem as well--the delights of the heart of man.
>
> 我又为自己积蓄金银和君王的财宝,并各省的财宝;又得唱歌的男女和世人所喜爱的物,并许多的妃嫔。


##### 传2:9
> I became greater by far than anyone in Jerusalem before me. In all this my wisdom stayed with me.
>
> 这样,我就日见昌盛,胜过以前在耶路撒冷的众人。我的智慧仍然存留。


##### 传2:10
> I denied myself nothing my eyes desired; I refused my heart no pleasure. My heart took delight in all my work, and this was the reward for all my labor.
>
> 凡我眼所求的,我没有留下不给他的;我心所乐的,我没有禁止不享受的;因我的心为我一切所劳碌的快乐,这就是我从劳碌中所得的分。


##### 传2:11
> Yet when I surveyed all that my hands had done and what I had toiled to achieve, everything was meaningless, a chasing after the wind; nothing was gained under the sun.
>
> 后来,我察看我手所经营的一切事和我劳碌所成的功,谁知都是虚空,都是捕风,在日光之下毫无益处。


##### 传2:12
> Then I turned my thoughts to consider wisdom, and also madness and folly. What more can the king's successor do than what has already been done?
>
> 我转念观看智慧、狂妄和愚昧。在王以后而来的人还能做什么呢?也不过行早先所行的就是了。


##### 传2:13
> I saw that wisdom is better than folly, just as light is better than darkness.
>
> 我便看出智慧胜过愚昧,如同光明胜过黑暗。


##### 传2:14
> The wise man has eyes in his head, while the fool walks in the darkness; but I came to realize that the same fate overtakes them both.
>
> 智慧人的眼目光明(原文作“在他头上”);愚昧人在黑暗里行。我却看明有一件事,这两等人都必遇见。


##### 传2:15
> Then I thought in my heart, "The fate of the fool will overtake me also. What then do I gain by being wise?" I said in my heart, "This too is meaningless."
>
> 我就心里说:“愚昧人所遇见的,我也必遇见,我为何更有智慧呢?”我心里说:“这也是虚空。”


##### 传2:16
> For the wise man, like the fool, will not be long remembered; in days to come both will be forgotten. Like the fool, the wise man too must die!
>
> 智慧人和愚昧人一样,永远无人记念,因为日后都被忘记;可叹智慧人死亡,与愚昧人无异。


##### 传2:17
> So I hated life, because the work that is done under the sun was grievous to me. All of it is meaningless, a chasing after the wind.
>
> 我所以恨恶生命,因为在日光之下所行的事,我都以为烦恼,都是虚空,都是捕风。


##### 传2:18
> I hated all the things I had toiled for under the sun, because I must leave them to the one who comes after me.
>
> 我恨恶一切的劳碌,就是我在日光之下的劳碌,因为我得来的必留给我以后的人。


##### 传2:19
> And who knows whether he will be a wise man or a fool? Yet he will have control over all the work into which I have poured my effort and skill under the sun. This too is meaningless.
>
> 那人是智慧是愚昧,谁能知道?他竟要管理我劳碌所得的,就是我在日光之下用智慧所得的。这也是虚空。


##### 传2:20
> So my heart began to despair over all my toilsome labor under the sun.
>
> 故此,我转想我在日光之下所劳碌的一切工作,心便绝望。


##### 传2:21
> For a man may do his work with wisdom, knowledge and skill, and then he must leave all he owns to someone who has not worked for it. This too is meaningless and a great misfortune.
>
> 因为有人用智慧、知识、灵巧所劳碌得来的,却要留给未曾劳碌的人为分。这也是虚空,也是大患!


##### 传2:22
> What does a man get for all the toil and anxious striving with which he labors under the sun?
>
> 人在日光之下劳碌累心,在他一切的劳碌上得着什么呢?


##### 传2:23
> All his days his work is pain and grief; even at night his mind does not rest. This too is meaningless.
>
> 因为他日日忧虑,他的劳苦成为愁烦,连夜间心也不安。这也是虚空。


##### 传2:24
> A man can do nothing better than to eat and drink and find satisfaction in his work. This too, I see, is from the hand of God,
>
> 人莫强如吃喝,且在劳碌中享福,我看这也是出于 神的手。


##### 传2:25
> for without him, who can eat or find enjoyment?
>
> 论到吃用、享福,谁能胜过我呢?


##### 传2:26
> To the man who pleases him, God gives wisdom, knowledge and happiness, but to the sinner he gives the task of gathering and storing up wealth to hand it over to the one who pleases God. This too is meaningless, a chasing after the wind.
>
> 神喜悦谁,就给谁智慧、知识和喜乐,惟有罪人, 神使他劳苦,叫他将所收聚的、所堆积的归给 神所喜悦的人。这也是虚空,也是捕风。


## 传道书第3章
##### 传3:1
> There is a time for everything, and a season for every activity under heaven:
>
> 凡事都有定期,天下万务都有定时。


##### 传3:2
> a time to be born and a time to die, a time to plant and a time to uproot,
>
> 生有时,死有时;栽种有时,拔出所栽种的也有时;


##### 传3:3
> a time to kill and a time to heal, a time to tear down and a time to build,
>
> 杀戮有时,医治有时;拆毁有时,建造有时;


##### 传3:4
> a time to weep and a time to laugh, a time to mourn and a time to dance,
>
> 哭有时,笑有时;哀恸有时,跳舞有时;


##### 传3:5
> a time to scatter stones and a time to gather them, a time to embrace and a time to refrain,
>
> 抛掷石头有时,堆聚石头有时;怀抱有时,不怀抱有时;


##### 传3:6
> a time to search and a time to give up, a time to keep and a time to throw away,
>
> 寻找有时,失落有时;保守有时,舍弃有时;


##### 传3:7
> a time to tear and a time to mend, a time to be silent and a time to speak,
>
> 撕裂有时,缝补有时;静默有时,言语有时;


##### 传3:8
> a time to love and a time to hate, a time for war and a time for peace.
>
> 喜爱有时,恨恶有时;争战有时,和好有时。


##### 传3:9
> What does the worker gain from his toil?
>
> 这样看来,做事的人在他的劳碌上有什么益处呢?


##### 传3:10
> I have seen the burden God has laid on men.
>
> 我见 神叫世人劳苦,使他们在其中受经练。


##### 传3:11
> He has made everything beautiful in its time. He has also set eternity in the hearts of men; yet they cannot fathom what God has done from beginning to end.
>
> 神造万物,各按其时成为美好,又将永生安置在世人心里(“永生”原文作“永远”)。然而 神从始至终的作为,人不能参透。


##### 传3:12
> I know that there is nothing better for men than to be happy and do good while they live.
>
> 我知道世人,莫强如终身喜乐行善,


##### 传3:13
> That everyone may eat and drink, and find satisfaction in all his toil--this is the gift of God.
>
> 并且人人吃喝,在他一切劳碌中享福,这也是 神的恩赐。


##### 传3:14
> I know that everything God does will endure forever; nothing can be added to it and nothing taken from it. God does it so that men will revere him.
>
> 我知道 神一切所做的都必永存,无所增添,无所减少。 神这样行,是要人在他面前存敬畏的心。


##### 传3:15
> Whatever is has already been, and what will be has been before; and God will call the past to account.
>
> 现今的事早先就有了,将来的事早已也有了,并且 神使已过的事重新再来(或作“并且 神再寻回已过的事”)。


##### 传3:16
> And I saw something else under the sun: In the place of judgment--wickedness was there, in the place of justice--wickedness was there.
>
> 我又见日光之下:在审判之处有奸恶,在公义之处也有奸恶。


##### 传3:17
> I thought in my heart, "God will bring to judgment both the righteous and the wicked, for there will be a time for every activity, a time for every deed."
>
> 我心里说:“ 神必审判义人和恶人,因为在那里,各样事务,一切工作,都有定时。”


##### 传3:18
> I also thought, "As for men, God tests them so that they may see that they are like the animals.
>
> 我心里说:“这乃为世人的缘故,是 神要试验他们,使他们觉得自己不过像兽一样。


##### 传3:19
> Man's fate is like that of the animals; the same fate awaits them both: As one dies, so dies the other. All have the same breath; man has no advantage over the animal. Everything is meaningless.
>
> 因为世人遭遇的,兽也遭遇,所遭遇的都是一样:这个怎样死,那个也怎样死,气息都是一样。人不能强于兽,都是虚空。


##### 传3:20
> All go to the same place; all come from dust, and to dust all return.
>
> 都归一处,都是出于尘土,也都归于尘土。


##### 传3:21
> Who knows if the spirit of man rises upward and if the spirit of the animal goes down into the earth?"
>
> 谁知道人的灵是往上升,兽的魂是下入地呢?”


##### 传3:22
> So I saw that there is nothing better for a man than to enjoy his work, because that is his lot. For who can bring him to see what will happen after him?
>
> 故此,我见人莫强如在他经营的事上喜乐,因为这是他的分;他身后的事,谁能使他回来得见呢?


## 传道书第4章
##### 传4:1
> Again I looked and saw all the oppression that was taking place under the sun: I saw the tears of the oppressed--and they have no comforter; power was on the side of their oppressors--and they have no comforter.
>
> 我又转念,见日光之下所行的一切欺压:看哪,受欺压的流泪,且无人安慰;欺压他们的有势力,也无人安慰他们。


##### 传4:2
> And I declared that the dead, who had already died, are happier than the living, who are still alive.
>
> 因此,我赞叹那早已死的死人,胜过那还活着的活人。


##### 传4:3
> But better than both is he who has not yet been, who has not seen the evil that is done under the sun.
>
> 并且我以为那未曾生的,就是未见过日光之下恶事的,比这两等人更强。


##### 传4:4
> And I saw that all labor and all achievement spring from man's envy of his neighbor. This too is meaningless, a chasing after the wind.
>
> 我又见人为一切的劳碌和各样灵巧的工作,就被邻舍嫉妒。这也是虚空,也是捕风。


##### 传4:5
> The fool folds his hands and ruins himself.
>
> 愚昧人抱着手,吃自己的肉。


##### 传4:6
> Better one handful with tranquillity than two handfuls with toil and chasing after the wind.
>
> 满了一把,得享安静,强如满了两把,劳碌捕风。


##### 传4:7
> Again I saw something meaningless under the sun:
>
> 我又转念,见日光之下有一件虚空的事:


##### 传4:8
> There was a man all alone; he had neither son nor brother. There was no end to his toil, yet his eyes were not content with his wealth. "For whom am I toiling," he asked, "and why am I depriving myself of enjoyment?" This too is meaningless--a miserable business!
>
> 有人孤单无二,无子无兄,竟劳碌不息,眼目也不以钱财为足。他说:“我劳劳碌碌,刻苦自己,不享福乐,到底是为谁呢?”这也是虚空,是极重的劳苦。


##### 传4:9
> Two are better than one, because they have a good return for their work:
>
> 两个人总比一个人好,因为二人劳碌同得美好的果效。


##### 传4:10
> If one falls down, his friend can help him up. But pity the man who falls and has no one to help him up!
>
> 若是跌倒,这人可以扶起他的同伴;若是孤身跌倒,没有别人扶起他来,这人就有祸了!


##### 传4:11
> Also, if two lie down together, they will keep warm. But how can one keep warm alone?
>
> 再者,二人同睡,就都暖和;一人独睡,怎能暖和呢?


##### 传4:12
> Though one may be overpowered, two can defend themselves. A cord of three strands is not quickly broken.
>
> 有人攻胜孤身一人,若有二人便能敌挡他;三股合成的绳子不容易折断。


##### 传4:13
> Better a poor but wise youth than an old but foolish king who no longer knows how to take warning.
>
> 贫穷而有智慧的少年人,胜过年老不肯纳谏的愚昧王。


##### 传4:14
> The youth may have come from prison to the kingship, or he may have been born in poverty within his kingdom.
>
> 这人是从监牢中出来作王;在他国中,生来原是贫穷的。


##### 传4:15
> I saw that all who lived and walked under the sun followed the youth, the king's successor.
>
> 我见日光之下一切行动的活人,都随从那第二位,就是起来代替老王的少年人。


##### 传4:16
> There was no end to all the people who were before them. But those who came later were not pleased with the successor. This too is meaningless, a chasing after the wind.
>
> 他所治理的众人,就是他的百姓,多得无数。在他后来的人,尚且不喜悦他。这真是虚空,也是捕风。


## 传道书第5章
##### 传5:1
> Guard your steps when you go to the house of God. Go near to listen rather than to offer the sacrifice of fools, who do not know that they do wrong.
>
> 你到 神的殿要谨慎脚步。因为近前听,胜过愚昧人献祭(或作“胜过献愚昧人的祭”),他们本不知道所做的是恶。


##### 传5:2
> Do not be quick with your mouth, do not be hasty in your heart to utter anything before God. God is in heaven and you are on earth, so let your words be few.
>
> 你在 神面前不可冒失开口,也不可心急发言。因为 神在天上,你在地下,所以你的言语要寡少。


##### 传5:3
> As a dream comes when there are many cares, so the speech of a fool when there are many words.
>
> 事务多,就令人做梦;言语多,就显出愚昧。


##### 传5:4
> When you make a vow to God, do not delay in fulfilling it. He has no pleasure in fools; fulfill your vow.
>
> 你向 神许愿,偿还不可迟延,因他不喜悦愚昧人,所以你许的愿应当偿还。


##### 传5:5
> It is better not to vow than to make a vow and not fulfill it.
>
> 你许愿不还,不如不许。


##### 传5:6
> Do not let your mouth lead you into sin. And do not protest to the temple messenger, "My vow was a mistake." Why should God be angry at what you say and destroy the work of your hands?
>
> 不可任你的口使肉体犯罪,也不可在祭司(原文作“使者”)面前说是错许了。为何使 神因你的声音发怒,败坏你手所做的呢?


##### 传5:7
> Much dreaming and many words are meaningless. Therefore stand in awe of God.
>
> 多梦和多言,其中多有虚幻。你只要敬畏 神。


##### 传5:8
> If you see the poor oppressed in a district, and justice and rights denied, do not be surprised at such things; for one official is eyed by a higher one, and over them both are others higher still.
>
> 你若在一省之中见穷人受欺压,并夺去公义公平的事,不要因此诧异。因有一位高过居高位的鉴察,在他们以上还有更高的。


##### 传5:9
> The increase from the land is taken by all; the king himself profits from the fields.
>
> 况且地的益处归众人,就是君王也受田地的供应。


##### 传5:10
> Whoever loves money never has money enough; whoever loves wealth is never satisfied with his income. This too is meaningless.
>
> 贪爱银子的,不因得银子知足;贪爱丰富的,也不因得利益知足。这也是虚空。


##### 传5:11
> As goods increase, so do those who consume them. And what benefit are they to the owner except to feast his eyes on them?
>
> 货物增添,吃的人也增添,物主得什么益处呢?不过眼看而已。


##### 传5:12
> The sleep of a laborer is sweet, whether he eats little or much, but the abundance of a rich man permits him no sleep.
>
> 劳碌的人不拘吃多吃少,睡得香甜;富足人的丰满,却不容他睡觉。


##### 传5:13
> I have seen a grievous evil under the sun: wealth hoarded to the harm of its owner,
>
> 我见日光之下,有一宗大祸患:就是财主积存资财,反害自己。


##### 传5:14
> or wealth lost through some misfortune, so that when he has a son there is nothing left for him.
>
> 因遭遇祸患,这些资财就消灭;那人若生了儿子,手里也一无所有。


##### 传5:15
> Naked a man comes from his mother's womb, and as he comes, so he departs. He takes nothing from his labor that he can carry in his hand.
>
> 他怎样从母胎赤身而来,也必照样赤身而去;他所劳碌得来的,手中分毫不能带去。


##### 传5:16
> This too is a grievous evil: As a man comes, so he departs, and what does he gain, since he toils for the wind?
>
> 他来的情形怎样,他去的情形也怎样。这也是一宗大祸患。他为风劳碌有什么益处呢?


##### 传5:17
> All his days he eats in darkness, with great frustration, affliction and anger.
>
> 并且他终身在黑暗中吃喝,多有烦恼,又有病患呕气。


##### 传5:18
> Then I realized that it is good and proper for a man to eat and drink, and to find satisfaction in his toilsome labor under the sun during the few days of life God has given him--for this is his lot.
>
> 我所见为善为美的,就是人在 神赐他一生的日子吃喝,享受日光之下劳碌得来的好处,因为这是他的分。


##### 传5:19
> Moreover, when God gives any man wealth and possessions, and enables him to enjoy them, to accept his lot and be happy in his work--this is a gift of God.
>
> 神赐人资财丰富,使他能以吃用,能取自己的分,在他劳碌中喜乐,这乃是 神的恩赐。


##### 传5:20
> He seldom reflects on the days of his life, because God keeps him occupied with gladness of heart.
>
> 他不多思念自己一生的年日,因为 神应他的心使他喜乐。


## 传道书第6章
##### 传6:1
> I have seen another evil under the sun, and it weighs heavily on men:
>
> 我见日光之下有一宗祸患重压在人身上,


##### 传6:2
> God gives a man wealth, possessions and honor, so that he lacks nothing his heart desires, but God does not enable him to enjoy them, and a stranger enjoys them instead. This is meaningless, a grievous evil.
>
> 就是人蒙 神赐他资财、丰富、尊荣,以致他心里所愿的一样都不缺,只是 神使他不能吃用,反有外人来吃用。这是虚空,也是祸患。


##### 传6:3
> A man may have a hundred children and live many years; yet no matter how long he lives, if he cannot enjoy his prosperity and does not receive proper burial, I say that a stillborn child is better off than he.
>
> 人若生一百个儿子,活许多岁数,以致他的年日甚多,心里却不得满享福乐,又不得埋葬;据我说,那不到期而落的胎比他倒好。


##### 传6:4
> It comes without meaning, it departs in darkness, and in darkness its name is shrouded.
>
> 因为虚虚而来,暗暗而去,名字被黑暗遮蔽,


##### 传6:5
> Though it never saw the sun or knew anything, it has more rest than does that man--
>
> 并且没有见过天日,也毫无知觉,这胎比那人倒享安息。


##### 传6:6
> even if he lives a thousand years twice over but fails to enjoy his prosperity. Do not all go to the same place?
>
> 那人虽然活千年,再活千年,却不享福,众人岂不都归一个地方去吗?


##### 传6:7
> All man's efforts are for his mouth, yet his appetite is never satisfied.
>
> 人的劳碌都为口腹,心里却不知足。


##### 传6:8
> What advantage has a wise man over a fool? What does a poor man gain by knowing how to conduct himself before others?
>
> 这样看来,智慧人比愚昧人,有什么长处呢?穷人在众人面前知道如何行,有什么长处呢?


##### 传6:9
> Better what the eye sees than the roving of the appetite. This too is meaningless, a chasing after the wind.
>
> 眼睛所看的,比心里妄想的倒好。这也是虚空,也是捕风。


##### 传6:10
> Whatever exists has already been named, and what man is has been known; no man can contend with one who is stronger than he.
>
> 先前所有的,早已起了名,并知道何为人,他也不能与那比自己力大的相争。


##### 传6:11
> The more the words, the less the meaning, and how does that profit anyone?
>
> 加增虚浮的事既多,这与人有什么益处呢?


##### 传6:12
> For who knows what is good for a man in life, during the few and meaningless days he passes through like a shadow? Who can tell him what will happen under the sun after he is gone?
>
> 人一生虚度的日子,就如影儿经过,谁知道什么与他有益呢?谁能告诉他身后在日光之下有什么事呢?


## 传道书第7章
##### 传7:1
> A good name is better than fine perfume, and the day of death better than the day of birth.
>
> 名誉强如美好的膏油;人死的日子,胜过人生的日子。


##### 传7:2
> It is better to go to a house of mourning than to go to a house of feasting, for death is the destiny of every man; the living should take this to heart.
>
> 往遭丧的家去,强如往宴乐的家去,因为死是众人的结局,活人也必将这事放在心上。


##### 传7:3
> Sorrow is better than laughter, because a sad face is good for the heart.
>
> 忧愁强如喜笑,因为面带愁容,终必使心喜乐。


##### 传7:4
> The heart of the wise is in the house of mourning, but the heart of fools is in the house of pleasure.
>
> 智慧人的心,在遭丧之家;愚昧人的心,在快乐之家。


##### 传7:5
> It is better to heed a wise man's rebuke than to listen to the song of fools.
>
> 听智慧人的责备,强如听愚昧人的歌唱。


##### 传7:6
> Like the crackling of thorns under the pot, so is the laughter of fools. This too is meaningless.
>
> 愚昧人的笑声,好像锅下烧荆棘的爆声,这也是虚空。


##### 传7:7
> Extortion turns a wise man into a fool, and a bribe corrupts the heart.
>
> 勒索使智慧人变为愚妄,贿赂能败坏人的慧心。


##### 传7:8
> The end of a matter is better than its beginning, and patience is better than pride.
>
> 事情的终局,强如事情的起头;存心忍耐的,胜过居心骄傲的。


##### 传7:9
> Do not be quickly provoked in your spirit, for anger resides in the lap of fools.
>
> 你不要心里急躁恼怒,因为恼怒存在愚昧人的怀中。


##### 传7:10
> Do not say, "Why were the old days better than these?" For it is not wise to ask such questions.
>
> 不要说:“先前的日子强过如今的日子,是什么缘故呢?”你这样问,不是出于智慧。


##### 传7:11
> Wisdom, like an inheritance, is a good thing and benefits those who see the sun.
>
> 智慧和产业并好,而且见天日的人得智慧更为有益。


##### 传7:12
> Wisdom is a shelter as money is a shelter, but the advantage of knowledge is this: that wisdom preserves the life of its possessor.
>
> 因为智慧护庇人,好像银钱护庇人一样。惟独智慧能保全智慧人的生命。这就是知识的益处。


##### 传7:13
> Consider what God has done: Who can straighten what he has made crooked?
>
> 你要察看 神的作为,因 神使为曲的,谁能变为直呢?


##### 传7:14
> When times are good, be happy; but when times are bad, consider: God has made the one as well as the other. Therefore, a man cannot discover anything about his future.
>
> 遇亨通的日子,你当喜乐;遭患难的日子,你当思想。因为 神使这两样并列,为的是叫人查不出身后有什么事。


##### 传7:15
> In this meaningless life of mine I have seen both of these: a righteous man perishing in his righteousness, and a wicked man living long in his wickedness.
>
> 有义人行义,反致灭亡;有恶人行恶,倒享长寿。这都是我在虚度之日中所见过的。


##### 传7:16
> Do not be overrighteous, neither be overwise--why destroy yourself?
>
> 不要行义过分,也不要过于自逞智慧,何必自取败亡呢?


##### 传7:17
> Do not be overwicked, and do not be a fool--why die before your time?
>
> 不要行恶过分,也不要为人愚昧,何必不到期而死呢?


##### 传7:18
> It is good to grasp the one and not let go of the other. The man who fears God will avoid all extremes.
>
> 你持守这个为美,那个也不要松手;因为敬畏 神的人,必从这两样出来。


##### 传7:19
> Wisdom makes one wise man more powerful than ten rulers in a city.
>
> 智慧使有智慧的人比城中十个官长更有能力。


##### 传7:20
> There is not a righteous man on earth who does what is right and never sins.
>
> 时常行善而不犯罪的义人,世上实在没有。


##### 传7:21
> Do not pay attention to every word people say, or you may hear your servant cursing you--
>
> 人所说的一切话,你不要放在心上,恐怕听见你的仆人咒诅你。


##### 传7:22
> for you know in your heart that many times you yourself have cursed others.
>
> 因为你心里知道,自己也曾屡次咒诅别人。


##### 传7:23
> All this I tested by wisdom and I said, "I am determined to be wise"--but this was beyond me.
>
> 我曾用智慧试验这一切事,我说“要得智慧”,智慧却离我远。


##### 传7:24
> Whatever wisdom may be, it is far off and most profound--who can discover it?
>
> 万事之理,离我甚远,而且最深,谁能测透呢?


##### 传7:25
> So I turned my mind to understand, to investigate and to search out wisdom and the scheme of things and to understand the stupidity of wickedness and the madness of folly.
>
> 我转念,一心要知道,要考察,要寻求智慧和万事的理由,又要知道邪恶为愚昧,愚昧为狂妄。


##### 传7:26
> I find more bitter than death the woman who is a snare, whose heart is a trap and whose hands are chains. The man who pleases God will escape her, but the sinner she will ensnare.
>
> 我得知有等妇人,比死还苦,她的心是网罗,手是锁链。凡蒙 神喜悦的人,必能躲避她;有罪的人,却被她缠住了。


##### 传7:27
> "Look," says the Teacher, "this is what I have discovered: "Adding one thing to another to discover the scheme of things--
>
> 传道者说,看哪,一千男子中,我找到一个正直人。但众女子中,没有找到一个。


##### 传7:28
> while I was still searching but not finding--I found one upright man among a thousand, but not one upright woman among them all.
>
> 我将这事一一比较,要寻求其理,我心仍要寻找,却未曾找到。


##### 传7:29
> This only have I found: God made mankind upright, but men have gone in search of many schemes."
>
> 我所找到的只有一件:就是 神造人原是正直,但他们寻出许多巧计。”


## 传道书第8章
##### 传8:1
> Who is like the wise man? Who knows the explanation of things? Wisdom brightens a man's face and changes its hard appearance.
>
> 谁如智慧人呢?谁知道事情的解释呢?人的智慧使他的脸发光,并使他脸上的暴气改变。


##### 传8:2
> Obey the king's command, I say, because you took an oath before God.
>
> 我劝你遵守王的命令,既指 神起誓,理当如此。


##### 传8:3
> Do not be in a hurry to leave the king's presence. Do not stand up for a bad cause, for he will do whatever he pleases.
>
> 不要急躁离开王的面前,不要固执行恶,因为他凡事都随自己的心意而行。


##### 传8:4
> Since a king's word is supreme, who can say to him, "What are you doing?"
>
> 王的话本有权力,谁敢问他说:“你做什么呢?”


##### 传8:5
> Whoever obeys his command will come to no harm, and the wise heart will know the proper time and procedure.
>
> 凡遵守命令的,必不经历祸患;智慧人的心,能辨明时候和定理(原文作“审判”。下节同)。


##### 传8:6
> For there is a proper time and procedure for every matter, though a man's misery weighs heavily upon him.
>
> 各样事务成就,都有时候和定理,因为人的苦难重压在他身上。


##### 传8:7
> Since no man knows the future, who can tell him what is to come?
>
> 他不知道将来的事,因为将来如何,谁能告诉他呢?


##### 传8:8
> No man has power over the wind to contain it; so no one has power over the day of his death. As no one is discharged in time of war, so wickedness will not release those who practice it.
>
> 无人有权力掌管生命,将生命留住;也无人有权力掌管死期。这场争战,无人能免,邪恶也不能救那好行邪恶的人。


##### 传8:9
> All this I saw, as I applied my mind to everything done under the sun. There is a time when a man lords it over others to his own hurt.
>
> 这一切我都见过,也专心查考日光之下所做的一切事。有时这人管辖那人,令人受害。


##### 传8:10
> Then too, I saw the wicked buried--those who used to come and go from the holy place and receive praise in the city where they did this. This too is meaningless.
>
> 我见恶人埋葬,归入坟墓;又见行正直事的离开圣地,在城中被人忘记。这也是虚空。


##### 传8:11
> When the sentence for a crime is not quickly carried out, the hearts of the people are filled with schemes to do wrong.
>
> 因为断定罪名,不立刻施刑,所以世人满心作恶。


##### 传8:12
> Although a wicked man commits a hundred crimes and still lives a long time, I know that it will go better with God-fearing men, who are reverent before God.
>
> 罪人虽然作恶百次,倒享长久的年日。然而我准知道,敬畏 神的,就是在他面前敬畏的人,终久必得福乐。


##### 传8:13
> Yet because the wicked do not fear God, it will not go well with them, and their days will not lengthen like a shadow.
>
> 恶人却不得福乐,也不得长久的年日;这年日好像影儿,因他不敬畏 神。


##### 传8:14
> There is something else meaningless that occurs on earth: righteous men who get what the wicked deserve, and wicked men who get what the righteous deserve. This too, I say, is meaningless.
>
> 世上有一件虚空的事,就是义人所遭遇的,反照恶人所行的;又有恶人所遭遇的,反照义人所行的。我说,这也是虚空。


##### 传8:15
> So I commend the enjoyment of life, because nothing is better for a man under the sun than to eat and drink and be glad. Then joy will accompany him in his work all the days of the life God has given him under the sun.
>
> 我就称赞快乐,原来人在日光之下,莫强如吃喝快乐,因为他在日光之下, 神赐他一生的年日,要从劳碌中时常享受所得的。


##### 传8:16
> When I applied my mind to know wisdom and to observe man's labor on earth--his eyes not seeing sleep day or night--
>
> 我专心求智慧,要看世上所做的事。(有昼夜不睡觉,不合眼的。)


##### 传8:17
> then I saw all that God has done. No one can comprehend what goes on under the sun. Despite all his efforts to search it out, man cannot discover its meaning. Even if a wise man claims he knows, he cannot really comprehend it.
>
> 我就看明 神一切的作为,知道人查不出日光之下所做的事;任凭他费多少力寻查,都查不出来,就是智慧人虽想知道,也是查不出来。


## 传道书第9章
##### 传9:1
> So I reflected on all this and concluded that the righteous and the wise and what they do are in God's hands, but no man knows whether love or hate awaits him.
>
> 我将这一切事放在心上,详细考究,就知道义人和智慧人并他们的作为都在 神手中;或是爱或是恨,都在他们的前面,人不能知道。


##### 传9:2
> All share a common destiny--the righteous and the wicked, the good and the bad, the clean and the unclean, those who offer sacrifices and those who do not. As it is with the good man, so with the sinner; as it is with those who take oaths, so with those who are afraid to take them.
>
> 凡临到众人的事都是一样:义人和恶人都遭遇一样的事;好人、洁净人和不洁净人、献祭的与不献祭的,也是一样。好人如何,罪人也如何;起誓的如何,怕起誓的也如何。


##### 传9:3
> This is the evil in everything that happens under the sun: The same destiny overtakes all. The hearts of men, moreover, are full of evil and there is madness in their hearts while they live, and afterward they join the dead.
>
> 在日光之下所行的一切事上,有一件祸患,就是众人所遭遇的都是一样,并且世人的心充满了恶。活着的时候心里狂妄,后来就归死人那里去了。


##### 传9:4
> Anyone who is among the living has hope--even a live dog is better off than a dead lion!
>
> 与一切活人相连的,那人还有指望,因为活着的狗比死了的狮子更强。


##### 传9:5
> For the living know that they will die, but the dead know nothing; they have no further reward, and even the memory of them is forgotten.
>
> 活着的人知道必死,死了的人毫无所知,也不再得赏赐,他们的名无人记念。


##### 传9:6
> Their love, their hate and their jealousy have long since vanished; never again will they have a part in anything that happens under the sun.
>
> 他们的爱,他们的恨,他们的嫉妒,早都消灭了。在日光之下所行的一切事上,他们永不再有分了。


##### 传9:7
> Go, eat your food with gladness, and drink your wine with a joyful heart, for it is now that God favors what you do.
>
> 你只管去欢欢喜喜吃你的饭,心中快乐喝你的酒,因为 神已经悦纳你的作为。


##### 传9:8
> Always be clothed in white, and always anoint your head with oil.
>
> 你的衣服当时常洁白,你头上也不要缺少膏油。


##### 传9:9
> Enjoy life with your wife, whom you love, all the days of this meaningless life that God has given you under the sun--all your meaningless days. For this is your lot in life and in your toilsome labor under the sun.
>
> 在你一生虚空的年日,就是 神赐你在日光之下虚空的年日,当同你所爱的妻快活度日,因为那是你生前在日光之下劳碌的事上所得的分。


##### 传9:10
> Whatever your hand finds to do, do it with all your might, for in the grave, where you are going, there is neither working nor planning nor knowledge nor wisdom.
>
> 凡你手所当做的事,要尽力去做,因为在你所必去的阴间,没有工作,没有谋算,没有知识,也没有智慧。


##### 传9:11
> I have seen something else under the sun: The race is not to the swift or the battle to the strong, nor does food come to the wise or wealth to the brilliant or favor to the learned; but time and chance happen to them all.
>
> 我又转念,见日光之下,快跑的未必能赢;力战的未必得胜;智慧的未必得粮食,明哲的未必得资财,灵巧的未必得喜悦;所临到众人的,是在乎当时的机会。


##### 传9:12
> Moreover, no man knows when his hour will come: As fish are caught in a cruel net, or birds are taken in a snare, so men are trapped by evil times that fall unexpectedly upon them.
>
> 原来人也不知道自己的定期,鱼被恶网圈住,鸟被网罗捉住,祸患忽然临到的时候,世人陷在其中,也是如此。


##### 传9:13
> I also saw under the sun this example of wisdom that greatly impressed me:
>
> 我见日光之下有一样智慧,据我看乃是广大。


##### 传9:14
> There was once a small city with only a few people in it. And a powerful king came against it, surrounded it and built huge siegeworks against it.
>
> 就是有一小城,其中的人数稀少,有大君王来攻击,修筑营垒,将城围困。


##### 传9:15
> Now there lived in that city a man poor but wise, and he saved the city by his wisdom. But nobody remembered that poor man.
>
> 城中有一个贫穷的智慧人,他用智慧救了那城,却没有人记念那穷人。


##### 传9:16
> So I said, "Wisdom is better than strength." But the poor man's wisdom is despised, and his words are no longer heeded.
>
> 我就说:“智慧胜过勇力。”然而那贫穷人的智慧被人藐视,他的话也无人听从。


##### 传9:17
> The quiet words of the wise are more to be heeded than the shouts of a ruler of fools.
>
> 宁可在安静之中听智慧人的言语,不听掌管愚昧人的喊声。


##### 传9:18
> Wisdom is better than weapons of war, but one sinner destroys much good.
>
> 智慧胜过打仗的兵器,但一个罪人能败坏许多善事。


## 传道书第10章
##### 传10:1
> As dead flies give perfume a bad smell, so a little folly outweighs wisdom and honor.
>
> 死苍蝇使做香的膏油发出臭气。这样,一点愚昧,也能败坏智慧和尊荣。


##### 传10:2
> The heart of the wise inclines to the right, but the heart of the fool to the left.
>
> 智慧人的心居右,愚昧人的心居左。


##### 传10:3
> Even as he walks along the road, the fool lacks sense and shows everyone how stupid he is.
>
> 并且愚昧人行路显出无知,对众人说,他是愚昧人。


##### 传10:4
> If a ruler's anger rises against you, do not leave your post; calmness can lay great errors to rest.
>
> 掌权者的心若向你发怒,不要离开你的本位,因为柔和能免大过。


##### 传10:5
> There is an evil I have seen under the sun, the sort of error that arises from a ruler:
>
> 我见日光之下有一件祸患,似乎出于掌权的错误,


##### 传10:6
> Fools are put in many high positions, while the rich occupy the low ones.
>
> 就是愚昧人立在高位,富足人坐在低位。


##### 传10:7
> I have seen slaves on horseback, while princes go on foot like slaves.
>
> 我见过仆人骑马,王子像仆人在地上步行。


##### 传10:8
> Whoever digs a pit may fall into it; whoever breaks through a wall may be bitten by a snake.
>
> 挖陷坑的,自己必掉在其中;拆墙垣的,必为蛇所咬。


##### 传10:9
> Whoever quarries stones may be injured by them; whoever splits logs may be endangered by them.
>
> 凿开(或作“挪移”)石头的,必受损伤;劈开木头的,必遭危险。


##### 传10:10
> If the ax is dull and its edge unsharpened, more strength is needed but skill will bring success.
>
> 铁器钝了,若不将刃磨快,就必多费气力,但得智慧指教,便有益处。


##### 传10:11
> If a snake bites before it is charmed, there is no profit for the charmer.
>
> 未行法术以先,蛇若咬人,后行法术也是无益。


##### 传10:12
> Words from a wise man's mouth are gracious, but a fool is consumed by his own lips.
>
> 智慧人的口说出恩言;愚昧人的嘴吞灭自己。


##### 传10:13
> At the beginning his words are folly; at the end they are wicked madness--
>
> 他口中的言语起头是愚昧;他话的末尾是奸恶的狂妄。


##### 传10:14
> and the fool multiplies words. No one knows what is coming--who can tell him what will happen after him?
>
> 愚昧人多有言语。人却不知将来有什么事;他身后的事,谁能告诉他呢?


##### 传10:15
> A fool's work wearies him; he does not know the way to town.
>
> 凡愚昧人,他的劳碌使自己困乏,因为连进城的路,他也不知道。


##### 传10:16
> Woe to you, O land whose king was a servant and whose princes feast in the morning.
>
> 邦国啊,你的王若是孩童,你的群臣早晨宴乐,你就有祸了!


##### 传10:17
> Blessed are you, O land whose king is of noble birth and whose princes eat at a proper time--for strength and not for drunkenness.
>
> 邦国啊,你的王若是贵胄之子,你的群臣按时吃喝,为要补力,不为酒醉,你就有福了!


##### 传10:18
> If a man is lazy, the rafters sag; if his hands are idle, the house leaks.
>
> 因人懒惰,房顶塌下;因人手懒,房屋滴漏。


##### 传10:19
> A feast is made for laughter, and wine makes life merry, but money is the answer for everything.
>
> 设摆筵席,是为喜笑。酒能使人快活,钱能叫万事应心。


##### 传10:20
> Do not revile the king even in your thoughts, or curse the rich in your bedroom, because a bird of the air may carry your words, and a bird on the wing may report what you say.
>
> 你不可咒诅君王,也不可心怀此念,在你卧房也不可咒诅富户,因为空中的鸟必传扬这声音,有翅膀的也必述说这事。


## 传道书第11章
##### 传11:1
> Cast your bread upon the waters, for after many days you will find it again.
>
> 当将你的粮食撒在水面,因为日久必能得着。


##### 传11:2
> Give portions to seven, yes to eight, for you do not know what disaster may come upon the land.
>
> 你要分给七人,或分给八人,因为你不知道将来有什么灾祸临到地上。


##### 传11:3
> If clouds are full of water, they pour rain upon the earth. Whether a tree falls to the south or to the north, in the place where it falls, there will it lie.
>
> 云若满了雨,就必倾倒在地上;树若向南倒,或向北倒,树倒在何处,就存在何处。


##### 传11:4
> Whoever watches the wind will not plant; whoever looks at the clouds will not reap.
>
> 看风的必不撒种;望云的必不收割。


##### 传11:5
> As you do not know the path of the wind, or how the body is formed in a mother's womb, so you cannot understand the work of God, the Maker of all things.
>
> 风从何道来,骨头在怀孕妇人的胎中如何长成,你尚且不得知道,这样,行万事之神的作为,你更不得知道。


##### 传11:6
> Sow your seed in the morning, and at evening let not your hands be idle, for you do not know which will succeed, whether this or that, or whether both will do equally well.
>
> 早晨要撒你的种,晚上也不要歇你的手,因为你不知道哪一样发旺:或是早撒的,或是晚撒的,或是两样都好。


##### 传11:7
> Light is sweet, and it pleases the eyes to see the sun.
>
> 光本是佳美的,眼见日光也是可悦的。


##### 传11:8
> However many years a man may live, let him enjoy them all. But let him remember the days of darkness, for they will be many. Everything to come is meaningless.
>
> 人活多年,就当快乐多年;然而也当想到黑暗的日子。因为这日子必多,所要来的都是虚空。


##### 传11:9
> Be happy, young man, while you are young, and let your heart give you joy in the days of your youth. Follow the ways of your heart and whatever your eyes see, but know that for all these things God will bring you to judgment.
>
> 少年人哪,你在幼年时当快乐。在幼年的日子,使你的心欢畅,行你心所愿行的,看你眼所爱看的;却要知道,为这一切的事, 神必审问你。


##### 传11:10
> So then, banish anxiety from your heart and cast off the troubles of your body, for youth and vigor are meaningless.
>
> 所以你当从心中除掉愁烦,从肉体克去邪恶,因为一生的开端和幼年之时,都是虚空的。


## 传道书第12章
##### 传12:1
> Remember your Creator in the days of your youth, before the days of trouble come and the years approach when you will say, "I find no pleasure in them"--
>
> 你趁着年幼,衰败的日子尚未来到,就是你所说“我毫无喜乐”的那些年日未曾临近之先,当记念造你的主!


##### 传12:2
> before the sun and the light and the moon and the stars grow dark, and the clouds return after the rain;
>
> 不要等到日头、光明、月亮、星宿变为黑暗,雨后云彩反回;


##### 传12:3
> when the keepers of the house tremble, and the strong men stoop, when the grinders cease because they are few, and those looking through the windows grow dim;
>
> 看守房屋的发颤,有力的屈身,推磨的稀少就止息;从窗户往外看的都昏暗,


##### 传12:4
> when the doors to the street are closed and the sound of grinding fades; when men rise up at the sound of birds, but all their songs grow faint;
>
> 街门关闭,推磨的响声微小,雀鸟一叫,人就起来,歌唱的女子也都衰微。


##### 传12:5
> when men are afraid of heights and of dangers in the streets; when the almond tree blossoms and the grasshopper drags himself along and desire no longer is stirred. Then man goes to his eternal home and mourners go about the streets.
>
> 人怕高处,路上有惊慌;杏树开花,蚱蜢成为重担;人所愿的也都废掉。因为人归他永远的家,吊丧的在街上往来。


##### 传12:6
> Remember him--before the silver cord is severed, or the golden bowl is broken; before the pitcher is shattered at the spring, or the wheel broken at the well,
>
> 银链折断,金罐破裂,瓶子在泉旁损坏,水轮在井口破烂;


##### 传12:7
> and the dust returns to the ground it came from, and the spirit returns to God who gave it.
>
> 尘土仍归于地,灵仍归于赐灵的 神。


##### 传12:8
> "Meaningless! Meaningless!" says the Teacher. "Everything is meaningless!"
>
> 传道者说:“虚空的虚空,凡事都是虚空。”


##### 传12:9
> Not only was the Teacher wise, but also he imparted knowledge to the people. He pondered and searched out and set in order many proverbs.
>
> 再者,传道者因有智慧,仍将知识教训众人;又默想,又考查,又陈说许多箴言。


##### 传12:10
> The Teacher searched to find just the right words, and what he wrote was upright and true.
>
> 传道者专心寻求可喜悦的言语,是凭正直写的诚实话。


##### 传12:11
> The words of the wise are like goads, their collected sayings like firmly embedded nails--given by one Shepherd.
>
> 智慧人的言语好像刺棍;会中之师的言语又像钉稳的钉子,都是一个牧者所赐的。


##### 传12:12
> Be warned, my son, of anything in addition to them. Of making many books there is no end, and much study wearies the body.
>
> 我儿,还有一层,你当受劝戒:著书多,没有穷尽;读书多,身体疲倦。


##### 传12:13
> Now all has been heard; here is the conclusion of the matter: Fear God and keep his commandments, for this is the whole duty of man.
>
> 这些事都已听见了,总意就是敬畏 神,谨守他的诫命,这是人所当尽的本分(或作“这是众人的本分”)。


##### 传12:14
> For God will bring every deed into judgment, including every hidden thing, whether it is good or evil.
>
> 因为人所做的事,连一切隐藏的事,无论是善是恶, 神都必审问。

