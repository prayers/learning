# 哥林多前书
<!-- TOC -->

- [哥林多前书](#哥林多前书)
    - [哥林多前书第1章](#哥林多前书第1章)
    - [哥林多前书第2章](#哥林多前书第2章)
    - [哥林多前书第3章](#哥林多前书第3章)
    - [哥林多前书第4章](#哥林多前书第4章)
    - [哥林多前书第5章](#哥林多前书第5章)
    - [哥林多前书第6章](#哥林多前书第6章)
    - [哥林多前书第7章](#哥林多前书第7章)
    - [哥林多前书第8章](#哥林多前书第8章)
    - [哥林多前书第9章](#哥林多前书第9章)
    - [哥林多前书第10章](#哥林多前书第10章)
    - [哥林多前书第11章](#哥林多前书第11章)
    - [哥林多前书第12章](#哥林多前书第12章)
    - [哥林多前书第13章](#哥林多前书第13章)
    - [哥林多前书第14章](#哥林多前书第14章)
    - [哥林多前书第15章](#哥林多前书第15章)
    - [哥林多前书第16章](#哥林多前书第16章)

<!-- /TOC -->
## 哥林多前书第1章
##### 林前1:1
> Paul, called to be an apostle of Christ Jesus by the will of God, and our brother Sosthenes,
>
> 奉 神旨意,蒙召作耶稣基督使徒的保罗同兄弟所提尼,


##### 林前1:2
> To the church of God in Corinth, to those sanctified in Christ Jesus and called to be holy, together with all those everywhere who call on the name of our Lord Jesus Christ--their Lord and ours:
>
> 写信给在哥林多 神的教会,就是在基督耶稣里成圣,蒙召作圣徒的,以及所有在各处求告我主耶稣基督之名的人。基督是他们的主,也是我们的主。


##### 林前1:3
> Grace and peace to you from God our Father and the Lord Jesus Christ.
>
> 愿恩惠、平安从 神我们的父并主耶稣基督归与你们!


##### 林前1:4
> I always thank God for you because of his grace given you in Christ Jesus.
>
> 我常为你们感谢我的 神,因 神在基督耶稣里所赐给你们的恩惠;


##### 林前1:5
> For in him you have been enriched in every way--in all your speaking and in all your knowledge--
>
> 又因你们在他里面凡事富足,口才、知识都全备。


##### 林前1:6
> because our testimony about Christ was confirmed in you.
>
> 正如我为基督作的见证在你们心里得以坚固,


##### 林前1:7
> Therefore you do not lack any spiritual gift as you eagerly wait for our Lord Jesus Christ to be revealed.
>
> 以致你们在恩赐上没有一样不及人的,等候我们的主耶稣基督显现。


##### 林前1:8
> He will keep you strong to the end, so that you will be blameless on the day of our Lord Jesus Christ.
>
> 他也必坚固你们到底,叫你们在我们主耶稣基督的日子无可责备。


##### 林前1:9
> God, who has called you into fellowship with his Son Jesus Christ our Lord, is faithful.
>
> 神是信实的,你们原是被他所召,好与他儿子,我们的主耶稣基督,一同得分。


##### 林前1:10
> I appeal to you, brothers, in the name of our Lord Jesus Christ, that all of you agree with one another so that there may be no divisions among you and that you may be perfectly united in mind and thought.
>
> 弟兄们,我藉我们主耶稣基督的名,劝你们都说一样的话。你们中间也不可分党,只要一心一意,彼此相合。


##### 林前1:11
> My brothers, some from Chloe's household have informed me that there are quarrels among you.
>
> 因为革来氏家里的人曾对我提起弟兄们来,说你们中间有纷争。


##### 林前1:12
> What I mean is this: One of you says, "I follow Paul"; another, "I follow Apollos"; another, "I follow Cephas"; still another, "I follow Christ."
>
> 我的意思就是你们各人说:“我是属保罗的”,“我是属亚波罗的”,“我是属矶法的”,“我是属基督的”。


##### 林前1:13
> Is Christ divided? Was Paul crucified for you? Were you baptized into the name of Paul?
>
> 基督是分开的吗?保罗为你们钉了十字架吗?你们是奉保罗的名受了洗吗?


##### 林前1:14
> I am thankful that I did not baptize any of you except Crispus and Gaius,
>
> 我感谢 神,除了基利司布并该犹以外,我没有给你们一个人施洗,


##### 林前1:15
> so no one can say that you were baptized into my name.
>
> 免得有人说,你们是奉我的名受洗。


##### 林前1:16
> (Yes, I also baptized the household of Stephanas; beyond that, I don't remember if I baptized anyone else.)
>
> 我也给司提反家施过洗,此外给别人施洗没有,我却记不清。


##### 林前1:17
> For Christ did not send me to baptize, but to preach the gospel--not with words of human wisdom, lest the cross of Christ be emptied of its power.
>
> 基督差遣我,原不是为施洗,乃是为传福音,并不用智慧的言语,免得基督的十字架落了空。


##### 林前1:18
> For the message of the cross is foolishness to those who are perishing, but to us who are being saved it is the power of God.
>
> 因为十字架的道理,在那灭亡的人为愚拙;在我们得救的人却为 神的大能。


##### 林前1:19
> For it is written: "I will destroy the wisdom of the wise; the intelligence of the intelligent I will frustrate."
>
> 就如经上所记:“我要灭绝智慧人的智慧,废弃聪明人的聪明。”


##### 林前1:20
> Where is the wise man? Where is the scholar? Where is the philosopher of this age? Has not God made foolish the wisdom of the world?
>
> 智慧人在哪里?文士在哪里?这世上的辩士在哪里? 神岂不是叫这世上的智慧变成愚拙吗?


##### 林前1:21
> For since in the wisdom of God the world through its wisdom did not know him, God was pleased through the foolishness of what was preached to save those who believe.
>
> 世人凭自己的智慧,既不认识 神, 神就乐意用人所当作愚拙的道理拯救那些信的人,这就是 神的智慧了。


##### 林前1:22
> Jews demand miraculous signs and Greeks look for wisdom,
>
> 犹太人是要神迹,希利尼人是求智慧;


##### 林前1:23
> but we preach Christ crucified: a stumbling block to Jews and foolishness to Gentiles,
>
> 我们却是传钉十字架的基督。在犹太人为绊脚石,在外邦人为愚拙;


##### 林前1:24
> but to those whom God has called, both Jews and Greeks, Christ the power of God and the wisdom of God.
>
> 但在那蒙召的,无论是犹太人、希利尼人,基督总为 神的能力, 神的智慧。


##### 林前1:25
> For the foolishness of God is wiser than man's wisdom, and the weakness of God is stronger than man's strength.
>
> 因 神的愚拙总比人智慧, 神的软弱总比人强壮。


##### 林前1:26
> Brothers, think of what you were when you were called. Not many of you were wise by human standards; not many were influential; not many were of noble birth.
>
> 弟兄们哪,可见你们蒙召的,按着肉体有智慧的不多,有能力的不多,有尊贵的也不多。


##### 林前1:27
> But God chose the foolish things of the world to shame the wise; God chose the weak things of the world to shame the strong.
>
> 神却拣选了世上愚拙的,叫有智慧的羞愧;又拣选了世上软弱的,叫那强壮的羞愧。


##### 林前1:28
> He chose the lowly things of this world and the despised things--and the things that are not--to nullify the things that are,
>
> 神也拣选了世上卑贱的,被人厌恶的,以及那无有的,为要废掉那有的,


##### 林前1:29
> so that no one may boast before him.
>
> 使一切有血气的,在 神面前一个也不能自夸。


##### 林前1:30
> It is because of him that you are in Christ Jesus, who has become for us wisdom from God--that is, our righteousness, holiness and redemption.
>
> 但你们得在基督耶稣里是本乎 神, 神又使他成为我们的智慧、公义、圣洁、救赎。


##### 林前1:31
> Therefore, as it is written: "Let him who boasts boast in the Lord."
>
> 如经上所记:“夸口的,当指着主夸口。”


## 哥林多前书第2章
##### 林前2:1
> When I came to you, brothers, I did not come with eloquence or superior wisdom as I proclaimed to you the testimony about God.
>
> 弟兄们,从前我到你们那里去,并没有用高言大智对你们宣传 神的奥秘。


##### 林前2:2
> For I resolved to know nothing while I was with you except Jesus Christ and him crucified.
>
> 因为我曾定了主意,在你们中间不知道别的,只知道耶稣基督并他钉十字架。


##### 林前2:3
> I came to you in weakness and fear, and with much trembling.
>
> 我在你们那里,又软弱,又惧怕,又甚战兢。


##### 林前2:4
> My message and my preaching were not with wise and persuasive words, but with a demonstration of the Spirit's power,
>
> 我说的话,讲的道,不是用智慧委婉的言语,乃是用圣灵和大能的明证,


##### 林前2:5
> so that your faith might not rest on men's wisdom, but on God's power.
>
> 叫你们的信不在乎人的智慧,只在乎 神的大能。


##### 林前2:6
> We do, however, speak a message of wisdom among the mature, but not the wisdom of this age or of the rulers of this age, who are coming to nothing.
>
> 然而,在完全的人中,我们也讲智慧。但不是这世上的智慧,也不是这世上有权有位将要败亡之人的智慧。


##### 林前2:7
> No, we speak of God's secret wisdom, a wisdom that has been hidden and that God destined for our glory before time began.
>
> 我们讲的,乃是从前所隐藏、 神奥秘的智慧,就是 神在万世以前预定使我们得荣耀的。


##### 林前2:8
> None of the rulers of this age understood it, for if they had, they would not have crucified the Lord of glory.
>
> 这智慧,世上有权有位的人没有一个知道的;他们若知道,就不把荣耀的主钉在十字架上了。


##### 林前2:9
> However, as it is written: "No eye has seen, no ear has heard, no mind has conceived what God has prepared for those who love him"--
>
> 如经上所记:“ 神为爱他的人所预备的,是眼睛未曾看见,耳朵未曾听见,人心也未曾想到的。”


##### 林前2:10
> but God has revealed it to us by his Spirit. The Spirit searches all things, even the deep things of God.
>
> 只有 神藉着圣灵向我们显明了,因为圣灵参透万事,就是 神深奥的事也参透了。


##### 林前2:11
> For who among men knows the thoughts of a man except the man's spirit within him? In the same way no one knows the thoughts of God except the Spirit of God.
>
> 除了在人里头的灵,谁知道人的事?像这样,除了 神的灵,也没有人知道 神的事。


##### 林前2:12
> We have not received the spirit of the world but the Spirit who is from God, that we may understand what God has freely given us.
>
> 我们所领受的,并不是世上的灵,乃是从 神来的灵,叫我们能知道 神开恩赐给我们的事。


##### 林前2:13
> This is what we speak, not in words taught us by human wisdom but in words taught by the Spirit, expressing spiritual truths in spiritual words.
>
> 并且我们讲说这些事,不是用人智慧所指教的言语,乃是用圣灵所指教的言语,将属灵的话解释属灵的事(或作“将属灵的事讲与属灵的人”)。


##### 林前2:14
> The man without the Spirit does not accept the things that come from the Spirit of God, for they are foolishness to him, and he cannot understand them, because they are spiritually discerned.
>
> 然而,属血气的人不领会 神圣灵的事,反倒以为愚拙,并且不能知道,因为这些事惟有属灵的人才能看透。


##### 林前2:15
> The spiritual man makes judgments about all things, but he himself is not subject to any man's judgment:
>
> 属灵的人能看透万事,却没有一人能看透了他。


##### 林前2:16
> "For who has known the mind of the Lord that he may instruct him?" But we have the mind of Christ.
>
> “谁曾知道主的心去教导他呢?”但我们是有基督的心了。


## 哥林多前书第3章
##### 林前3:1
> Brothers, I could not address you as spiritual but as worldly--mere infants in Christ.
>
> 弟兄们,我从前对你们说话,不能把你们当作属灵的,只得把你们当作属肉体,在基督里为婴孩的。


##### 林前3:2
> I gave you milk, not solid food, for you were not yet ready for it. Indeed, you are still not ready.
>
> 我是用奶喂你们,没有用饭喂你们。那时你们不能吃,就是如今还是不能。


##### 林前3:3
> You are still worldly. For since there is jealousy and quarreling among you, are you not worldly? Are you not acting like mere men?
>
> 你们仍是属肉体的。因为在你们中间有嫉妒分争,这岂不是属乎肉体,照着世人的样子行吗?


##### 林前3:4
> For when one says, "I follow Paul," and another, "I follow Apollos," are you not mere men?
>
> 有说:“我是属保罗的。”有说:“我是属亚波罗的。”这岂不是你们和世人一样吗?


##### 林前3:5
> What, after all, is Apollos? And what is Paul? Only servants, through whom you came to believe--as the Lord has assigned to each his task.
>
> 亚波罗算什么?保罗算什么?无非是执事,照主所赐给他们各人的,引导你们相信。


##### 林前3:6
> I planted the seed, Apollos watered it, but God made it grow.
>
> 我栽种了,亚波罗浇灌了,惟有 神叫他生长。


##### 林前3:7
> So neither he who plants nor he who waters is anything, but only God, who makes things grow.
>
> 可见栽种的算不得什么,浇灌的也算不得什么,只在那叫他生长的 神。


##### 林前3:8
> The man who plants and the man who waters have one purpose, and each will be rewarded according to his own labor.
>
> 栽种的和浇灌的,都是一样。但将来各人要照自己的工夫得自己的赏赐。


##### 林前3:9
> For we are God's fellow workers; you are God's field, God's building.
>
> 因为我们是与 神同工的;你们是 神所耕种的田地,所建造的房屋。


##### 林前3:10
> By the grace God has given me, I laid a foundation as an expert builder, and someone else is building on it. But each one should be careful how he builds.
>
> 我照 神所给我的恩,好像一个聪明的工头,立好了根基,有别人在上面建造,只是各人要谨慎怎样在上面建造。


##### 林前3:11
> For no one can lay any foundation other than the one already laid, which is Jesus Christ.
>
> 因为那已经立好的根基就是耶稣基督,此外没有人能立别的根基。


##### 林前3:12
> If any man builds on this foundation using gold, silver, costly stones, wood, hay or straw,
>
> 若有人用金、银、宝石、草木、禾秸在这根基上建造,


##### 林前3:13
> his work will be shown for what it is, because the Day will bring it to light. It will be revealed with fire, and the fire will test the quality of each man's work.
>
> 各人的工程必然显露,因为那日子要将它表明出来,有火发现,这火要试验各人的工程怎样。


##### 林前3:14
> If what he has built survives, he will receive his reward.
>
> 人在那根基上所建造的工程若存得住,他就要得赏赐;


##### 林前3:15
> If it is burned up, he will suffer loss; he himself will be saved, but only as one escaping through the flames.
>
> 人的工程若被烧了,他就要受亏损,自己却要得救;虽然得救,乃像从火里经过的一样。


##### 林前3:16
> Don't you know that you yourselves are God's temple and that God's Spirit lives in you?
>
> 岂不知你们是 神的殿, 神的灵住在你们里头吗?


##### 林前3:17
> If anyone destroys God's temple, God will destroy him; for God's temple is sacred, and you are that temple.
>
> 若有人毁坏 神的殿, 神必要毁坏那人,因为 神的殿是圣的,这殿就是你们。


##### 林前3:18
> Do not deceive yourselves. If any one of you thinks he is wise by the standards of this age, he should become a "fool" so that he may become wise.
>
> 人不可自欺。你们中间若有人在这世界自以为有智慧,倒不如变作愚拙,好成为有智慧的。


##### 林前3:19
> For the wisdom of this world is foolishness in God's sight. As it is written: "He catches the wise in their craftiness";
>
> 因这世界的智慧,在 神看是愚拙。如经上记着说:“主叫有智慧的,中了自己的诡计。”


##### 林前3:20
> and again, "The Lord knows that the thoughts of the wise are futile."
>
> 又说:“主知道智慧人的意念是虚妄的。”


##### 林前3:21
> So then, no more boasting about men! All things are yours,
>
> 所以无论谁,都不可拿人夸口,因为万有全是你们的。


##### 林前3:22
> whether Paul or Apollos or Cephas or the world or life or death or the present or the future--all are yours,
>
> 或保罗,或亚波罗,或矶法,或世界,或生,或死,或现今的事,或将来的事,全是你们的。


##### 林前3:23
> and you are of Christ, and Christ is of God.
>
> 并且你们是属基督的,基督又是属 神的。


## 哥林多前书第4章
##### 林前4:1
> So then, men ought to regard us as servants of Christ and as those entrusted with the secret things of God.
>
> 人应当以我们为基督的执事,为 神奥秘事的管家。


##### 林前4:2
> Now it is required that those who have been given a trust must prove faithful.
>
> 所求于管家的,是要他有忠心。


##### 林前4:3
> I care very little if I am judged by you or by any human court; indeed, I do not even judge myself.
>
> 我被你们论断,或被别人论断,我都以为极小的事,连我自己也不论断自己。


##### 林前4:4
> My conscience is clear, but that does not make me innocent. It is the Lord who judges me.
>
> 我虽不觉得自己有错,却也不能因此得以称义;但判断我的乃是主。


##### 林前4:5
> Therefore judge nothing before the appointed time; wait till the Lord comes. He will bring to light what is hidden in darkness and will expose the motives of men's hearts. At that time each will receive his praise from God.
>
> 所以,时候未到,什么都不要论断,只等主来,他要照出暗中的隐情,显明人心的意念。那时,各人要从 神那里得着称赞。


##### 林前4:6
> Now, brothers, I have applied these things to myself and Apollos for your benefit, so that you may learn from us the meaning of the saying, "Do not go beyond what is written." Then you will not take pride in one man over against another.
>
> 弟兄们,我为你们的缘故,拿这些事转比自己和亚波罗,叫你们效法我们不可过于圣经所记,免得你们自高自大,贵重这个,轻看那个。


##### 林前4:7
> For who makes you different from anyone else? What do you have that you did not receive? And if you did receive it, why do you boast as though you did not?
>
> 使你与人不同的是谁呢?你有什么不是领受的呢?若是领受的,为何自夸,仿佛不是领受的呢?


##### 林前4:8
> Already you have all you want! Already you have become rich! You have become kings--and that without us! How I wish that you really had become kings so that we might be kings with you!
>
> 你们已经饱足了,已经丰富了,不用我们,自己就作王了。我愿意你们果真作王,叫我们也得与你们一同作王。


##### 林前4:9
> For it seems to me that God has put us apostles on display at the end of the procession, like men condemned to die in the arena. We have been made a spectacle to the whole universe, to angels as well as to men.
>
> 我想 神把我们使徒明明列在末后,好像定死罪的囚犯;因为我们成了一台戏,给世人和天使观看。


##### 林前4:10
> We are fools for Christ, but you are so wise in Christ! We are weak, but you are strong! You are honored, we are dishonored!
>
> 我们为基督的缘故算是愚拙的,你们在基督里倒是聪明的;我们软弱,你们倒强壮;你们有荣耀,我们倒被藐视。


##### 林前4:11
> To this very hour we go hungry and thirsty, we are in rags, we are brutally treated, we are homeless.
>
> 直到如今,我们还是又饥、又渴、又赤身露体、又挨打、又没有一定的住处,


##### 林前4:12
> We work hard with our own hands. When we are cursed, we bless; when we are persecuted, we endure it;
>
> 并且劳苦,亲手做工。被人咒骂,我们就祝福;被人逼迫,我们就忍受;


##### 林前4:13
> when we are slandered, we answer kindly. Up to this moment we have become the scum of the earth, the refuse of the world.
>
> 被人毁谤,我们就善劝。直到如今,人还把我们看作世界上的污秽,万物中的渣滓。


##### 林前4:14
> I am not writing this to shame you, but to warn you, as my dear children.
>
> 我写这话,不是叫你们羞愧,乃是警戒你们,好像我所亲爱的儿女一样。


##### 林前4:15
> Even though you have ten thousand guardians in Christ, you do not have many fathers, for in Christ Jesus I became your father through the gospel.
>
> 你们学基督的,师傅虽有一万,为父的却是不多,因我在基督耶稣里用福音生了你们。


##### 林前4:16
> Therefore I urge you to imitate me.
>
> 所以,我求你们效法我。


##### 林前4:17
> For this reason I am sending to you Timothy, my son whom I love, who is faithful in the Lord. He will remind you of my way of life in Christ Jesus, which agrees with what I teach everywhere in every church.
>
> 因此我已打发提摩太到你们那里去;他在主里面,是我所亲爱、有忠心的儿子。他必提醒你们,记念我在基督里怎样行事,在各处各教会中怎样教导人。


##### 林前4:18
> Some of you have become arrogant, as if I were not coming to you.
>
> 有些人自高自大,以为我不到你们那里去;


##### 林前4:19
> But I will come to you very soon, if the Lord is willing, and then I will find out not only how these arrogant people are talking, but what power they have.
>
> 然而主若许我,我必快到你们那里去;并且我所要知道的,不是那些自高自大之人的言语,乃是他们的权能。


##### 林前4:20
> For the kingdom of God is not a matter of talk but of power.
>
> 因为 神的国不在乎言语,乃在乎权能。


##### 林前4:21
> What do you prefer? Shall I come to you with a whip, or in love and with a gentle spirit?
>
> 你们愿意怎么样呢?是愿意我带着刑杖到你们那里去呢?还是要我存慈爱温柔的心呢?


## 哥林多前书第5章
##### 林前5:1
> It is actually reported that there is sexual immorality among you, and of a kind that does not occur even among pagans: A man has his father's wife.
>
> 风闻在你们中间有淫乱的事。这样的淫乱连外邦人中也没有,就是有人收了他的继母。


##### 林前5:2
> And you are proud! Shouldn't you rather have been filled with grief and have put out of your fellowship the man who did this?
>
> 你们还是自高自大,并不哀痛,把行这事的人从你们中间赶出去。


##### 林前5:3
> Even though I am not physically present, I am with you in spirit. And I have already passed judgment on the one who did this, just as if I were present.
>
> 我身子虽不在你们那里,心却在你们那里,好像我亲自与你们同在,已经判断了行这事的人。


##### 林前5:4
> When you are assembled in the name of our Lord Jesus and I am with you in spirit, and the power of our Lord Jesus is present,
>
> 就是你们聚会的时候,我的心也同在。奉我们主耶稣的名,并用我们主耶稣的权能,


##### 林前5:5
> hand this man over to Satan, so that the sinful nature may be destroyed and his spirit saved on the day of the Lord.
>
> 要把这样的人交给撒但,败坏他的肉体,使他的灵魂在主耶稣的日子可以得救。


##### 林前5:6
> Your boasting is not good. Don't you know that a little yeast works through the whole batch of dough?
>
> 你们这自夸是不好的,岂不知一点面酵能使全团发起来吗?


##### 林前5:7
> Get rid of the old yeast that you may be a new batch without yeast--as you really are. For Christ, our Passover lamb, has been sacrificed.
>
> 你们既是无酵的面,应当把旧酵除净,好使你们成为新团;因为我们逾越节的羔羊基督,已经被杀献祭了。


##### 林前5:8
> Therefore let us keep the Festival, not with the old yeast, the yeast of malice and wickedness, but with bread without yeast, the bread of sincerity and truth.
>
> 所以我们守这节不可用旧酵,也不可用恶毒(或作“阴毒”)、邪恶的酵,只用诚实真正的无酵饼。


##### 林前5:9
> I have written you in my letter not to associate with sexually immoral people--
>
> 我先前写信给你们说:不可与淫乱的人相交。


##### 林前5:10
> not at all meaning the people of this world who are immoral, or the greedy and swindlers, or idolaters. In that case you would have to leave this world.
>
> 此话不是指这世上一概行淫乱的,或贪婪的、勒索的,或拜偶像的;若是这样,你们除非离开世界方可。


##### 林前5:11
> But now I am writing you that you must not associate with anyone who calls himself a brother but is sexually immoral or greedy, an idolater or a slanderer, a drunkard or a swindler. With such a man do not even eat.
>
> 但如今我写信给你们说:若有称为弟兄是行淫乱的,或贪婪的,或拜偶像的,或辱骂的,或醉酒的,或勒索的,这样的人不可与他相交,就是与他吃饭都不可。


##### 林前5:12
> What business is it of mine to judge those outside the church? Are you not to judge those inside?
>
> 因为审判教外的人与我何干?教内的人岂不是你们审判的吗?


##### 林前5:13
> God will judge those outside. "Expel the wicked man from among you."
>
> 至于外人,有 神审判他们。你们应当把那恶人从你们中间赶出去。


## 哥林多前书第6章
##### 林前6:1
> If any of you has a dispute with another, dare he take it before the ungodly for judgment instead of before the saints?
>
> 你们中间有彼此相争的事,怎敢在不义的人面前求审,不在圣徒面前求审呢?


##### 林前6:2
> Do you not know that the saints will judge the world? And if you are to judge the world, are you not competent to judge trivial cases?
>
> 岂不知圣徒要审判世界吗?若世界为你们所审,难道你们不配审判这最小的事吗?


##### 林前6:3
> Do you not know that we will judge angels? How much more the things of this life!
>
> 岂不知我们要审判天使吗?何况今生的事呢!


##### 林前6:4
> Therefore, if you have disputes about such matters, appoint as judges even men of little account in the church!
>
> 既是这样,你们若有今生的事当审判,是派教会所轻看的人审判吗?


##### 林前6:5
> I say this to shame you. Is it possible that there is nobody among you wise enough to judge a dispute between believers?
>
> 我说这话是要叫你们羞耻。难道你们中间没有一个智慧人能审断弟兄们的事吗?


##### 林前6:6
> But instead, one brother goes to law against another--and this in front of unbelievers!
>
> 你们竟是弟兄与弟兄告状,而且告在不信主的人面前。


##### 林前6:7
> The very fact that you have lawsuits among you means you have been completely defeated already. Why not rather be wronged? Why not rather be cheated?
>
> 你们彼此告状,这已经是你们的大错了。为什么不情愿受欺呢?为什么不情愿吃亏呢?


##### 林前6:8
> Instead, you yourselves cheat and do wrong, and you do this to your brothers.
>
> 你们倒是欺压人、亏负人,况且所欺压、所亏负的就是弟兄!


##### 林前6:9
> Do you not know that the wicked will not inherit the kingdom of God? Do not be deceived: Neither the sexually immoral nor idolaters nor adulterers nor male prostitutes nor homosexual offenders
>
> 你们岂不知不义的人不能承受 神的国吗?不要自欺!无论是淫乱的、拜偶像的、奸淫的、作娈童的、亲男色的、


##### 林前6:10
> nor thieves nor the greedy nor drunkards nor slanderers nor swindlers will inherit the kingdom of God.
>
> 偷窃的、贪婪的、醉酒的、辱骂的、勒索的,都不能承受 神的国。


##### 林前6:11
> And that is what some of you were. But you were washed, you were sanctified, you were justified in the name of the Lord Jesus Christ and by the Spirit of our God.
>
> 你们中间也有人从前是这样;但如今你们奉主耶稣基督的名,并藉着我们 神的灵,已经洗净、成圣、称义了。


##### 林前6:12
> "Everything is permissible for me"--but not everything is beneficial. "Everything is permissible for me"--but I will not be mastered by anything.
>
> 凡事我都可行,但不都有益处;凡事我都可行,但无论哪一件,我总不受它的辖制。


##### 林前6:13
> "Food for the stomach and the stomach for food"--but God will destroy them both. The body is not meant for sexual immorality, but for the Lord, and the Lord for the body.
>
> 食物是为肚腹,肚腹是为食物;但 神要叫这两样都废坏。身子不是为淫乱,乃是为主;主也是为身子。


##### 林前6:14
> By his power God raised the Lord from the dead, and he will raise us also.
>
> 并且 神已经叫主复活,也要用自己的能力叫我们复活。


##### 林前6:15
> Do you not know that your bodies are members of Christ himself? Shall I then take the members of Christ and unite them with a prostitute? Never!
>
> 岂不知你们的身子是基督的肢体吗?我可以将基督的肢体作为娼妓的肢体吗?断乎不可!


##### 林前6:16
> Do you not know that he who unites himself with a prostitute is one with her in body? For it is said, "The two will become one flesh."
>
> 岂不知与娼妓联合的,便是与她成为一体吗?因为主说:“二人要成为一体。”


##### 林前6:17
> But he who unites himself with the Lord is one with him in spirit.
>
> 但与主联合的,便是与主成为一灵。


##### 林前6:18
> Flee from sexual immorality. All other sins a man commits are outside his body, but he who sins sexually sins against his own body.
>
> 你们要逃避淫行。人所犯的,无论什么罪,都在身子以外;惟有行淫的,是得罪自己的身子。


##### 林前6:19
> Do you not know that your body is a temple of the Holy Spirit, who is in you, whom you have received from God? You are not your own;
>
> 岂不知你们的身子就是圣灵的殿吗?这圣灵是从 神而来,住在你们里头的;并且你们不是自己的人,


##### 林前6:20
> you were bought at a price. Therefore honor God with your body.
>
> 因为你们是重价买来的,所以要在你们的身子上荣耀 神。


## 哥林多前书第7章
##### 林前7:1
> Now for the matters you wrote about: It is good for a man not to marry.
>
> 论到你们信上所提的事,我说男不近女倒好。


##### 林前7:2
> But since there is so much immorality, each man should have his own wife, and each woman her own husband.
>
> 但要免淫乱的事,男子当各有自己的妻子,女子也当各有自己的丈夫。


##### 林前7:3
> The husband should fulfill his marital duty to his wife, and likewise the wife to her husband.
>
> 丈夫当用合宜之分待妻子,妻子待丈夫也要如此。


##### 林前7:4
> The wife's body does not belong to her alone but also to her husband. In the same way, the husband's body does not belong to him alone but also to his wife.
>
> 妻子没有权柄主张自己的身子,乃在丈夫;丈夫也没有权柄主张自己的身子,乃在妻子。


##### 林前7:5
> Do not deprive each other except by mutual consent and for a time, so that you may devote yourselves to prayer. Then come together again so that Satan will not tempt you because of your lack of self-control.
>
> 夫妻不可彼此亏负,除非两相情愿,暂时分房,为要专心祷告方可;以后仍要同房,免得撒但趁着你们情不自禁引诱你们。


##### 林前7:6
> I say this as a concession, not as a command.
>
> 我说这话,原是准你们的,不是命你们的。


##### 林前7:7
> I wish that all men were as I am. But each man has his own gift from God; one has this gift, another has that.
>
> 我愿意众人像我一样,只是各人领受 神的恩赐,一个是这样,一个是那样。


##### 林前7:8
> Now to the unmarried and the widows I say: It is good for them to stay unmarried, as I am.
>
> 我对着没有嫁娶的和寡妇说,若他们常像我就好。


##### 林前7:9
> But if they cannot control themselves, they should marry, for it is better to marry than to burn with passion.
>
> 倘若自己禁止不住,就可以嫁娶。与其欲火攻心,倒不如嫁娶为妙。


##### 林前7:10
> To the married I give this command (not I, but the Lord): A wife must not separate from her husband.
>
> 至于那已经嫁娶的,我吩咐他们,其实不是我吩咐,乃是主吩咐说:“妻子不可离开丈夫。


##### 林前7:11
> But if she does, she must remain unmarried or else be reconciled to her husband. And a husband must not divorce his wife.
>
> 若是离开了,不可再嫁,或是仍同丈夫和好。丈夫也不可离弃妻子。”


##### 林前7:12
> To the rest I say this (I, not the Lord): If any brother has a wife who is not a believer and she is willing to live with him, he must not divorce her.
>
> 我对其余的人说,不是主说,倘若某弟兄有不信的妻子,妻子也情愿和他同住,他就不要离弃妻子。


##### 林前7:13
> And if a woman has a husband who is not a believer and he is willing to live with her, she must not divorce him.
>
> 妻子有不信的丈夫,丈夫也情愿和她同住,她就不要离弃丈夫。


##### 林前7:14
> For the unbelieving husband has been sanctified through his wife, and the unbelieving wife has been sanctified through her believing husband. Otherwise your children would be unclean, but as it is, they are holy.
>
> 因为不信的丈夫就因着妻子成了圣洁,并且不信的妻子就因着丈夫成了圣洁(“丈夫”原文作“弟兄”)。不然,你们的儿女就不洁净,但如今他们是圣洁的了。


##### 林前7:15
> But if the unbeliever leaves, let him do so. A believing man or woman is not bound in such circumstances; God has called us to live in peace.
>
> 倘若那不信的人要离去,就由他离去吧!无论是弟兄,是姐妹,遇着这样的事都不必拘束。 神召我们原是要我们和睦。


##### 林前7:16
> How do you know, wife, whether you will save your husband? Or, how do you know, husband, whether you will save your wife?
>
> 你这作妻子的,怎么知道不能救你的丈夫呢?你这作丈夫的,怎么知道不能救你的妻子呢?


##### 林前7:17
> Nevertheless, each one should retain the place in life that the Lord assigned to him and to which God has called him. This is the rule I lay down in all the churches.
>
> 只要照主所分给各人的,和 神所召各人的而行。我吩咐各教会都是这样。


##### 林前7:18
> Was a man already circumcised when he was called? He should not become uncircumcised. Was a man uncircumcised when he was called? He should not be circumcised.
>
> 有人已受割礼蒙召呢,就不要废割礼;有人未受割礼蒙召呢,就不要受割礼。


##### 林前7:19
> Circumcision is nothing and uncircumcision is nothing. Keeping God's commands is what counts.
>
> 受割礼算不得什么,不受割礼也算不得什么,只要守 神的诫命就是了。


##### 林前7:20
> Each one should remain in the situation which he was in when God called him.
>
> 各人蒙召的时候是什么身分,仍要守住这身分。


##### 林前7:21
> Were you a slave when you were called? Don't let it trouble you--although if you can gain your freedom, do so.
>
> 你是作奴隶蒙召的吗?不要因此忧虑。若能以自由,就求自由更好。


##### 林前7:22
> For he who was a slave when he was called by the Lord is the Lord's freedman; similarly, he who was a free man when he was called is Christ's slave.
>
> 因为作奴仆蒙召于主的,就是主所释放的人;作自由之人蒙召的,就是基督的奴仆。


##### 林前7:23
> You were bought at a price; do not become slaves of men.
>
> 你们是重价买来的,不要作人的奴仆。


##### 林前7:24
> Brothers, each man, as responsible to God, should remain in the situation God called him to.
>
> 弟兄们,你们各人蒙召的时候是什么身分,仍要在 神面前守住这身分。


##### 林前7:25
> Now about virgins: I have no command from the Lord, but I give a judgment as one who by the Lord's mercy is trustworthy.
>
> 论到童身的人,我没有主的命令,但我既蒙主怜恤能作忠心的人,就把自己的意见告诉你们。


##### 林前7:26
> Because of the present crisis, I think that it is good for you to remain as you are.
>
> 因现今的艰难,据我看来,人不如守素安常才好。


##### 林前7:27
> Are you married? Do not seek a divorce. Are you unmarried? Do not look for a wife.
>
> 你有妻子缠着呢,就不要求脱离;你没有妻子缠着呢,就不要求妻子。


##### 林前7:28
> But if you do marry, you have not sinned; and if a virgin marries, she has not sinned. But those who marry will face many troubles in this life, and I want to spare you this.
>
> 你若娶妻,并不是犯罪;处女若出嫁,也不是犯罪。然而这等人肉身必受苦难,我却愿意你们免这苦难。


##### 林前7:29
> What I mean, brothers, is that the time is short. From now on those who have wives should live as if they had none;
>
> 弟兄们,我对你们说,时候减少了。从此以后,那有妻子的,要像没有妻子;


##### 林前7:30
> those who mourn, as if they did not; those who are happy, as if they were not; those who buy something, as if it were not theirs to keep;
>
> 哀哭的,要像不哀哭;快乐的,要像不快乐;置买的,要像无有所得;


##### 林前7:31
> those who use the things of the world, as if not engrossed in them. For this world in its present form is passing away.
>
> 用世物的,要像不用世物;因为这世界的样子将要过去了。


##### 林前7:32
> I would like you to be free from concern. An unmarried man is concerned about the Lord's affairs--how he can please the Lord.
>
> 我愿你们无所挂虑。没有娶妻的,是为主的事挂虑,想怎样叫主喜悦;


##### 林前7:33
> But a married man is concerned about the affairs of this world--how he can please his wife--
>
> 娶了妻的,是为世上的事挂虑,想怎样叫妻子喜悦。


##### 林前7:34
> and his interests are divided. An unmarried woman or virgin is concerned about the Lord's affairs: Her aim is to be devoted to the Lord in both body and spirit. But a married woman is concerned about the affairs of this world--how she can please her husband.
>
> 妇人和处女也有分别。没有出嫁的,是为主的事挂虑,要身体、灵魂都圣洁;已经出嫁的,是为世上的事挂虑,想怎样叫丈夫喜悦。


##### 林前7:35
> I am saying this for your own good, not to restrict you, but that you may live in a right way in undivided devotion to the Lord.
>
> 我说这话是为你们的益处,不是要牢笼你们,乃是要叫你们行合宜的事,得以殷勤服事主,没有分心的事。


##### 林前7:36
> If anyone thinks he is acting improperly toward the virgin he is engaged to, and if she is getting along in years and he feels he ought to marry, he should do as he wants. He is not sinning. They should get married.
>
> 若有人以为自己待他的女儿不合宜,女儿也过了年岁,事又当行,他就可随意办理,不算有罪,叫二人成亲就是了。


##### 林前7:37
> But the man who has settled the matter in his own mind, who is under no compulsion but has control over his own will, and who has made up his mind not to marry the virgin--this man also does the right thing.
>
> 倘若人心里坚定,没有不得已的事,并且由得自己作主,心里又决定了留下女儿不出嫁,如此行也好。


##### 林前7:38
> So then, he who marries the virgin does right, but he who does not marry her does even better.
>
> 这样看来,叫自己的女儿出嫁是好,不叫她出嫁更是好。


##### 林前7:39
> A woman is bound to her husband as long as he lives. But if her husband dies, she is free to marry anyone she wishes, but he must belong to the Lord.
>
> 丈夫活着的时候,妻子是被约束的;丈夫若死了,妻子就可以自由,随意再嫁,只是要嫁这在主里面的人。


##### 林前7:40
> In my judgment, she is happier if she stays as she is--and I think that I too have the Spirit of God.
>
> 然而按我的意见,若常守节更有福气。我也想自己是被 神的灵感动了。


## 哥林多前书第8章
##### 林前8:1
> Now about food sacrificed to idols: We know that we all possess knowledge. Knowledge puffs up, but love builds up.
>
> 论到祭偶像之物,我们晓得我们都有知识。但知识是叫人自高自大,惟有爱心能造就人。


##### 林前8:2
> The man who thinks he knows something does not yet know as he ought to know.
>
> 若有人以为自己知道什么,按他所当知道的,他仍是不知道。


##### 林前8:3
> But the man who loves God is known by God.
>
> 若有人爱 神,这人乃是 神所知道的。


##### 林前8:4
> So then, about eating food sacrificed to idols: We know that an idol is nothing at all in the world and that there is no God but one.
>
> 论到吃祭偶像之物,我们知道偶像在世上算不得什么,也知道 神只有一位,再没有别的 神。


##### 林前8:5
> For even if there are so-called gods, whether in heaven or on earth (as indeed there are many "gods" and many "lords"),
>
> 虽有称为 神的,或在天、或在地,就如那许多的 神,许多的主;


##### 林前8:6
> yet for us there is but one God, the Father, from whom all things came and for whom we live; and there is but one Lord, Jesus Christ, through whom all things came and through whom we live.
>
> 然而我们只有一位 神,就是父,万物都本于他,我们也归于他;并有一位主,就是耶稣基督,万物都是藉着他有的,我们也是藉着他有的。


##### 林前8:7
> But not everyone knows this. Some people are still so accustomed to idols that when they eat such food they think of it as having been sacrificed to an idol, and since their conscience is weak, it is defiled.
>
> 但人不都有这等知识。有人到如今因拜惯了偶像,就以为所吃的是祭偶像之物,他们的良心既然软弱,也就污秽了。


##### 林前8:8
> But food does not bring us near to God; we are no worse if we do not eat, and no better if we do.
>
> 其实食物不能叫 神看中我们,因为我们不吃也无损,吃也无益。


##### 林前8:9
> Be careful, however, that the exercise of your freedom does not become a stumbling block to the weak.
>
> 只是你们要谨慎,恐怕你们这自由竟成了那软弱人的绊脚石。


##### 林前8:10
> For if anyone with a weak conscience sees you who have this knowledge eating in an idol's temple, won't he be emboldened to eat what has been sacrificed to idols?
>
> 若有人见你这有知识的在偶像的庙里坐席,这人的良心若是软弱,岂不放胆去吃那祭偶像之物吗?


##### 林前8:11
> So this weak brother, for whom Christ died, is destroyed by your knowledge.
>
> 因此,基督为他死的那软弱弟兄,也就因你的知识沉沦了。


##### 林前8:12
> When you sin against your brothers in this way and wound their weak conscience, you sin against Christ.
>
> 你们这样得罪弟兄们,伤了他们软弱的良心,就是得罪基督。


##### 林前8:13
> Therefore, if what I eat causes my brother to fall into sin, I will never eat meat again, so that I will not cause him to fall.
>
> 所以,食物若叫我弟兄跌倒,我就永远不吃肉,免得叫我弟兄跌倒了。


## 哥林多前书第9章
##### 林前9:1
> Am I not free? Am I not an apostle? Have I not seen Jesus our Lord? Are you not the result of my work in the Lord?
>
> 我不是自由的吗?我不是使徒吗?我不是见过我们的主耶稣吗?你们不是我在主里面所做之工吗?


##### 林前9:2
> Even though I may not be an apostle to others, surely I am to you! For you are the seal of my apostleship in the Lord.
>
> 假若在别人我不是使徒,在你们我总是使徒。因为你们在主里正是我作使徒的印证。


##### 林前9:3
> This is my defense to those who sit in judgment on me.
>
> 我对那盘问我的人就是这样分诉。


##### 林前9:4
> Don't we have the right to food and drink?
>
> 难道我们没有权柄靠福音吃喝吗?


##### 林前9:5
> Don't we have the right to take a believing wife along with us, as do the other apostles and the Lord's brothers and Cephas?
>
> 难道我们没有权柄娶信主的姊妹为妻,带着一同往来,仿佛其余的使徒和主的弟兄,并矶法一样吗?


##### 林前9:6
> Or is it only I and Barnabas who must work for a living?
>
> 独有我与巴拿巴没有权柄不做工吗?


##### 林前9:7
> Who serves as a soldier at his own expense? Who plants a vineyard and does not eat of its grapes? Who tends a flock and does not drink of the milk?
>
> 有谁当兵自备粮饷呢?有谁栽葡萄园不吃园里的果子呢?有谁牧养牛羊不吃牛羊的奶呢?


##### 林前9:8
> Do I say this merely from a human point of view? Doesn't the Law say the same thing?
>
> 我说这话,岂是照人的意见?律法不也是这样说吗?


##### 林前9:9
> For it is written in the Law of Moses: "Do not muzzle an ox while it is treading out the grain." Is it about oxen that God is concerned?
>
> 就如摩西的律法记着说:“牛在场上踹谷的时候,不可笼住它的嘴。”难道 神所挂念的是牛吗?


##### 林前9:10
> Surely he says this for us, doesn't he? Yes, this was written for us, because when the plowman plows and the thresher threshes, they ought to do so in the hope of sharing in the harvest.
>
> 不全是为我们说的吗?分明是为我们说的。因为耕种的当存着指望去耕种;打场的也当存得粮的指望去打场。


##### 林前9:11
> If we have sown spiritual seed among you, is it too much if we reap a material harvest from you?
>
> 我们若把属灵的种子撒在你们中间,就是从你们收割奉养肉身之物,这还算大事吗?


##### 林前9:12
> If others have this right of support from you, shouldn't we have it all the more? But we did not use this right. On the contrary, we put up with anything rather than hinder the gospel of Christ.
>
> 若别人在你们身上有这权柄,何况我们呢?然而,我们没有用过这权柄,倒凡事忍受,免得基督的福音被阻隔。


##### 林前9:13
> Don't you know that those who work in the temple get their food from the temple, and those who serve at the altar share in what is offered on the altar?
>
> 你们岂不知为圣事劳碌的,就吃殿中的物吗?伺候祭坛的,就分领坛上的物吗?


##### 林前9:14
> In the same way, the Lord has commanded that those who preach the gospel should receive their living from the gospel.
>
> 主也是这样命定,叫传福音的靠着福音养生。


##### 林前9:15
> But I have not used any of these rights. And I am not writing this in the hope that you will do such things for me. I would rather die than have anyone deprive me of this boast.
>
> 但这权柄我全没有用过。我写这话,并非要你们这样待我,因为我宁可死,也不叫人使我所夸的落了空!


##### 林前9:16
> Yet when I preach the gospel, I cannot boast, for I am compelled to preach. Woe to me if I do not preach the gospel!
>
> 我传福音原没有可夸的,因为我是不得已的;若不传福音,我便有祸了。


##### 林前9:17
> If I preach voluntarily, I have a reward; if not voluntarily, I am simply discharging the trust committed to me.
>
> 我若甘心做这事,就有赏赐;若不甘心,责任却已经托付我了。


##### 林前9:18
> What then is my reward? Just this: that in preaching the gospel I may offer it free of charge, and so not make use of my rights in preaching it.
>
> 既是这样,我的赏赐是什么呢?就是我传福音的时候,叫人不花钱得福音,免得用尽我传福音的权柄。


##### 林前9:19
> Though I am free and belong to no man, I make myself a slave to everyone, to win as many as possible.
>
> 我虽是自由的,无人辖管,然而我甘心作了众人的仆人,为要多得人。


##### 林前9:20
> To the Jews I became like a Jew, to win the Jews. To those under the law I became like one under the law (though I myself am not under the law), so as to win those under the law.
>
> 向犹太人,我就作犹太人,为要得犹太人;向律法以下的人,我虽不在律法以下,还是作律法以下的人,为要得律法以下的人;


##### 林前9:21
> To those not having the law I became like one not having the law (though I am not free from God's law but am under Christ's law), so as to win those not having the law.
>
> 向没有律法的人,我就作没有律法的人,为要得没有律法的人。其实我在 神面前,不是没有律法;在基督面前,正在律法之下。


##### 林前9:22
> To the weak I became weak, to win the weak. I have become all things to all men so that by all possible means I might save some.
>
> 向软弱的人,我就作软弱的人,为要得软弱的人;向什么样的人,我就作什么样的人。无论如何总要救些人。


##### 林前9:23
> I do all this for the sake of the gospel, that I may share in its blessings.
>
> 凡我所行的,都是为福音的缘故,为要与人同得这福音的好处。


##### 林前9:24
> Do you not know that in a race all the runners run, but only one gets the prize? Run in such a way as to get the prize.
>
> 岂不知在场上赛跑的都跑,但得奖赏的只有一人?你们也当这样跑,好叫你们得着奖赏。


##### 林前9:25
> Everyone who competes in the games goes into strict training. They do it to get a crown that will not last; but we do it to get a crown that will last forever.
>
> 凡较力争胜的,诸事都有节制,他们不过是要得能坏的冠冕;我们却是要得不能坏的冠冕。


##### 林前9:26
> Therefore I do not run like a man running aimlessly; I do not fight like a man beating the air.
>
> 所以,我奔跑,不像无定向的;我斗拳,不像打空气的。


##### 林前9:27
> No, I beat my body and make it my slave so that after I have preached to others, I myself will not be disqualified for the prize.
>
> 我是攻克己身,叫身服我,恐怕我传福音给别人,自己反被弃绝了。


## 哥林多前书第10章
##### 林前10:1
> For I do not want you to be ignorant of the fact, brothers, that our forefathers were all under the cloud and that they all passed through the sea.
>
> 弟兄们,我不愿意你们不晓得,我们的祖宗从前都在云下,都从海中经过,


##### 林前10:2
> They were all baptized into Moses in the cloud and in the sea.
>
> 都在云里、海里受洗归了摩西,


##### 林前10:3
> They all ate the same spiritual food
>
> 并且都吃了一样的灵食,


##### 林前10:4
> and drank the same spiritual drink; for they drank from the spiritual rock that accompanied them, and that rock was Christ.
>
> 也都喝了一样的灵水;所喝的,是出于随着他们的灵磐石,那磐石就是基督。


##### 林前10:5
> Nevertheless, God was not pleased with most of them; their bodies were scattered over the desert.
>
> 但他们中间多半是 神不喜欢的人,所以在旷野倒毙。


##### 林前10:6
> Now these things occurred as examples to keep us from setting our hearts on evil things as they did.
>
> 这些事都是我们的监戒,叫我们不要贪恋恶事,像他们那样贪恋的;


##### 林前10:7
> Do not be idolaters, as some of them were; as it is written: "The people sat down to eat and drink and got up to indulge in pagan revelry."
>
> 也不要拜偶像,像他们有人拜的。如经上所记:“百姓坐下吃喝,起来玩耍。”


##### 林前10:8
> We should not commit sexual immorality, as some of them did--and in one day twenty-three thousand of them died.
>
> 我们也不要行奸淫,像他们有人行的,一天就倒毙了二万三千人。


##### 林前10:9
> We should not test the Lord, as some of them did--and were killed by snakes.
>
> 也不要试探主(“主”有古卷作“基督”),像他们有人试探的,就被蛇所灭;


##### 林前10:10
> And do not grumble, as some of them did--and were killed by the destroying angel.
>
> 你们也不要发怨言,像他们有发怨言的,就被灭命的所灭。


##### 林前10:11
> These things happened to them as examples and were written down as warnings for us, on whom the fulfillment of the ages has come.
>
> 他们遭遇这些事都要作为监戒,并且写在经上,正是警戒我们这末世的人。


##### 林前10:12
> So, if you think you are standing firm, be careful that you don't fall!
>
> 所以,自己以为站得稳的,须要谨慎,免得跌倒。


##### 林前10:13
> No temptation has seized you except what is common to man. And God is faithful; he will not let you be tempted beyond what you can bear. But when you are tempted, he will also provide a way out so that you can stand up under it.
>
> 你们所遇见的试探,无非是人所能受的。 神是信实的,必不叫你们受试探过于所能受的。在受试探的时候,总要给你们开一条出路,叫你们能忍受得住。


##### 林前10:14
> Therefore, my dear friends, flee from idolatry.
>
> 我所亲爱的弟兄啊,你们要逃避拜偶像的事。


##### 林前10:15
> I speak to sensible people; judge for yourselves what I say.
>
> 我好像对明白人说的,你们要审察我的话。


##### 林前10:16
> Is not the cup of thanksgiving for which we give thanks a participation in the blood of Christ? And is not the bread that we break a participation in the body of Christ?
>
> 我们所祝福的杯,岂不是同领基督的血吗?我们所擘开的饼,岂不是同领基督的身体吗?


##### 林前10:17
> Because there is one loaf, we, who are many, are one body, for we all partake of the one loaf.
>
> 我们虽多,仍是一个饼、一个身体,因为我们都是分受这一个饼。


##### 林前10:18
> Consider the people of Israel: Do not those who eat the sacrifices participate in the altar?
>
> 你们看属肉体的以色列人,那吃祭物的岂不是在祭坛上有分吗?


##### 林前10:19
> Do I mean then that a sacrifice offered to an idol is anything, or that an idol is anything?
>
> 我是怎么说呢?岂是说祭偶像之物算得什么呢?或说偶像算得什么呢?


##### 林前10:20
> No, but the sacrifices of pagans are offered to demons, not to God, and I do not want you to be participants with demons.
>
> 我乃是说:外邦人所献的祭是祭鬼,不是祭 神,我不愿意你们与鬼相交。


##### 林前10:21
> You cannot drink the cup of the Lord and the cup of demons too; you cannot have a part in both the Lord's table and the table of demons.
>
> 你们不能喝主的杯,又喝鬼的杯;不能吃主的筵席又吃鬼的筵席。


##### 林前10:22
> Are we trying to arouse the Lord's jealousy? Are we stronger than he?
>
> 我们可惹主的愤恨吗?我们比他还有能力吗?


##### 林前10:23
> "Everything is permissible"--but not everything is beneficial. "Everything is permissible"--but not everything is constructive.
>
> 凡事都可行,但不都有益处。凡事都可行,但不都造就人。


##### 林前10:24
> Nobody should seek his own good, but the good of others.
>
> 无论何人,不要求自己的益处,乃要求别人的益处。


##### 林前10:25
> Eat anything sold in the meat market without raising questions of conscience,
>
> 凡市上所卖的,你们只管吃,不要为良心的缘故问什么话,


##### 林前10:26
> for, "The earth is the Lord's, and everything in it."
>
> 因为地和其中所充满的都属乎主。


##### 林前10:27
> If some unbeliever invites you to a meal and you want to go, eat whatever is put before you without raising questions of conscience.
>
> 倘有一个不信的人请你们赴席,你们若愿意去,凡摆在你们面前的,只管吃,不要为良心的缘故问什么话。


##### 林前10:28
> But if anyone says to you, "This has been offered in sacrifice," then do not eat it, both for the sake of the man who told you and for conscience' sake--
>
> 若有人对你们说“这是献过祭的物”,就要为那告诉你们的人,并为良心的缘故不吃。


##### 林前10:29
> the other man's conscience, I mean, not yours. For why should my freedom be judged by another's conscience?
>
> 我说的良心不是你的,乃是他的。我这自由为什么被别人的良心论断呢?


##### 林前10:30
> If I take part in the meal with thankfulness, why am I denounced because of something I thank God for?
>
> 我若谢恩而吃,为什么因我谢恩的物被人毁谤呢?


##### 林前10:31
> So whether you eat or drink or whatever you do, do it all for the glory of God.
>
> 所以,你们或吃或喝,无论做什么,都要为荣耀 神而行。


##### 林前10:32
> Do not cause anyone to stumble, whether Jews, Greeks or the church of God--
>
> 不拘是犹太人,是希利尼人,是 神的教会,你们都不要使他跌倒;


##### 林前10:33
> even as I try to please everybody in every way. For I am not seeking my own good but the good of many, so that they may be saved.
>
> 就好像我凡事都叫众人喜欢,不求自己的益处,只求众人的益处,叫他们得救。


## 哥林多前书第11章
##### 林前11:1
> Follow my example, as I follow the example of Christ.
>
> 你们该效法我,像我效法基督一样。


##### 林前11:2
> I praise you for remembering me in everything and for holding to the teachings, just as I passed them on to you.
>
> 我称赞你们,因你们凡事记念我,又坚守我所传给你们的。


##### 林前11:3
> Now I want you to realize that the head of every man is Christ, and the head of the woman is man, and the head of Christ is God.
>
> 我愿意你们知道,基督是各人的头,男人是女人的头, 神是基督的头。


##### 林前11:4
> Every man who prays or prophesies with his head covered dishonors his head.
>
> 凡男人祷告或是讲道(“讲道”或作“说预言”。下同),若蒙着头,就羞辱自己的头。


##### 林前11:5
> And every woman who prays or prophesies with her head uncovered dishonors her head--it is just as though her head were shaved.
>
> 凡女人祷告或是讲道,若不蒙着头,就羞辱自己的头,因为这就如同剃了头发一样。


##### 林前11:6
> If a woman does not cover her head, she should have her hair cut off; and if it is a disgrace for a woman to have her hair cut or shaved off, she should cover her head.
>
> 女人若不蒙着头,就该剪了头发;女人若以剪发剃发为羞愧,就该蒙着头。


##### 林前11:7
> A man ought not to cover his head, since he is the image and glory of God; but the woman is the glory of man.
>
> 男人本不该蒙着头,因为他是 神的形像和荣耀,但女人是男人的荣耀。


##### 林前11:8
> For man did not come from woman, but woman from man;
>
> 起初,男人不是由女人而出,女人乃是由男人而出。


##### 林前11:9
> neither was man created for woman, but woman for man.
>
> 并且男人不是为女人造的,女人乃是为男人造的。


##### 林前11:10
> For this reason, and because of the angels, the woman ought to have a sign of authority on her head.
>
> 因此,女人为天使的缘故,应当在头上有服权柄的记号。


##### 林前11:11
> In the Lord, however, woman is not independent of man, nor is man independent of woman.
>
> 然而照主的安排,女也不是无男,男也不是无女。


##### 林前11:12
> For as woman came from man, so also man is born of woman. But everything comes from God.
>
> 因为女人原是由男人而出,男人也是由女人而出;但万有都是出乎 神。


##### 林前11:13
> Judge for yourselves: Is it proper for a woman to pray to God with her head uncovered?
>
> 你们自己审察,女人祷告 神,不蒙着头是合宜的吗?


##### 林前11:14
> Does not the very nature of things teach you that if a man has long hair, it is a disgrace to him,
>
> 你们的本性不也指示你们,男人若有长头发,便是他的羞辱吗?


##### 林前11:15
> but that if a woman has long hair, it is her glory? For long hair is given to her as a covering.
>
> 但女人有长头发,乃是她的荣耀,因为这头发是给她作盖头的。


##### 林前11:16
> If anyone wants to be contentious about this, we have no other practice--nor do the churches of God.
>
> 若有人想要辩驳,我们却没有这样的规矩, 神的众教会也是没有的。


##### 林前11:17
> In the following directives I have no praise for you, for your meetings do more harm than good.
>
> 我现今吩咐你们的话,不是称赞你们,因为你们聚会不是受益,乃是招损。


##### 林前11:18
> In the first place, I hear that when you come together as a church, there are divisions among you, and to some extent I believe it.
>
> 第一,我听说你们聚会的时候,彼此分门别类,我也稍微地信这话。


##### 林前11:19
> No doubt there have to be differences among you to show which of you have God's approval.
>
> 在你们中间不免有分门结党的事,好叫那些有经验的人显明出来。


##### 林前11:20
> When you come together, it is not the Lord's Supper you eat,
>
> 你们聚会的时候,算不得吃主的晚餐;


##### 林前11:21
> for as you eat, each of you goes ahead without waiting for anybody else. One remains hungry, another gets drunk.
>
> 因为吃的时候,各人先吃自己的饭,甚至这个饥饿,那个酒醉。


##### 林前11:22
> Don't you have homes to eat and drink in? Or do you despise the church of God and humiliate those who have nothing? What shall I say to you? Shall I praise you for this? Certainly not!
>
> 你们要吃喝,难道没有家吗?还是藐视 神的教会,叫那没有的羞愧呢?我向你们可怎么说呢?可因此称赞你们吗?我不称赞!


##### 林前11:23
> For I received from the Lord what I also passed on to you: The Lord Jesus, on the night he was betrayed, took bread,
>
> 我当日传给你们的,原是从主领受的,就是主耶稣被卖的那一夜,拿起饼来,


##### 林前11:24
> and when he had given thanks, he broke it and said, "This is my body, which is for you; do this in remembrance of me."
>
> 祝谢了,就掰开,说,这是我的身体,为你们舍的(舍有古卷作掰开)。你们应当如此行,为的是记念我。


##### 林前11:25
> In the same way, after supper he took the cup, saying, "This cup is the new covenant in my blood; do this, whenever you drink it, in remembrance of me."
>
> 饭后,也照样拿起杯来,说:“这杯是用我的血所立的新约。你们每逢喝的时候,要如此行,为的是记念我。”


##### 林前11:26
> For whenever you eat this bread and drink this cup, you proclaim the Lord's death until he comes.
>
> 你们每逢吃这饼,喝这杯,是表明主的死,直等到他来。


##### 林前11:27
> Therefore, whoever eats the bread or drinks the cup of the Lord in an unworthy manner will be guilty of sinning against the body and blood of the Lord.
>
> 所以,无论何人不按理吃主的饼、喝主的杯,就是干犯主的身、主的血了。


##### 林前11:28
> A man ought to examine himself before he eats of the bread and drinks of the cup.
>
> 人应当自己省察,然后吃这饼、喝这杯。


##### 林前11:29
> For anyone who eats and drinks without recognizing the body of the Lord eats and drinks judgment on himself.
>
> 因为人吃喝,若不分辨是主的身体,就是吃喝自己的罪了。


##### 林前11:30
> That is why many among you are weak and sick, and a number of you have fallen asleep.
>
> 因此,在你们中间有好些软弱的与患病的,死的也不少(“死”原文作“睡”)。


##### 林前11:31
> But if we judged ourselves, we would not come under judgment.
>
> 我们若是先分辨自己,就不至于受审。


##### 林前11:32
> When we are judged by the Lord, we are being disciplined so that we will not be condemned with the world.
>
> 我们受审的时候,乃是被主惩治,免得我们和世人一同定罪。


##### 林前11:33
> So then, my brothers, when you come together to eat, wait for each other.
>
> 所以我弟兄们,你们聚会吃的时候,要彼此等待。


##### 林前11:34
> If anyone is hungry, he should eat at home, so that when you meet together it may not result in judgment. And when I come I will give further directions.
>
> 若有人饥饿,可以在家里先吃,免得你们聚会,自己取罪。其余的事,我来的时候再安排。


## 哥林多前书第12章
##### 林前12:1
> Now about spiritual gifts, brothers, I do not want you to be ignorant.
>
> 弟兄们,论到属灵的恩赐,我不愿意你们不明白。


##### 林前12:2
> You know that when you were pagans, somehow or other you were influenced and led astray to mute idols.
>
> 你们作外邦人的时候,随事被牵引、受迷惑,去服事那哑巴偶像,这是你们知道的。


##### 林前12:3
> Therefore I tell you that no one who is speaking by the Spirit of God says, "Jesus be cursed," and no one can say, "Jesus is Lord," except by the Holy Spirit.
>
> 所以我告诉你们:被 神的灵感动的,没有说耶稣是可咒诅的;若不是被圣灵感动的,也没有能说耶稣是主的。


##### 林前12:4
> There are different kinds of gifts, but the same Spirit.
>
> 恩赐原有分别,圣灵却是一位。


##### 林前12:5
> There are different kinds of service, but the same Lord.
>
> 职事也有分别,主却是一位。


##### 林前12:6
> There are different kinds of working, but the same God works all of them in all men.
>
> 功用也有分别, 神却是一位,在众人里面运行一切的事。


##### 林前12:7
> Now to each one the manifestation of the Spirit is given for the common good.
>
> 圣灵显在各人身上,是叫人得益处。


##### 林前12:8
> To one there is given through the Spirit the message of wisdom, to another the message of knowledge by means of the same Spirit,
>
> 这人蒙圣灵赐他智慧的言语,那人也蒙这位圣灵赐他知识的言语,


##### 林前12:9
> to another faith by the same Spirit, to another gifts of healing by that one Spirit,
>
> 又有一人蒙这位圣灵赐他信心,还有一人蒙这位圣灵赐他医病的恩赐,


##### 林前12:10
> to another miraculous powers, to another prophecy, to another distinguishing between spirits, to another speaking in different kinds of tongues, and to still another the interpretation of tongues.
>
> 又叫一人能行异能,又叫一人能作先知,又叫一人能辨别诸灵,又叫一人能说方言,又叫一人能翻方言。


##### 林前12:11
> All these are the work of one and the same Spirit, and he gives them to each one, just as he determines.
>
> 这一切都是这位圣灵所运行、随己意分给各人的。


##### 林前12:12
> The body is a unit, though it is made up of many parts; and though all its parts are many, they form one body. So it is with Christ.
>
> 就如身子是一个,却有许多肢体;而且肢体虽多,仍是一个身子。基督也是这样。


##### 林前12:13
> For we were all baptized by one Spirit into one body--whether Jews or Greeks, slave or free--and we were all given the one Spirit to drink.
>
> 我们不拘是犹太人,是希利尼人,是为奴的,是自主的,都从一位圣灵受洗,成了一个身体,饮于一位圣灵。


##### 林前12:14
> Now the body is not made up of one part but of many.
>
> 身子原不是一个肢体,乃是许多肢体。


##### 林前12:15
> If the foot should say, "Because I am not a hand, I do not belong to the body," it would not for that reason cease to be part of the body.
>
> 设若脚说:“我不是手,所以不属乎身子。”它不能因此就不属乎身子。


##### 林前12:16
> And if the ear should say, "Because I am not an eye, I do not belong to the body," it would not for that reason cease to be part of the body.
>
> 设若耳说:“我不是眼,所以不属乎身子。”它也不能因此就不属乎身子。


##### 林前12:17
> If the whole body were an eye, where would the sense of hearing be? If the whole body were an ear, where would the sense of smell be?
>
> 若全身是眼,从哪里听声呢?若全身是耳,从哪里闻味呢?


##### 林前12:18
> But in fact God has arranged the parts in the body, every one of them, just as he wanted them to be.
>
> 但如今 神随自己的意思把肢体俱各安排在身上了。


##### 林前12:19
> If they were all one part, where would the body be?
>
> 若都是一个肢体,身子在哪里呢?


##### 林前12:20
> As it is, there are many parts, but one body.
>
> 但如今肢体是多的,身子却是一个。


##### 林前12:21
> The eye cannot say to the hand, "I don't need you!" And the head cannot say to the feet, "I don't need you!"
>
> 眼不能对手说:“我用不着你。”头也不能对脚说:“我用不着你。”


##### 林前12:22
> On the contrary, those parts of the body that seem to be weaker are indispensable,
>
> 不但如此,身上肢体,人以为软弱的,更是不可少的。


##### 林前12:23
> and the parts that we think are less honorable we treat with special honor. And the parts that are unpresentable are treated with special modesty,
>
> 身上肢体,我们看为不体面的,越发给它加上体面;不俊美的,越发得着俊美。


##### 林前12:24
> while our presentable parts need no special treatment. But God has combined the members of the body and has given greater honor to the parts that lacked it,
>
> 我们俊美的肢体,自然用不着装饰;但 神配搭这身子,把加倍的体面给那有缺欠的肢体,


##### 林前12:25
> so that there should be no division in the body, but that its parts should have equal concern for each other.
>
> 免得身上分门别类,总要肢体彼此相顾。


##### 林前12:26
> If one part suffers, every part suffers with it; if one part is honored, every part rejoices with it.
>
> 若一个肢体受苦,所有的肢体就一同受苦;若一个肢体得荣耀,所有的肢体就一同快乐。


##### 林前12:27
> Now you are the body of Christ, and each one of you is a part of it.
>
> 你们就是基督的身子,并且各自作肢体。


##### 林前12:28
> And in the church God has appointed first of all apostles, second prophets, third teachers, then workers of miracles, also those having gifts of healing, those able to help others, those with gifts of administration, and those speaking in different kinds of tongues.
>
> 神在教会所设立的:第一是使徒,第二是先知,第三是教师,其次是行异能的,再次是得恩赐医病的,帮助人的,治理事的,说方言的。


##### 林前12:29
> Are all apostles? Are all prophets? Are all teachers? Do all work miracles?
>
> 岂都是使徒吗?岂都是先知吗?岂都是教师吗?岂都是行异能的吗?


##### 林前12:30
> Do all have gifts of healing? Do all speak in tongues? Do all interpret?
>
> 岂都是得恩赐医病的吗?岂都是说方言的吗?岂都是翻方言的吗?


##### 林前12:31
> But eagerly desire the greater gifts. And now I will show you the most excellent way.
>
> 你们要切切的求那更大的恩赐。我现今把最妙的道指示你们。


## 哥林多前书第13章
##### 林前13:1
> If I speak in the tongues of men and of angels, but have not love, I am only a resounding gong or a clanging cymbal.
>
> 我若能说万人的方言,并天使的话语,却没有爱,我就成了鸣的锣、响的钹一般。


##### 林前13:2
> If I have the gift of prophecy and can fathom all mysteries and all knowledge, and if I have a faith that can move mountains, but have not love, I am nothing.
>
> 我若有先知讲道之能,也明白各样的奥秘、各样的知识,而且有全备的信,叫我能够移山,却没有爱,我就算不得什么。


##### 林前13:3
> If I give all I possess to the poor and surrender my body to the flames, but have not love, I gain nothing.
>
> 我若将所有的周济穷人,又舍己身叫人焚烧,却没有爱,仍然与我无益。


##### 林前13:4
> Love is patient, love is kind. It does not envy, it does not boast, it is not proud.
>
> 爱是恒久忍耐,又有恩慈;爱是不嫉妒,爱是不自夸,不张狂,


##### 林前13:5
> It is not rude, it is not self-seeking, it is not easily angered, it keeps no record of wrongs.
>
> 不做害羞的事,不求自己的益处,不轻易发怒,不计算人的恶,


##### 林前13:6
> Love does not delight in evil but rejoices with the truth.
>
> 不喜欢不义,只喜欢真理;


##### 林前13:7
> It always protects, always trusts, always hopes, always perseveres.
>
> 凡事包容,凡事相信,凡事盼望,凡事忍耐;


##### 林前13:8
> Love never fails. But where there are prophecies, they will cease; where there are tongues, they will be stilled; where there is knowledge, it will pass away.
>
> 爱是永不止息。先知讲道之能,终必归于无有;说方言之能,终必停止;知识也终必归于无有。


##### 林前13:9
> For we know in part and we prophesy in part,
>
> 我们现在所知道的有限,先知所讲的也有限,


##### 林前13:10
> but when perfection comes, the imperfect disappears.
>
> 等那完全的来到,这有限的必归于无有了。


##### 林前13:11
> When I was a child, I talked like a child, I thought like a child, I reasoned like a child. When I became a man, I put childish ways behind me.
>
> 我作孩子的时候,话语像孩子,心思像孩子,意念像孩子;既成了人,就把孩子的事丢弃了。


##### 林前13:12
> Now we see but a poor reflection as in a mirror; then we shall see face to face. Now I know in part; then I shall know fully, even as I am fully known.
>
> 我们如今仿佛对着镜子观看,模糊不清(“模糊不清”原文作“如同猜谜”),到那时,就要面对面了。我如今所知道的有限,到那时就全知道,如同主知道我一样。


##### 林前13:13
> And now these three remain: faith, hope and love. But the greatest of these is love.
>
> 如今常存的有信,有望,有爱;这三样,其中最大的是爱。


## 哥林多前书第14章
##### 林前14:1
> Follow the way of love and eagerly desire spiritual gifts, especially the gift of prophecy.
>
> 你们要追求爱,也要切慕属灵的恩赐,其中更要羡慕的,是作先知讲道(原文作“是说预言”。下同)。


##### 林前14:2
> For anyone who speaks in a tongue does not speak to men but to God. Indeed, no one understands him; he utters mysteries with his spirit.
>
> 那说方言的,原不是对人说,乃是对 神说,因为没有人听出来。然而他在心灵里,却是讲说各样的奥秘。


##### 林前14:3
> But everyone who prophesies speaks to men for their strengthening, encouragement and comfort.
>
> 但作先知讲道的,是对人说,要造就、安慰、劝勉人。


##### 林前14:4
> He who speaks in a tongue edifies himself, but he who prophesies edifies the church.
>
> 说方言的,是造就自己;作先知讲道的,乃是造就教会。


##### 林前14:5
> I would like every one of you to speak in tongues, but I would rather have you prophesy. He who prophesies is greater than one who speaks in tongues, unless he interprets, so that the church may be edified.
>
> 我愿意你们都说方言,更愿意你们作先知讲道,因为说方言的,若不翻出来,使教会被造就,那作先知讲道的,就比他强了。


##### 林前14:6
> Now, brothers, if I come to you and speak in tongues, what good will I be to you, unless I bring you some revelation or knowledge or prophecy or word of instruction?
>
> 弟兄们,我到你们那里去,若只说方言,不用启示、或知识、或预言、或教训,给你们讲解,我与你们有什么益处呢?


##### 林前14:7
> Even in the case of lifeless things that make sounds, such as the flute or harp, how will anyone know what tune is being played unless there is a distinction in the notes?
>
> 就是那有声无气的物,或箫、或琴,若发出来的声音没有分别,怎能知道所吹、所弹的是什么呢?


##### 林前14:8
> Again, if the trumpet does not sound a clear call, who will get ready for battle?
>
> 若吹无定的号声,谁能预备打仗呢?


##### 林前14:9
> So it is with you. Unless you speak intelligible words with your tongue, how will anyone know what you are saying? You will just be speaking into the air.
>
> 你们也是如此,舌头若不说容易明白的话,怎能知道所说的是什么呢?这就是向空说话了。


##### 林前14:10
> Undoubtedly there are all sorts of languages in the world, yet none of them is without meaning.
>
> 世上的声音或者甚多,却没有一样是无意思的。


##### 林前14:11
> If then I do not grasp the meaning of what someone is saying, I am a foreigner to the speaker, and he is a foreigner to me.
>
> 我若不明白那声音的意思,这说话的人必以我为化外之人,我也以他为化外之人。


##### 林前14:12
> So it is with you. Since you are eager to have spiritual gifts, try to excel in gifts that build up the church.
>
> 你们也是如此,既是切慕属灵的恩赐,就当求多得造就教会的恩赐。


##### 林前14:13
> For this reason anyone who speaks in a tongue should pray that he may interpret what he says.
>
> 所以那说方言的,就当求着能翻出来。


##### 林前14:14
> For if I pray in a tongue, my spirit prays, but my mind is unfruitful.
>
> 我若用方言祷告,是我的灵祷告,但我的悟性没有果效。


##### 林前14:15
> So what shall I do? I will pray with my spirit, but I will also pray with my mind; I will sing with my spirit, but I will also sing with my mind.
>
> 这却怎么样呢?我要用灵祷告,也要用悟性祷告;我要用灵歌唱,也要用悟性歌唱。


##### 林前14:16
> If you are praising God with your spirit, how can one who finds himself among those who do not understand say "Amen" to your thanksgiving, since he does not know what you are saying?
>
> 不然,你用灵祝谢,那在座不通方言的人,既然不明白你的话,怎能在你感谢的时候说“阿们”呢?


##### 林前14:17
> You may be giving thanks well enough, but the other man is not edified.
>
> 你感谢的固然是好,无奈不能造就别人。


##### 林前14:18
> I thank God that I speak in tongues more than all of you.
>
> 我感谢 神,我说方言比你们众人还多。


##### 林前14:19
> But in the church I would rather speak five intelligible words to instruct others than ten thousand words in a tongue.
>
> 但在教会中,宁可用悟性说五句教导人的话,强如说万句方言。


##### 林前14:20
> Brothers, stop thinking like children. In regard to evil be infants, but in your thinking be adults.
>
> 弟兄们,在心志上不要作小孩子;然而,在恶事上要作婴孩,在心志上总要作大人。


##### 林前14:21
> In the Law it is written: "Through men of strange tongues and through the lips of foreigners I will speak to this people, but even then they will not listen to me," says the Lord.
>
> 律法上记着:“主说:‘我要用外邦人的舌头和外邦人的嘴唇,向这百姓说话,虽然如此,他们还是不听从我。’”


##### 林前14:22
> Tongues, then, are a sign, not for believers but for unbelievers; prophecy, however, is for believers, not for unbelievers.
>
> 这样看来,说方言不是为信的人作证据,乃是为不信的人;作先知讲道不是为不信的人作证据,乃是为信的人。


##### 林前14:23
> So if the whole church comes together and everyone speaks in tongues, and some who do not understand or some unbelievers come in, will they not say that you are out of your mind?
>
> 所以全教会聚在一处的时候,若都说方言,偶然有不通方言的,或是不信的人进来,岂不说你们癫狂了吗?


##### 林前14:24
> But if an unbeliever or someone who does not understand comes in while everybody is prophesying, he will be convinced by all that he is a sinner and will be judged by all,
>
> 若都作先知讲道,偶然有不信的,或是不通方言的人进来,就被众人劝醒,被众人审明,


##### 林前14:25
> and the secrets of his heart will be laid bare. So he will fall down and worship God, exclaiming, "God is really among you!"
>
> 他心里的隐情显露出来,就必将脸伏地,敬拜 神,说:“ 神真是在你们中间了。”


##### 林前14:26
> What then shall we say, brothers? When you come together, everyone has a hymn, or a word of instruction, a revelation, a tongue or an interpretation. All of these must be done for the strengthening of the church.
>
> 弟兄们,这却怎么样呢?你们聚会的时候,各人或有诗歌,或有教训,或有启示,或有方言,或有翻出来的话,凡事都当造就人。


##### 林前14:27
> If anyone speaks in a tongue, two--or at the most three--should speak, one at a time, and someone must interpret.
>
> 若有说方言的,只好两个人,至多三个人,且要轮流着说,也要一个人翻出来。


##### 林前14:28
> If there is no interpreter, the speaker should keep quiet in the church and speak to himself and God.
>
> 若没有人翻,就当在会中闭口,只对自己和 神说就是了。


##### 林前14:29
> Two or three prophets should speak, and the others should weigh carefully what is said.
>
> 至于作先知讲道的,只好两个人,或是三个人,其余的就当慎思明辨。


##### 林前14:30
> And if a revelation comes to someone who is sitting down, the first speaker should stop.
>
> 若旁边坐着的得了启示,那先说话的就当闭口不言。


##### 林前14:31
> For you can all prophesy in turn so that everyone may be instructed and encouraged.
>
> 因为你们都可以一个一个地作先知讲道,叫众人学道理,叫众人得劝勉。


##### 林前14:32
> The spirits of prophets are subject to the control of prophets.
>
> 先知的灵原是顺服先知的,


##### 林前14:33
> For God is not a God of disorder but of peace. As in all the congregations of the saints,
>
> 因为 神不是叫人混乱,乃是叫人安静。


##### 林前14:34
> women should remain silent in the churches. They are not allowed to speak, but must be in submission, as the Law says.
>
> 妇女在会中要闭口不言,像在圣徒的众教会一样,因为不准她们说话。她们总要顺服,正如律法所说的。


##### 林前14:35
> If they want to inquire about something, they should ask their own husbands at home; for it is disgraceful for a woman to speak in the church.
>
> 她们若要学什么,可以在家里问自己的丈夫,因为妇女在会中说话原是可耻的。


##### 林前14:36
> Did the word of God originate with you? Or are you the only people it has reached?
>
> 神的道理岂是从你们出来吗?岂是单临到你们吗?


##### 林前14:37
> If anybody thinks he is a prophet or spiritually gifted, let him acknowledge that what I am writing to you is the Lord's command.
>
> 若有人以为自己是先知,或是属灵的,就该知道我所写给你们的是主的命令。


##### 林前14:38
> If he ignores this, he himself will be ignored.
>
> 若有不知道的,就由他不知道吧!


##### 林前14:39
> Therefore, my brothers, be eager to prophesy, and do not forbid speaking in tongues.
>
> 所以我弟兄们,你们要切慕作先知讲道,也不要禁止说方言。


##### 林前14:40
> But everything should be done in a fitting and orderly way.
>
> 凡事都要规规矩矩地按着次序行。


## 哥林多前书第15章
##### 林前15:1
> Now, brothers, I want to remind you of the gospel I preached to you, which you received and on which you have taken your stand.
>
> 弟兄们,我如今把先前所传给你们的福音,告诉你们知道。这福音你们也领受了,又靠着站立得住;


##### 林前15:2
> By this gospel you are saved, if you hold firmly to the word I preached to you. Otherwise, you have believed in vain.
>
> 并且你们若不是徒然相信,能以持守我所传给你们的,就必因这福音得救。


##### 林前15:3
> For what I received I passed on to you as of first importance: that Christ died for our sins according to the Scriptures,
>
> 我当日所领受又传给你们的,第一,就是基督照圣经所说,为我们的罪死了,


##### 林前15:4
> that he was buried, that he was raised on the third day according to the Scriptures,
>
> 而且埋葬了,又照圣经所说,第三天复活了,


##### 林前15:5
> and that he appeared to Peter, and then to the Twelve.
>
> 并且显给矶法看,然后显给十二使徒看,


##### 林前15:6
> After that, he appeared to more than five hundred of the brothers at the same time, most of whom are still living, though some have fallen asleep.
>
> 后来一时显给五百多弟兄看,其中一大半到如今还在,却也有已经睡了的。


##### 林前15:7
> Then he appeared to James, then to all the apostles,
>
> 以后显给雅各看,再显给众使徒看,


##### 林前15:8
> and last of all he appeared to me also, as to one abnormally born.
>
> 末了,也显给我看;我如同未到产期而生的人一般。


##### 林前15:9
> For I am the least of the apostles and do not even deserve to be called an apostle, because I persecuted the church of God.
>
> 我原是使徒中最小的,不配称为使徒,因为我从前逼迫 神的教会。


##### 林前15:10
> But by the grace of God I am what I am, and his grace to me was not without effect. No, I worked harder than all of them--yet not I, but the grace of God that was with me.
>
> 然而我今日成了何等人,是蒙 神的恩才成的;并且他所赐我的恩不是徒然的。我比众使徒格外劳苦,这原不是我,乃是 神的恩与我同在。


##### 林前15:11
> Whether, then, it was I or they, this is what we preach, and this is what you believed.
>
> 不拘是我,是众使徒,我们如此传,你们也如此信了。


##### 林前15:12
> But if it is preached that Christ has been raised from the dead, how can some of you say that there is no resurrection of the dead?
>
> 既传基督是从死里复活了,怎么在你们中间有人说没有死人复活的事呢?


##### 林前15:13
> If there is no resurrection of the dead, then not even Christ has been raised.
>
> 若没有死人复活的事,基督也就没有复活了。


##### 林前15:14
> And if Christ has not been raised, our preaching is useless and so is your faith.
>
> 若基督没有复活,我们所传的便是枉然,你们所信的也是枉然!


##### 林前15:15
> More than that, we are then found to be false witnesses about God, for we have testified about God that he raised Christ from the dead. But he did not raise him if in fact the dead are not raised.
>
> 并且明显我们是为 神妄作见证的,因我们见证 神是叫基督复活了。若死人真不复活, 神也就没有叫基督复活了。


##### 林前15:16
> For if the dead are not raised, then Christ has not been raised either.
>
> 因为死人若不复活,基督也就没有复活了。


##### 林前15:17
> And if Christ has not been raised, your faith is futile; you are still in your sins.
>
> 基督若没有复活,你们的信便是徒然,你们仍在罪里,


##### 林前15:18
> Then those also who have fallen asleep in Christ are lost.
>
> 就是在基督里睡了的人也灭亡了。


##### 林前15:19
> If only for this life we have hope in Christ, we are to be pitied more than all men.
>
> 我们若靠基督只在今生有指望,就算比众人更可怜。


##### 林前15:20
> But Christ has indeed been raised from the dead, the firstfruits of those who have fallen asleep.
>
> 但基督已经从死里复活,成为睡了之人初熟的果子。


##### 林前15:21
> For since death came through a man, the resurrection of the dead comes also through a man.
>
> 死既是因一人而来,死人复活也是因一人而来。


##### 林前15:22
> For as in Adam all die, so in Christ all will be made alive.
>
> 在亚当里众人都死了;照样,在基督里众人也都要复活。


##### 林前15:23
> But each in his own turn: Christ, the firstfruits; then, when he comes, those who belong to him.
>
> 但各人是按着自己的次序复活,初熟的果子是基督,以后在他来的时候,是那些属基督的。


##### 林前15:24
> Then the end will come, when he hands over the kingdom to God the Father after he has destroyed all dominion, authority and power.
>
> 再后,末期到了,那时,基督既将一切执政的、掌权的、有能的,都毁灭了,就把国交与父 神。


##### 林前15:25
> For he must reign until he has put all his enemies under his feet.
>
> 因为基督必要作王,等 神把一切仇敌都放在他的脚下。


##### 林前15:26
> The last enemy to be destroyed is death.
>
> 尽末了所毁灭的仇敌就是死。


##### 林前15:27
> For he "has put everything under his feet." Now when it says that "everything" has been put under him, it is clear that this does not include God himself, who put everything under Christ.
>
> 因为经上说:“ 神叫万物都服在他的脚下。”既说万物都服了他,明显那叫万物服他的,不在其内了。


##### 林前15:28
> When he has done this, then the Son himself will be made subject to him who put everything under him, so that God may be all in all.
>
> 万物既服了他,那时,子也要自己服那叫万物服他的,叫 神在万物之上,为万物之主。


##### 林前15:29
> Now if there is no resurrection, what will those do who are baptized for the dead? If the dead are not raised at all, why are people baptized for them?
>
> 不然,那些为死人受洗的,将来怎样呢?若死人总不复活,因何为他们受洗呢?


##### 林前15:30
> And as for us, why do we endanger ourselves every hour?
>
> 我们又因何时刻冒险呢?


##### 林前15:31
> I die every day--I mean that, brothers--just as surely as I glory over you in Christ Jesus our Lord.
>
> 弟兄们,我在我主基督耶稣里指着你们所夸的口极力地说,我是天天冒死。


##### 林前15:32
> If I fought wild beasts in Ephesus for merely human reasons, what have I gained? If the dead are not raised, "Let us eat and drink, for tomorrow we die."
>
> 我若当日像寻常人在以弗所同野兽战斗,那于我有什么益处呢?若死人不复活,我们就吃吃喝喝吧!因为明天要死了。


##### 林前15:33
> Do not be misled: "Bad company corrupts good character."
>
> 你们不要自欺,滥交是败坏善行。


##### 林前15:34
> Come back to your senses as you ought, and stop sinning; for there are some who are ignorant of God--I say this to your shame.
>
> 你们要醒悟为善,不要犯罪,因为有人不认识 神。我说这话是要叫你们羞愧。


##### 林前15:35
> But someone may ask, "How are the dead raised? With what kind of body will they come?"
>
> 或有人问:“死人怎样复活,带着什么身体来呢?”


##### 林前15:36
> How foolish! What you sow does not come to life unless it dies.
>
> 无知的人哪,你所种的,若不死,就不能生!


##### 林前15:37
> When you sow, you do not plant the body that will be, but just a seed, perhaps of wheat or of something else.
>
> 并且你所种的,不是那将来的形体,不过是子粒,即如麦子,或是别样的谷。


##### 林前15:38
> But God gives it a body as he has determined, and to each kind of seed he gives its own body.
>
> 但 神随自己的意思给它一个形体,并叫各等子粒各有自己的形体。


##### 林前15:39
> All flesh is not the same: Men have one kind of flesh, animals have another, birds another and fish another.
>
> 凡肉体各有不同:人是一样,兽又是一样,鸟又是一样,鱼又是一样。


##### 林前15:40
> There are also heavenly bodies and there are earthly bodies; but the splendor of the heavenly bodies is one kind, and the splendor of the earthly bodies is another.
>
> 有天上的形体,也有地上的形体。但天上形体的荣光是一样,地上形体的荣光又是一样。


##### 林前15:41
> The sun has one kind of splendor, the moon another and the stars another; and star differs from star in splendor.
>
> 日有日的荣光,月有月的荣光,星有星的荣光;这星和那星的荣光也有分别。


##### 林前15:42
> So will it be with the resurrection of the dead. The body that is sown is perishable, it is raised imperishable;
>
> 死人复活也是这样:所种的是必朽坏的,复活的是不朽坏的;


##### 林前15:43
> it is sown in dishonor, it is raised in glory; it is sown in weakness, it is raised in power;
>
> 所种的是羞辱的,复活的是荣耀的;所种的是软弱的,复活的是强壮的;


##### 林前15:44
> it is sown a natural body, it is raised a spiritual body. If there is a natural body, there is also a spiritual body.
>
> 所种的是血气的身体,复活的是灵性的身体。若有血气的身体,也必有灵性的身体。


##### 林前15:45
> So it is written: "The first man Adam became a living being"; the last Adam, a life-giving spirit.
>
> 经上也是这样记着说,“首先的人亚当成了有灵的活人(“灵”或作“血气”)”;末后的亚当成了叫人活的灵。


##### 林前15:46
> The spiritual did not come first, but the natural, and after that the spiritual.
>
> 但属灵的不在先,属血气的在先,以后才有属灵的。


##### 林前15:47
> The first man was of the dust of the earth, the second man from heaven.
>
> 头一个人是出于地,乃属土;第二个人是出于天。


##### 林前15:48
> As was the earthly man, so are those who are of the earth; and as is the man from heaven, so also are those who are of heaven.
>
> 那属土的怎样,凡属土的也就怎样;属天的怎样,凡属天的也就怎样。


##### 林前15:49
> And just as we have borne the likeness of the earthly man, so shall we bear the likeness of the man from heaven.
>
> 我们既有属土的形状,将来也必有属天的形状。


##### 林前15:50
> I declare to you, brothers, that flesh and blood cannot inherit the kingdom of God, nor does the perishable inherit the imperishable.
>
> 弟兄们,我告诉你们说,血肉之体不能承受 神的国,必朽坏的不能承受不朽坏的。


##### 林前15:51
> Listen, I tell you a mystery: We will not all sleep, but we will all be changed--
>
> 我如今把一件奥秘的事告诉你们:我们不是都要睡觉,乃是都要改变,


##### 林前15:52
> in a flash, in the twinkling of an eye, at the last trumpet. For the trumpet will sound, the dead will be raised imperishable, and we will be changed.
>
> 就在一霎时,眨眼之间,号筒末次吹响的时候。因号筒要响,死人要复活,成为不朽坏的,我们也要改变。


##### 林前15:53
> For the perishable must clothe itself with the imperishable, and the mortal with immortality.
>
> 这必朽坏的,总要变成不朽坏的。(变成原文作穿下同)这必死的,总要变成不死的。


##### 林前15:54
> When the perishable has been clothed with the imperishable, and the mortal with immortality, then the saying that is written will come true: "Death has been swallowed up in victory."
>
> 这必朽坏的既变成不朽坏的,这必死的既变成不死的,那时经上所记“死被得胜吞灭”的话就应验了。


##### 林前15:55
> "Where, O death, is your victory? Where, O death, is your sting?"
>
> “死啊,你得胜的权势在哪里?死啊,你的毒钩在哪里?”


##### 林前15:56
> The sting of death is sin, and the power of sin is the law.
>
> 死的毒钩就是罪,罪的权势就是律法。


##### 林前15:57
> But thanks be to God! He gives us the victory through our Lord Jesus Christ.
>
> 感谢 神,使我们藉着我们的主耶稣基督得胜。


##### 林前15:58
> Therefore, my dear brothers, stand firm. Let nothing move you. Always give yourselves fully to the work of the Lord, because you know that your labor in the Lord is not in vain.
>
> 所以,我亲爱的弟兄们,你们务要坚固,不可摇动,常常竭力多做主工,因为知道你们的劳苦,在主里面不是徒然的。


## 哥林多前书第16章
##### 林前16:1
> Now about the collection for God's people: Do what I told the Galatian churches to do.
>
> 论到为圣徒捐钱,我从前怎样吩咐加拉太的众教会,你们也当怎样行。


##### 林前16:2
> On the first day of every week, each one of you should set aside a sum of money in keeping with his income, saving it up, so that when I come no collections will have to be made.
>
> 每逢七日的第一日,各人要照自己的进项抽出来留着,免得我来的时候现凑。


##### 林前16:3
> Then, when I arrive, I will give letters of introduction to the men you approve and send them with your gift to Jerusalem.
>
> 及至我来到了,你们写信举荐谁,我就打发他们,把你们的捐资送到耶路撒冷去。


##### 林前16:4
> If it seems advisable for me to go also, they will accompany me.
>
> 若我也该去,他们可以和我同去。


##### 林前16:5
> After I go through Macedonia, I will come to you--for I will be going through Macedonia.
>
> 我要从马其顿经过,既经过了,就要到你们那里去,


##### 林前16:6
> Perhaps I will stay with you awhile, or even spend the winter, so that you can help me on my journey, wherever I go.
>
> 或者和你们同住几时,或者也过冬。无论我往哪里去,你们就可以给我送行。


##### 林前16:7
> I do not want to see you now and make only a passing visit; I hope to spend some time with you, if the Lord permits.
>
> 我如今不愿意路过见你们,主若许我,我就指望和你们同住几时。


##### 林前16:8
> But I will stay on at Ephesus until Pentecost,
>
> 但我要仍旧住在以弗所,直等到五旬节,


##### 林前16:9
> because a great door for effective work has opened to me, and there are many who oppose me.
>
> 因为有宽大又有功效的门为我开了,并且反对的人也多。


##### 林前16:10
> If Timothy comes, see to it that he has nothing to fear while he is with you, for he is carrying on the work of the Lord, just as I am.
>
> 若是提摩太来到,你们要留心,叫他在你们那里无所惧怕,因为他劳力做主的工,像我一样。


##### 林前16:11
> No one, then, should refuse to accept him. Send him on his way in peace so that he may return to me. I am expecting him along with the brothers.
>
> 所以无论谁,都不可藐视他,只要送他平安前行,叫他到我这里来,因我指望他和弟兄们同来。


##### 林前16:12
> Now about our brother Apollos: I strongly urged him to go to you with the brothers. He was quite unwilling to go now, but he will go when he has the opportunity.
>
> 至于兄弟亚波罗,我再三地劝他同弟兄们到你们那里去,但这时他决不愿意去,几时有了机会他必去。


##### 林前16:13
> Be on your guard; stand firm in the faith; be men of courage; be strong.
>
> 你们务要警醒,在真道上站立得稳,要作大丈夫,要刚强。


##### 林前16:14
> Do everything in love.
>
> 凡你们所做的,都要凭爱心而做。


##### 林前16:15
> You know that the household of Stephanas were the first converts in Achaia, and they have devoted themselves to the service of the saints. I urge you, brothers,
>
> 弟兄们,你们晓得司提反一家是亚该亚初结的果子,并且他们专以服侍圣徒为念。


##### 林前16:16
> to submit to such as these and to everyone who joins in the work, and labors at it.
>
> 我劝你们顺服这样的人,并一切同工同劳的人。


##### 林前16:17
> I was glad when Stephanas, Fortunatus and Achaicus arrived, because they have supplied what was lacking from you.
>
> 司提反和福徒拿都并亚该古到这里来,我很喜欢,因为你们待我有不及之处,他们补上了。


##### 林前16:18
> For they refreshed my spirit and yours also. Such men deserve recognition.
>
> 他们叫我和你们心里都快活。这样的人,你们务要敬重。


##### 林前16:19
> The churches in the province of Asia send you greetings. Aquila and Priscilla greet you warmly in the Lord, and so does the church that meets at their house.
>
> 亚西亚的众教会问你们安。亚居拉和百基拉并在他们家里的教会,因主多多地问你们安。


##### 林前16:20
> All the brothers here send you greetings. Greet one another with a holy kiss.
>
> 众弟兄都问你们安。你们要亲嘴问安,彼此务要圣洁。


##### 林前16:21
> I, Paul, write this greeting in my own hand.
>
> 我保罗亲笔问安。


##### 林前16:22
> If anyone does not love the Lord--a curse be on him. Come, O Lord!
>
> 若有人不爱主,这人可诅可咒。主必要来!


##### 林前16:23
> The grace of the Lord Jesus be with you.
>
> 愿主耶稣基督的恩常与你们众人同在!


##### 林前16:24
> My love to all of you in Christ Jesus. Amen.
>
> 我在基督耶稣里的爱与你们众人同在。阿们。

