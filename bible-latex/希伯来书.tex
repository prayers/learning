# 希伯来书
<!-- TOC -->

- [希伯来书](#希伯来书)
    - [希伯来书第1章](#希伯来书第1章)
    - [希伯来书第2章](#希伯来书第2章)
    - [希伯来书第3章](#希伯来书第3章)
    - [希伯来书第4章](#希伯来书第4章)
    - [希伯来书第5章](#希伯来书第5章)
    - [希伯来书第6章](#希伯来书第6章)
    - [希伯来书第7章](#希伯来书第7章)
    - [希伯来书第8章](#希伯来书第8章)
    - [希伯来书第9章](#希伯来书第9章)
    - [希伯来书第10章](#希伯来书第10章)
    - [希伯来书第11章](#希伯来书第11章)
    - [希伯来书第12章](#希伯来书第12章)
    - [希伯来书第13章](#希伯来书第13章)

<!-- /TOC -->
## 希伯来书第1章
##### 来1:1
> In the past God spoke to our forefathers through the prophets at many times and in various ways,
>
> 神既在古时藉着众先知多次多方地晓谕列祖,


##### 来1:2
> but in these last days he has spoken to us by his Son, whom he appointed heir of all things, and through whom he made the universe.
>
> 就在这末世,藉着他儿子晓谕我们,又早已立他为承受万有的,也曾藉着他创造诸世界。


##### 来1:3
> The Son is the radiance of God's glory and the exact representation of his being, sustaining all things by his powerful word. After he had provided purification for sins, he sat down at the right hand of the Majesty in heaven.
>
> 他是 神荣耀所发的光辉,是 神本体的真像,常用他权能的命令托住万有。他洗净了人的罪,就坐在高天至大者的右边。


##### 来1:4
> So he became as much superior to the angels as the name he has inherited is superior to theirs.
>
> 他所承受的名,既比天使的名更尊贵,就远超过天使。


##### 来1:5
> For to which of the angels did God ever say, "You are my Son; today I have become your Father"? Or again, "I will be his Father, and he will be my Son"?
>
> 所有的天使, 神从来对哪一个说:“你是我的儿子,我今日生你”?又指着哪一个说:“我要作他的父,他要作我的子”?


##### 来1:6
> And again, when God brings his firstborn into the world, he says, "Let all God's angels worship him."
>
> 再者, 神使长子到世上来的时候(或作“ 神再使长子到世上来的时候”),就说:“ 神的使者都要拜他。”


##### 来1:7
> In speaking of the angels he says, "He makes his angels winds, his servants flames of fire."
>
> 论到使者,又说:“ 神以风为使者,以火焰为仆役。”


##### 来1:8
> But about the Son he says, "Your throne, O God, will last for ever and ever, and righteousness will be the scepter of your kingdom.
>
> 论到子却说:“ 神啊,你的宝座是永永远远的,你的国权是正直的。


##### 来1:9
> You have loved righteousness and hated wickedness; therefore God, your God, has set you above your companions by anointing you with the oil of joy."
>
> 你喜爱公义,恨恶罪恶,所以 神,就是你的 神,用喜乐油膏你,胜过膏你的同伴。”


##### 来1:10
> He also says, "In the beginning, O Lord, you laid the foundations of the earth, and the heavens are the work of your hands.
>
> 又说:“主啊,你起初立了地的根基,天也是你手所造的。


##### 来1:11
> They will perish, but you remain; they will all wear out like a garment.
>
> 天地都要灭没,你却要长存;天地都要像衣服渐渐旧了,


##### 来1:12
> You will roll them up like a robe; like a garment they will be changed. But you remain the same, and your years will never end."
>
> 你要将天地卷起来,像一件外衣,天地就都改变了。惟有你永不改变,你的年数没有穷尽。”


##### 来1:13
> To which of the angels did God ever say, "Sit at my right hand until I make your enemies a footstool for your feet"?
>
> 所有的天使, 神从来对哪一个说:“你坐在我的右边,等我使你仇敌作你的脚凳”?


##### 来1:14
> Are not all angels ministering spirits sent to serve those who will inherit salvation?
>
> 天使岂不都是服役的灵、奉差遣为那将要承受救恩的人效力吗?


## 希伯来书第2章
##### 来2:1
> We must pay more careful attention, therefore, to what we have heard, so that we do not drift away.
>
> 所以,我们当越发郑重所听见的道理,恐怕我们随流失去。


##### 来2:2
> For if the message spoken by angels was binding, and every violation and disobedience received its just punishment,
>
> 那藉着天使所传的话,既是确定的,凡干犯悖逆的,都受了该受的报应。


##### 来2:3
> how shall we escape if we ignore such a great salvation? This salvation, which was first announced by the Lord, was confirmed to us by those who heard him.
>
> 我们若忽略这么大的救恩,怎能逃罪呢?这救恩起先是主亲自讲的,后来是听见的人给我们证实了。


##### 来2:4
> God also testified to it by signs, wonders and various miracles, and gifts of the Holy Spirit distributed according to his will.
>
> 神又按自己的旨意,用神迹奇事和百般的异能,并圣灵的恩赐,同他们作见证。


##### 来2:5
> It is not to angels that he has subjected the world to come, about which we are speaking.
>
> 我们所说将来的世界, 神原没有交给天使管辖。


##### 来2:6
> But there is a place where someone has testified: "What is man that you are mindful of him, the son of man that you care for him?
>
> 但有人在经上某处证明说:“人算什么,你竟顾念他?世人算什么,你竟眷顾他?


##### 来2:7
> You made him a little lower than the angels; you crowned him with glory and honor
>
> 你叫他比天使微小一点(或作“你叫他暂时比天使小”),赐他荣耀、尊贵为冠冕,并将你手所造的都派他管理,


##### 来2:8
> and put everything under his feet." In putting everything under him, God left nothing that is not subject to him. Yet at present we do not see everything subject to him.
>
> 叫万物都服在他的脚下。”既叫万物都服他,就没有剩下一样不服他的。只是如今我们还不见万物都服他,


##### 来2:9
> But we see Jesus, who was made a little lower than the angels, now crowned with glory and honor because he suffered death, so that by the grace of God he might taste death for everyone.
>
> 惟独见那成为比天使小一点的耶稣(或作“惟独见耶稣暂时比天使小”),因为受死的苦,就得了尊贵、荣耀为冠冕,叫他因着 神的恩,为人人尝了死味。


##### 来2:10
> In bringing many sons to glory, it was fitting that God, for whom and through whom everything exists, should make the author of their salvation perfect through suffering.
>
> 原来那为万物所属、为万物所本的,要领许多的儿子进荣耀里去,使救他们的元帅因受苦难得以完全,本是合宜的。


##### 来2:11
> Both the one who makes men holy and those who are made holy are of the same family. So Jesus is not ashamed to call them brothers.
>
> 因那使人成圣的和那些得以成圣的,都是出于一。所以他称他们为弟兄,也不以为耻,


##### 来2:12
> He says, "I will declare your name to my brothers; in the presence of the congregation I will sing your praises."
>
> 说:“我要将你的名传与我的弟兄,在会中我要颂扬你。”


##### 来2:13
> And again, "I will put my trust in him." And again he says, "Here am I, and the children God has given me."
>
> 又说:“我要倚赖他。”又说:“看哪,我与 神所给我的儿女。”


##### 来2:14
> Since the children have flesh and blood, he too shared in their humanity so that by his death he might destroy him who holds the power of death--that is, the devil--
>
> 儿女既同有血肉之体,他也照样亲自成了血肉之体,特要藉着死,败坏那掌死权的,就是魔鬼,


##### 来2:15
> and free those who all their lives were held in slavery by their fear of death.
>
> 并要释放那些一生因怕死而为奴仆的人。


##### 来2:16
> For surely it is not angels he helps, but Abraham's descendants.
>
> 他并不救拔天使,乃是救拔亚伯拉罕的后裔。


##### 来2:17
> For this reason he had to be made like his brothers in every way, in order that he might become a merciful and faithful high priest in service to God, and that he might make atonement for the sins of the people.
>
> 所以,他凡事该与他的弟兄相同,为要在 神的事上成为慈悲忠信的大祭司,为百姓的罪献上挽回祭。


##### 来2:18
> Because he himself suffered when he was tempted, he is able to help those who are being tempted.
>
> 他自己既然被试探而受苦,就能搭救被试探的人。


## 希伯来书第3章
##### 来3:1
> Therefore, holy brothers, who share in the heavenly calling, fix your thoughts on Jesus, the apostle and high priest whom we confess.
>
> 同蒙天召的圣洁弟兄啊,你们应当思想我们所认为使者、为大祭司的耶稣。


##### 来3:2
> He was faithful to the one who appointed him, just as Moses was faithful in all God's house.
>
> 他为那设立他的尽忠,如同摩西在 神的全家尽忠一样。


##### 来3:3
> Jesus has been found worthy of greater honor than Moses, just as the builder of a house has greater honor than the house itself.
>
> 他比摩西算是更配多得荣耀,好像建造房屋的比房屋更尊荣。


##### 来3:4
> For every house is built by someone, but God is the builder of everything.
>
> 因为房屋都必有人建造,但建造万物的就是 神。


##### 来3:5
> Moses was faithful as a servant in all God's house, testifying to what would be said in the future.
>
> 摩西为仆人,在 神的全家诚然尽忠,为要证明将来必传说的事。


##### 来3:6
> But Christ is faithful as a son over God's house. And we are his house, if we hold on to our courage and the hope of which we boast.
>
> 但基督为儿子,治理 神的家;我们若将可夸的盼望和胆量坚持到底,便是他的家了。


##### 来3:7
> So, as the Holy Spirit says: "Today, if you hear his voice,
>
> 圣灵有话说:“你们今日若听他的话,


##### 来3:8
> do not harden your hearts as you did in the rebellion, during the time of testing in the desert,
>
> 就不可硬着心,像在旷野惹他发怒、试探他的时候一样。


##### 来3:9
> where your fathers tested and tried me and for forty years saw what I did.
>
> 在那里,你们的祖宗试我探我,并且观看我的作为有四十年之久。


##### 来3:10
> That is why I was angry with that generation, and I said, 'Their hearts are always going astray, and they have not known my ways.'
>
> 所以我厌烦那世代的人,说:‘他们心里常常迷糊,竟不晓得我的作为!’


##### 来3:11
> So I declared on oath in my anger, 'They shall never enter my rest.'"
>
> 我就在怒中起誓说:‘他们断不可进入我的安息。’”


##### 来3:12
> See to it, brothers, that none of you has a sinful, unbelieving heart that turns away from the living God.
>
> 弟兄们,你们要谨慎,免得你们中间或有人存着不信的恶心,把永生 神离弃了。


##### 来3:13
> But encourage one another daily, as long as it is called Today, so that none of you may be hardened by sin's deceitfulness.
>
> 总要趁着还有今日,天天彼此相劝,免得你们中间有人被罪迷惑,心里就刚硬了。


##### 来3:14
> We have come to share in Christ if we hold firmly till the end the confidence we had at first.
>
> 我们若将起初确实的信心坚持到底,就在基督里有分了。


##### 来3:15
> As has just been said: "Today, if you hear his voice, do not harden your hearts as you did in the rebellion."
>
> 经上说:“你们今日若听他的话,就不可硬着心,像惹他发怒的日子一样。”


##### 来3:16
> Who were they who heard and rebelled? Were they not all those Moses led out of Egypt?
>
> 那时听见他话、惹他发怒的是谁呢?岂不是跟着摩西从埃及出来的众人吗?


##### 来3:17
> And with whom was he angry for forty years? Was it not with those who sinned, whose bodies fell in the desert?
>
> 神四十年之久又厌烦谁呢?岂不是那些犯罪、尸首倒在旷野的人吗?


##### 来3:18
> And to whom did God swear that they would never enter his rest if not to those who disobeyed?
>
> 又向谁起誓,不容他们进入他的安息呢?岂不是向那些不信从的人吗?


##### 来3:19
> So we see that they were not able to enter, because of their unbelief.
>
> 这样看来,他们不能进入安息是因为不信的缘故了。


## 希伯来书第4章
##### 来4:1
> Therefore, since the promise of entering his rest still stands, let us be careful that none of you be found to have fallen short of it.
>
> 我们既蒙留下有进入他安息的应许,就当畏惧,免得我们中间(“我们”原文作“你们”)或有人似乎是赶不上了。


##### 来4:2
> For we also have had the gospel preached to us, just as they did; but the message they heard was of no value to them, because those who heard did not combine it with faith.
>
> 因为有福音传给我们,像传给他们一样;只是所听见的道与他们无益,因为他们没有信心与所听见的道调和。


##### 来4:3
> Now we who have believed enter that rest, just as God has said, "So I declared on oath in my anger, 'They shall never enter my rest.'" And yet his work has been finished since the creation of the world.
>
> 但我们已经相信的人得以进入那安息,正如 神所说:“我在怒中起誓说,‘他们断不可进入我的安息!’”其实造物之工,从创世以来已经成全了。


##### 来4:4
> For somewhere he has spoken about the seventh day in these words: "And on the seventh day God rested from all his work."
>
> 论到第七日,有一处说:“到第七日, 神就歇了他一切的工。”


##### 来4:5
> And again in the passage above he says, "They shall never enter my rest."
>
> 又有一处说:“他们断不可进入我的安息!”


##### 来4:6
> It still remains that some will enter that rest, and those who formerly had the gospel preached to them did not go in, because of their disobedience.
>
> 既有必进安息的人,那先前听见福音的,因为不信从,不得进去。


##### 来4:7
> Therefore God again set a certain day, calling it Today, when a long time later he spoke through David, as was said before: "Today, if you hear his voice, do not harden your hearts."
>
> 所以过了多年,就在大卫的书上,又限定一日,如以上所引的说:“你们今日若听他的话,就不可硬着心。”


##### 来4:8
> For if Joshua had given them rest, God would not have spoken later about another day.
>
> 若是约书亚已叫他们享了安息,后来 神就不再提别的日子了。


##### 来4:9
> There remains, then, a Sabbath-rest for the people of God;
>
> 这样看来,必另有一安息日的安息,为 神的子民存留。


##### 来4:10
> for anyone who enters God's rest also rests from his own work, just as God did from his.
>
> 因为那进入安息的,乃是歇了自己的工,正如 神歇了他的工一样。


##### 来4:11
> Let us, therefore, make every effort to enter that rest, so that no one will fall by following their example of disobedience.
>
> 所以,我们务必竭力进入那安息,免得有人学那不信从的样子跌倒了。


##### 来4:12
> For the word of God is living and active. Sharper than any double-edged sword, it penetrates even to dividing soul and spirit, joints and marrow; it judges the thoughts and attitudes of the heart.
>
> 神的道是活泼的,是有功效的,比一切两刃的剑更快,甚至魂与灵、骨节与骨髓,都能刺入、剖开,连心中的思念和主意都能辨明。


##### 来4:13
> Nothing in all creation is hidden from God's sight Everything is uncovered and laid bare before the eyes of him to whom we must give account.
>
> 并且被造的没有一样在他面前不显然的;原来万物在那与我们有关系的主眼前,都是赤露敞开的。


##### 来4:14
> Therefore, since we have a great high priest who has gone through the heavens, Jesus the Son of God, let us hold firmly to the faith we profess.
>
> 我们既然有一位已经升入高天尊荣的大祭司,就是 神的儿子耶稣,便当持定所承认的道。


##### 来4:15
> For we do not have a high priest who is unable to sympathize with our weaknesses, but we have one who has been tempted in every way, just as we are--yet was without sin.
>
> 因我们的大祭司并非不能体恤我们的软弱,他也曾凡事受过试探,与我们一样;只是他没有犯罪。


##### 来4:16
> Let us then approach the throne of grace with confidence, so that we may receive mercy and find grace to help us in our time of need.
>
> 所以我们只管坦然无惧地来到施恩的宝座前,为要得怜恤,蒙恩惠,作随时的帮助。


## 希伯来书第5章
##### 来5:1
> Every high priest is selected from among men and is appointed to represent them in matters related to God, to offer gifts and sacrifices for sins.
>
> 凡从人间挑选的大祭司,是奉派替人办理属 神的事,为要献上礼物和赎罪祭(或作“要为罪献上礼物和祭物”)。


##### 来5:2
> He is able to deal gently with those who are ignorant and are going astray, since he himself is subject to weakness.
>
> 他能体谅那愚蒙的和失迷的人,因为他自己也是被软弱所困。


##### 来5:3
> This is why he has to offer sacrifices for his own sins, as well as for the sins of the people.
>
> 故此,他理当为百姓和自己献祭赎罪。


##### 来5:4
> No one takes this honor upon himself; he must be called by God, just as Aaron was.
>
> 这大祭司的尊荣,没有人自取,惟要蒙 神所召,像亚伦一样。


##### 来5:5
> So Christ also did not take upon himself the glory of becoming a high priest. But God said to him, "You are my Son; today I have become your Father."
>
> 如此,基督也不是自取荣耀作大祭司,乃是在乎向他说:“你是我的儿子,我今日生你”的那一位。


##### 来5:6
> And he says in another place, "You are a priest forever, in the order of Melchizedek."
>
> 就如经上又有一处说:“你是照着麦基洗德的等次永远为祭司。”


##### 来5:7
> During the days of Jesus' life on earth, he offered up prayers and petitions with loud cries and tears to the one who could save him from death, and he was heard because of his reverent submission.
>
> 基督在肉体的时候既大声哀哭,流泪祷告,恳求那能救他免死的主,就因他的虔诚蒙了应允。


##### 来5:8
> Although he was a son, he learned obedience from what he suffered
>
> 他虽然为儿子,还是因所受的苦难学了顺从。


##### 来5:9
> and, once made perfect, he became the source of eternal salvation for all who obey him
>
> 他既得以完全,就为凡顺从他的人成了永远得救的根源,


##### 来5:10
> and was designated by God to be high priest in the order of Melchizedek.
>
> 并蒙 神照着麦基洗德的等次称他为大祭司。


##### 来5:11
> We have much to say about this, but it is hard to explain because you are slow to learn.
>
> 论到麦基洗德,我们有好些话,并且难以解明,因为你们听不进去。


##### 来5:12
> In fact, though by this time you ought to be teachers, you need someone to teach you the elementary truths of God's word all over again. You need milk, not solid food!
>
> 看你们学习的工夫,本该作师傅,谁知还得有人将 神圣言小学的开端另教导你们,并且成了那必须吃奶、不能吃干粮的人。


##### 来5:13
> Anyone who lives on milk, being still an infant, is not acquainted with the teaching about righteousness.
>
> 凡只能吃奶的,都不熟练仁义的道理,因为他是婴孩;


##### 来5:14
> But solid food is for the mature, who by constant use have trained themselves to distinguish good from evil.
>
> 惟独长大成人的,才能吃干粮,他们的心窍习练得通达,就能分辨好歹了。


## 希伯来书第6章
##### 来6:1
> Therefore let us leave the elementary teachings about Christ and go on to maturity, not laying again the foundation of repentance from acts that lead to death, and of faith in God,
>
> 所以,我们应当离开基督道理的开端,竭力进到完全的地步,不必再立根基,就如那懊悔死行、信靠 神、


##### 来6:2
> instruction about baptisms, the laying on of hands, the resurrection of the dead, and eternal judgment.
>
> 各样洗礼,按手之礼,死人复活,以及永远审判,各等教训。


##### 来6:3
> And God permitting, we will do so.
>
> 神若许我们,我们必如此行。


##### 来6:4
> It is impossible for those who have once been enlightened, who have tasted the heavenly gift, who have shared in the Holy Spirit,
>
> 论到那些已经蒙了光照、尝过天恩的滋味、又于圣灵有分、


##### 来6:5
> who have tasted the goodness of the word of God and the powers of the coming age,
>
> 并尝过 神善道的滋味、觉悟来世权能的人,


##### 来6:6
> if they fall away, to be brought back to repentance, because to their loss they are crucifying the Son of God all over again and subjecting him to public disgrace.
>
> 若是离弃道理,就不能叫他们从新懊悔了。因为他们把 神的儿子重钉十字架,明明地羞辱他。


##### 来6:7
> Land that drinks in the rain often falling on it and that produces a crop useful to those for whom it is farmed receives the blessing of God.
>
> 就如一块田地,吃过屡次下的雨水,生长菜蔬,合乎耕种的人用,就从 神得福;


##### 来6:8
> But land that produces thorns and thistles is worthless and is in danger of being cursed. In the end it will be burned.
>
> 若长荆棘和蒺藜,必被废弃,近于咒诅,结局就是焚烧。


##### 来6:9
> Even though we speak like this, dear friends, we are confident of better things in your case--things that accompany salvation.
>
> 亲爱的弟兄们,我们虽是这样说,却深信你们的行为强过这些,而且近乎得救。


##### 来6:10
> God is not unjust; he will not forget your work and the love you have shown him as you have helped his people and continue to help them.
>
> 因为 神并非不公义,竟忘记你们所做的工和你们为他名所显的爱心,就是先前伺候圣徒,如今还是伺候。


##### 来6:11
> We want each of you to show this same diligence to the very end, in order to make your hope sure.
>
> 我们愿你们各人都显出这样的殷勤,使你们有满足的指望,一直到底。


##### 来6:12
> We do not want you to become lazy, but to imitate those who through faith and patience inherit what has been promised.
>
> 并且不懈怠,总要效法那些凭信心和忍耐承受应许的人。


##### 来6:13
> When God made his promise to Abraham, since there was no one greater for him to swear by, he swore by himself,
>
> 当初 神应许亚伯拉罕的时候,因为没有比自己更大可以指着起誓的,就指着自己起誓,说:


##### 来6:14
> saying, "I will surely bless you and give you many descendants."
>
> “论福,我必赐大福给你;论子孙,我必叫你的子孙多起来。”


##### 来6:15
> And so after waiting patiently, Abraham received what was promised.
>
> 这样,亚伯拉罕既恒久忍耐,就得了所应许的。


##### 来6:16
> Men swear by someone greater than themselves, and the oath confirms what is said and puts an end to all argument.
>
> 人都是指着比自己大的起誓,并且以起誓为实据,了结各样的争论。


##### 来6:17
> Because God wanted to make the unchanging nature of his purpose very clear to the heirs of what was promised, he confirmed it with an oath.
>
> 照样, 神愿意为那承受应许的人,格外显明他的旨意是不更改的,就起誓为证。


##### 来6:18
> God did this so that, by two unchangeable things in which it is impossible for God to lie, we who have fled to take hold of the hope offered to us may be greatly encouraged.
>
> 藉这两件不更改的事, 神决不能说谎,好叫我们这逃往避难所、持定摆在我们前头指望的人可以大得勉励。


##### 来6:19
> We have this hope as an anchor for the soul, firm and secure. It enters the inner sanctuary behind the curtain,
>
> 我们有这指望,如同灵魂的锚,又坚固、又牢靠,且通入幔内。


##### 来6:20
> where Jesus, who went before us, has entered on our behalf. He has become a high priest forever, in the order of Melchizedek.
>
> 作先锋的耶稣,既照着麦基洗德的等次成了永远的大祭司,就为我们进入幔内。


## 希伯来书第7章
##### 来7:1
> This Melchizedek was king of Salem and priest of God Most High. He met Abraham returning from the defeat of the kings and blessed him,
>
> 这麦基洗德就是撒冷王,又是至高 神的祭司,本是长远为祭司的。他当亚伯拉罕杀败诸王回来的时候,就迎接他,给他祝福。


##### 来7:2
> and Abraham gave him a tenth of everything. First, his name means "king of righteousness"; then also, "king of Salem" means "king of peace."
>
> 亚伯拉罕也将自己所得来的,取十分之一给他。他头一个名翻出来就是仁义王,他又名撒冷王,就是平安王的意思。


##### 来7:3
> Without father or mother, without genealogy, without beginning of days or end of life, like the Son of God he remains a priest forever.
>
> 他无父、无母、无族谱、无生之始、无命之终,乃是与 神的儿子相似。


##### 来7:4
> Just think how great he was: Even the patriarch Abraham gave him a tenth of the plunder!
>
> 你们想一想,先祖亚伯拉罕将自己所掳来上等之物取十分之一给他,这人是何等尊贵呢!


##### 来7:5
> Now the law requires the descendants of Levi who become priests to collect a tenth from the people--that is, their brothers--even though their brothers are descended from Abraham.
>
> 那得祭司职任的利未子孙,领命照例向百姓取十分之一,这百姓是自己的弟兄,虽是从亚伯拉罕身中生的(“身”原文作“腰”),还是照例取十分之一。


##### 来7:6
> This man, however, did not trace his descent from Levi, yet he collected a tenth from Abraham and blessed him who had the promises.
>
> 独有麦基洗德,不与他们同谱,倒收纳亚伯拉罕的十分之一,为那蒙应许的亚伯拉罕祝福。


##### 来7:7
> And without doubt the lesser person is blessed by the greater.
>
> 从来位分大的给位分小的祝福,这是驳不倒的理。


##### 来7:8
> In the one case, the tenth is collected by men who die; but in the other case, by him who is declared to be living.
>
> 在这里收十分之一的都是必死的人;但在那里收十分之一的,有为他作见证的说,他是活的。


##### 来7:9
> One might even say that Levi, who collects the tenth, paid the tenth through Abraham,
>
> 并且可说那受十分之一的利未,也是藉着亚伯拉罕纳了十分之一。


##### 来7:10
> because when Melchizedek met Abraham, Levi was still in the body of his ancestor.
>
> 因为麦基洗德迎接亚伯拉罕的时候,利未已经在他先祖的身中(“身”原文作“腰”)。


##### 来7:11
> If perfection could have been attained through the Levitical priesthood (for on the basis of it the law was given to the people), why was there still need for another priest to come--one in the order of Melchizedek, not in the order of Aaron?
>
> 从前百姓在利未人祭司职任以下受律法,倘若藉这职任能得完全,又何用另外兴起一位祭司,照麦基洗德的等次,不照亚伦的等次呢?


##### 来7:12
> For when there is a change of the priesthood, there must also be a change of the law.
>
> 祭司的职任既已更改,律法也必须更改。


##### 来7:13
> He of whom these things are said belonged to a different tribe, and no one from that tribe has ever served at the altar.
>
> 因为这话所指的人本属别的支派,那支派里从来没有一人伺候祭坛。


##### 来7:14
> For it is clear that our Lord descended from Judah, and in regard to that tribe Moses said nothing about priests.
>
> 我们的主分明是从犹大出来的,但这支派,摩西并没有提到祭司。


##### 来7:15
> And what we have said is even more clear if another priest like Melchizedek appears,
>
> 倘若照麦基洗德的样式,另外兴起一位祭司来,我的话更是显而易见的了。


##### 来7:16
> one who has become a priest not on the basis of a regulation as to his ancestry but on the basis of the power of an indestructible life.
>
> 他成为祭司,并不是照属肉体的条例,乃是照无穷之生命的大能(“无穷”原文作“不能毁坏”)。


##### 来7:17
> For it is declared: "You are a priest forever, in the order of Melchizedek."
>
> 因为有给他作见证的说:“你是照着麦基洗德的等次永远为祭司。”


##### 来7:18
> The former regulation is set aside because it was weak and useless
>
> 先前的条例因软弱无益,所以废掉了,


##### 来7:19
> (for the law made nothing perfect), and a better hope is introduced, by which we draw near to God.
>
> (律法原来一无所成)就引进了更美的指望,靠这指望,我们便可以进到 神面前。


##### 来7:20
> And it was not without an oath! Others became priests without any oath,
>
> 再者,耶稣为祭司,并不是不起誓立的。


##### 来7:21
> but he became a priest with an oath when God said to him: "The Lord has sworn and will not change his mind: 'You are a priest forever.'"
>
> 至于那些祭司,原不是起誓立的,只有耶稣是起誓立的。因为那立他的对他说:“主起了誓,决不后悔,你是永远为祭司。”


##### 来7:22
> Because of this oath, Jesus has become the guarantee of a better covenant.
>
> 既是起誓立的,耶稣就作了更美之约的中保。


##### 来7:23
> Now there have been many of those priests, since death prevented them from continuing in office;
>
> 那些成为祭司的,数目本来多,是因为有死阻隔,不能长久。


##### 来7:24
> but because Jesus lives forever, he has a permanent priesthood.
>
> 这位既是永远常存的,他祭司的职任就长久不更换。


##### 来7:25
> Therefore he is able to save completely those who come to God through him, because he always lives to intercede for them.
>
> 凡靠着他进到 神面前的人,他都能拯救到底,因为他是长远活着,替他们祈求。


##### 来7:26
> Such a high priest meets our need--one who is holy, blameless, pure, set apart from sinners, exalted above the heavens.
>
> 像这样圣洁、无邪恶、无玷污、远离罪人、高过诸天的大祭司,原是与我们合宜的。


##### 来7:27
> Unlike the other high priests, he does not need to offer sacrifices day after day, first for his own sins, and then for the sins of the people. He sacrificed for their sins once for all when he offered himself.
>
> 他不像那些大祭司,每日必须先为自己的罪,后为百姓的罪献祭,因为他只一次将自己献上,就把这事成全了。


##### 来7:28
> For the law appoints as high priests men who are weak; but the oath, which came after the law, appointed the Son, who has been made perfect forever.
>
> 律法本是立软弱的人为大祭司;但在律法以后起誓的话,是立儿子为大祭司,乃是成全到永远的。


## 希伯来书第8章
##### 来8:1
> The point of what we are saying is this: We do have such a high priest, who sat down at the right hand of the throne of the Majesty in heaven,
>
> 我们所讲的事,其中第一要紧的,就是我们有这样的大祭司,已经坐在天上至大者宝座的右边,


##### 来8:2
> and who serves in the sanctuary, the true tabernacle set up by the Lord, not by man.
>
> 在圣所,就是真帐幕里作执事;这帐幕是主所支的,不是人所支的。


##### 来8:3
> Every high priest is appointed to offer both gifts and sacrifices, and so it was necessary for this one also to have something to offer.
>
> 凡大祭司都是为献礼物和祭物设立的,所以这位大祭司也必须有所献的。


##### 来8:4
> If he were on earth, he would not be a priest, for there are already men who offer the gifts prescribed by the law.
>
> 他若在地上,必不得为祭司,因为已经有照律法献礼物的祭司。


##### 来8:5
> They serve at a sanctuary that is a copy and shadow of what is in heaven. This is why Moses was warned when he was about to build the tabernacle: "See to it that you make everything according to the pattern shown you on the mountain."
>
> 他们供奉的事,本是天上事的形状和影像,正如摩西将要造帐幕的时候,蒙 神警戒他,说,你要谨慎,作各样的物件,都要照着在山上指示你的样式。


##### 来8:6
> But the ministry Jesus has received is as superior to theirs as the covenant of which he is mediator is superior to the old one, and it is founded on better promises.
>
> 如今耶稣所得的职任是更美的,正如他作更美之约的中保;这约原是凭更美之应许立的。


##### 来8:7
> For if there had been nothing wrong with that first covenant, no place would have been sought for another.
>
> 那前约若没有瑕疵,就无处寻求后约了。


##### 来8:8
> But God found fault with the people and said: "The time is coming, declares the Lord, when I will make a new covenant with the house of Israel and with the house of Judah.
>
> 所以主指责他的百姓说(或作“所以主指前约的缺欠说”):“日子将到,我要与以色列家和犹大家另立新约,


##### 来8:9
> It will not be like the covenant I made with their forefathers when I took them by the hand to lead them out of Egypt, because they did not remain faithful to my covenant, and I turned away from them, declares the Lord.
>
> 不像我拉着他们祖宗的手,领他们出埃及的时候,与他们所立的约。因为他们不恒心守我的约,我也不理他们。这是主说的。”


##### 来8:10
> This is the covenant I will make with the house of Israel after that time, declares the Lord. I will put my laws in their minds and write them on their hearts. I will be their God, and they will be my people.
>
> 主又说:“那些日子以后,我与以色列家所立的约乃是这样:我要将我的律法放在他们里面,写在他们心上;我要作他们的 神,他们要作我的子民。


##### 来8:11
> No longer will a man teach his neighbor, or a man his brother, saying, 'Know the Lord,' because they will all know me, from the least of them to the greatest.
>
> 他们不用各人教导自己的乡邻和自己的弟兄说:‘你该认识主,’因为他们从最小的到至大的,都必认识我。


##### 来8:12
> For I will forgive their wickedness and will remember their sins no more."
>
> 我要宽恕他们的不义,不再记念他们的罪愆。”


##### 来8:13
> By calling this covenant "new," he has made the first one obsolete; and what is obsolete and aging will soon disappear.
>
> 既说新约,就以前约为旧了;但那渐旧渐衰的,就必快归无有了。


## 希伯来书第9章
##### 来9:1
> Now the first covenant had regulations for worship and also an earthly sanctuary.
>
> 原来前约有礼拜的条例和属世界的圣幕。


##### 来9:2
> A tabernacle was set up. In its first room were the lampstand, the table and the consecrated bread; this was called the Holy Place.
>
> 因为有预备的帐幕,头一层叫作圣所,里面有灯台、桌子和陈设饼。


##### 来9:3
> Behind the second curtain was a room called the Most Holy Place,
>
> 第二幔子后又有一层帐幕,叫作至圣所,


##### 来9:4
> which had the golden altar of incense and the gold-covered ark of the covenant. This ark contained the gold jar of manna, Aaron's staff that had budded, and the stone tablets of the covenant.
>
> 有金香炉(“炉”或作“坛”),有包金的约柜,柜里有盛吗哪的金罐和亚伦发过芽的杖并两块约版。


##### 来9:5
> Above the ark were the cherubim of the Glory, overshadowing the atonement cover. But we cannot discuss these things in detail now.
>
> 柜上面有荣耀基路伯的影罩着施恩座(“施恩”原文作“蔽罪”)。这几件我现在不能一一细说。


##### 来9:6
> When everything had been arranged like this, the priests entered regularly into the outer room to carry on their ministry.
>
> 这些物件既如此预备齐了,众祭司就常进头一层帐幕,行拜 神的礼。


##### 来9:7
> But only the high priest entered the inner room, and that only once a year, and never without blood, which he offered for himself and for the sins the people had committed in ignorance.
>
> 至于第二层帐幕,惟有大祭司一年一次独自进去,没有不带着血为自己和百姓的过错献上。


##### 来9:8
> The Holy Spirit was showing by this that the way into the Most Holy Place had not yet been disclosed as long as the first tabernacle was still standing.
>
> 圣灵用此指明,头一层帐幕仍存的时候,进入至圣所的路还未显明。


##### 来9:9
> This is an illustration for the present time, indicating that the gifts and sacrifices being offered were not able to clear the conscience of the worshiper.
>
> 那头一层帐幕作现今的一个表样,所献的礼物和祭物,就着良心说,都不能叫礼拜的人得以完全。


##### 来9:10
> They are only a matter of food and drink and various ceremonial washings--external regulations applying until the time of the new order.
>
> 这些事,连那饮食和诸般洗濯的规矩,都不过是属肉体的条例,命定到振兴的时候为止。


##### 来9:11
> When Christ came as high priest of the good things that are already here, he went through the greater and more perfect tabernacle that is not man-made, that is to say, not a part of this creation.
>
> 但现在基督已经来到,作了将来美事的大祭司,经过那更大、更全备的帐幕,不是人手所造,也不是属乎这世界的。


##### 来9:12
> He did not enter by means of the blood of goats and calves; but he entered the Most Holy Place once for all by his own blood, having obtained eternal redemption.
>
> 并且不用山羊和牛犊的血,乃用自己的血,只一次进入圣所,成了永远赎罪的事。


##### 来9:13
> The blood of goats and bulls and the ashes of a heifer sprinkled on those who are ceremonially unclean sanctify them so that they are outwardly clean.
>
> 若山羊和公牛的血,并母牛犊的灰,洒在不洁的人身上,尚且叫人成圣,身体洁净,


##### 来9:14
> How much more, then, will the blood of Christ, who through the eternal Spirit offered himself unblemished to God, cleanse our consciences from acts that lead to death, so that we may serve the living God!
>
> 何况基督藉着永远的灵,将自己无瑕无疵献给 神,他的血岂不更能洗净你们的心(原文作“良心”),除去你们的死行,使你们侍奉那永生 神吗?


##### 来9:15
> For this reason Christ is the mediator of a new covenant, that those who are called may receive the promised eternal inheritance--now that he has died as a ransom to set them free from the sins committed under the first covenant.
>
> 为此,他作了新约的中保,既然受死赎了人在前约之时所犯的罪过,便叫蒙召之人得着所应许永远的产业。


##### 来9:16
> In the case of a will, it is necessary to prove the death of the one who made it,
>
> 凡有遗命,必须等到留遗命的人死了(“遗命”原文与“约”字同)。


##### 来9:17
> because a will is in force only when somebody has died; it never takes effect while the one who made it is living.
>
> 因为人死了,遗命才有效力;若留遗命的尚在,那遗命还有用处吗?


##### 来9:18
> This is why even the first covenant was not put into effect without blood.
>
> 所以,前约也不是不用血立的。


##### 来9:19
> When Moses had proclaimed every commandment of the law to all the people, he took the blood of calves, together with water, scarlet wool and branches of hyssop, and sprinkled the scroll and all the people.
>
> 因为摩西当日照着律法将各样诫命传给众百姓,就拿朱红色绒和牛膝草,把牛犊、山羊的血和水洒在书上,又洒在众百姓身上,说:


##### 来9:20
> He said, "This is the blood of the covenant, which God has commanded you to keep."
>
> “这血就是 神与你们立约的凭据。”


##### 来9:21
> In the same way, he sprinkled with the blood both the tabernacle and everything used in its ceremonies.
>
> 他又照样把血洒在帐幕和各样器皿上。


##### 来9:22
> In fact, the law requires that nearly everything be cleansed with blood, and without the shedding of blood there is no forgiveness.
>
> 按着律法,凡物差不多都是用血洁净的,若不流血,罪就不得赦免了。


##### 来9:23
> It was necessary, then, for the copies of the heavenly things to be purified with these sacrifices, but the heavenly things themselves with better sacrifices than these.
>
> 照着天上样式做的物件,必须用这些祭物去洁净;但那天上的本物自然当用更美的祭物去洁净。


##### 来9:24
> For Christ did not enter a man-made sanctuary that was only a copy of the true one; he entered heaven itself, now to appear for us in God's presence.
>
> 因为基督并不是进了人手所造的圣所(这不过是真圣所的影像),乃是进了天堂,如今为我们显在 神面前;


##### 来9:25
> Nor did he enter heaven to offer himself again and again, the way the high priest enters the Most Holy Place every year with blood that is not his own.
>
> 也不是多次将自己献上,像那大祭司每年带着牛羊的血进入圣所(“牛羊的血”原文作“不是自己的血”)。


##### 来9:26
> Then Christ would have had to suffer many times since the creation of the world. But now he has appeared once for all at the end of the ages to do away with sin by the sacrifice of himself.
>
> 如果这样,他从创世以来,就必多次受苦了。但如今在这末世显现一次,把自己献为祭,好除掉罪。


##### 来9:27
> Just as man is destined to die once, and after that to face judgment,
>
> 按着定命,人人都有一死,死后且有审判。


##### 来9:28
> so Christ was sacrificed once to take away the sins of many people; and he will appear a second time, not to bear sin, but to bring salvation to those who are waiting for him.
>
> 象这样,基督既然一次被献,担当了多人的罪,将来要向那等候他的人第二次显现,并与罪无关,乃是为拯救他们。


## 希伯来书第10章
##### 来10:1
> The law is only a shadow of the good things that are coming--not the realities themselves. For this reason it can never, by the same sacrifices repeated endlessly year after year, make perfect those who draw near to worship.
>
> 律法既是将来美事的影儿,不是本物的真像,总不能藉着每年常献一样的祭物,叫那近前来的人得以完全。


##### 来10:2
> If it could, would they not have stopped being offered? For the worshipers would have been cleansed once for all, and would no longer have felt guilty for their sins.
>
> 若不然,献祭的事岂不早已止住了吗?因为礼拜的人,良心既被洁净,就不再觉得有罪了。


##### 来10:3
> But those sacrifices are an annual reminder of sins,
>
> 但这些祭物是叫人每年想起罪来,


##### 来10:4
> because it is impossible for the blood of bulls and goats to take away sins.
>
> 因为公牛和山羊的血断不能除罪。


##### 来10:5
> Therefore, when Christ came into the world, he said: "Sacrifice and offering you did not desire, but a body you prepared for me;
>
> 所以,基督到世上来的时候,就说:“ 神啊,祭物和礼物是你不愿意的;你曾给我预备了身体。


##### 来10:6
> with burnt offerings and sin offerings you were not pleased.
>
> 燔祭和赎罪祭是你不喜欢的。


##### 来10:7
> Then I said, 'Here I am--it is written about me in the scroll--I have come to do your will, O God.'"
>
> 那时我说:‘ 神啊!我来了,为要照你的旨意行;我的事在经卷上已经记载了。’”


##### 来10:8
> First he said, "Sacrifices and offerings, burnt offerings and sin offerings you did not desire, nor were you pleased with them" (although the law required them to be made).
>
> 以上说:“祭物和礼物,燔祭和赎罪祭,是你不愿意的,也是你不喜欢的”(这都是按着律法献的)。


##### 来10:9
> Then he said, "Here I am, I have come to do your will." He sets aside the first to establish the second.
>
> 后又说:“我来了为要照你的旨意行。”可见他是除去在先的,为要立定在后的。


##### 来10:10
> And by that will, we have been made holy through the sacrifice of the body of Jesus Christ once for all.
>
> 我们凭这旨意,靠耶稣基督只一次献上他的身体,就得以成圣。


##### 来10:11
> Day after day every priest stands and performs his religious duties; again and again he offers the same sacrifices, which can never take away sins.
>
> 凡祭司天天站着事奉 神,屡次献上一样的祭物,这祭物永不能除罪。


##### 来10:12
> But when this priest had offered for all time one sacrifice for sins, he sat down at the right hand of God.
>
> 但基督献了一次永远的赎罪祭,就在 神的右边坐下了。


##### 来10:13
> Since that time he waits for his enemies to be made his footstool,
>
> 从此等候他仇敌成了他的脚凳。


##### 来10:14
> because by one sacrifice he has made perfect forever those who are being made holy.
>
> 因为他一次献祭,便叫那得以成圣的人永远完全。


##### 来10:15
> The Holy Spirit also testifies to us about this. First he says:
>
> 圣灵也对我们作见证,因为他既已说过:


##### 来10:16
> "This is the covenant I will make with them after that time, says the Lord. I will put my laws in their hearts, and I will write them on their minds."
>
> “主说,那些日子以后,我与他们所立的约乃是这样:我要将我的律法写在他们心上,又要放在他们的里面。”


##### 来10:17
> Then he adds: "Their sins and lawless acts I will remember no more."
>
> 以后就说:“我不再记念他们的罪愆和他们的过犯。”


##### 来10:18
> And where these have been forgiven, there is no longer any sacrifice for sin.
>
> 这些罪过既已赦免,就不用再为罪献祭了。


##### 来10:19
> Therefore, brothers, since we have confidence to enter the Most Holy Place by the blood of Jesus,
>
> 弟兄们,我们既因耶稣的血,得以坦然进入至圣所,


##### 来10:20
> by a new and living way opened for us through the curtain, that is, his body,
>
> 是藉着他给我们开了一条又新又活的路,从幔子经过,这幔子就是他的身体。


##### 来10:21
> and since we have a great priest over the house of God,
>
> 又有一位大祭司治理 神的家,


##### 来10:22
> let us draw near to God with a sincere heart in full assurance of faith, having our hearts sprinkled to cleanse us from a guilty conscience and having our bodies washed with pure water.
>
> 并我们心中天良的亏欠已经洒去,身体用清水洗净了,就当存着诚心和充足的信心来到 神面前;


##### 来10:23
> Let us hold unswervingly to the hope we profess, for he who promised is faithful.
>
> 也要坚守我们所承认的指望,不至摇动,因为那应许我们的是信实的。


##### 来10:24
> And let us consider how we may spur one another on toward love and good deeds.
>
> 又要彼此相顾,激发爱心,勉励行善。


##### 来10:25
> Let us not give up meeting together, as some are in the habit of doing, but let us encourage one another--and all the more as you see the Day approaching.
>
> 你们不可停止聚会,好像那些停止惯了的人,倒要彼此劝勉。既知道(原文作“看见”)那日子临近,就更当如此。


##### 来10:26
> If we deliberately keep on sinning after we have received the knowledge of the truth, no sacrifice for sins is left,
>
> 因为我们得知真道以后,若故意犯罪,赎罪的祭就再没有了,


##### 来10:27
> but only a fearful expectation of judgment and of raging fire that will consume the enemies of God.
>
> 惟有战惧等候审判和那烧灭众敌人的烈火。


##### 来10:28
> Anyone who rejected the law of Moses died without mercy on the testimony of two or three witnesses.
>
> 人干犯摩西的律法,凭两三个见证人尚且不得怜恤而死;


##### 来10:29
> How much more severely do you think a man deserves to be punished who has trampled the Son of God under foot, who has treated as an unholy thing the blood of the covenant that sanctified him, and who has insulted the Spirit of grace?
>
> 何况人践踏 神的儿子,将那使他成圣之约的血当作平常,又亵慢施恩的圣灵,你们想,他要受的刑罚该怎样加重呢?


##### 来10:30
> For we know him who said, "It is mine to avenge; I will repay," and again, "The Lord will judge his people."
>
> 因为我们知道谁说:“伸冤在我,我必报应。”又说:“主要审判他的百姓。”


##### 来10:31
> It is a dreadful thing to fall into the hands of the living God.
>
> 落在永生 神的手里,真是可怕的!


##### 来10:32
> Remember those earlier days after you had received the light, when you stood your ground in a great contest in the face of suffering.
>
> 你们要追念往日,蒙了光照以后,所忍受大争战的各样苦难。


##### 来10:33
> Sometimes you were publicly exposed to insult and persecution; at other times you stood side by side with those who were so treated.
>
> 一面被毁谤,遭患难,成了戏景,叫众人观看;一面陪伴那些受这样苦难的人。


##### 来10:34
> You sympathized with those in prison and joyfully accepted the confiscation of your property, because you knew that you yourselves had better and lasting possessions.
>
> 因为你们体恤了那些被捆锁的人,并且你们的家业被人抢去,也甘心忍受,知道自己有更美长存的家业。


##### 来10:35
> So do not throw away your confidence; it will be richly rewarded.
>
> 所以,你们不可丢弃勇敢的心,存这样的心必得大赏赐。


##### 来10:36
> You need to persevere so that when you have done the will of God, you will receive what he has promised.
>
> 你们必须忍耐,使你们行完了 神的旨意,就可以得着所应许的。


##### 来10:37
> For in just a very little while, "He who is coming will come and will not delay.
>
> 因为还有一点点时候,“那要来的就来,并不迟延。


##### 来10:38
> But my righteous one will live by faith. And if he shrinks back, I will not be pleased with him."
>
> 只是义人必因信得生(“义人”有古卷作“我的义人”);他若退后,我心里就不喜欢他。”


##### 来10:39
> But we are not of those who shrink back and are destroyed, but of those who believe and are saved.
>
> 我们却不是退后入沉沦的那等人,乃是有信心以致灵魂得救的人。


## 希伯来书第11章
##### 来11:1
> Now faith is being sure of what we hope for and certain of what we do not see.
>
> 信就是所望之事的实底,是未见之事的确据。


##### 来11:2
> This is what the ancients were commended for.
>
> 古人在这信上得了美好的证据。


##### 来11:3
> By faith we understand that the universe was formed at God's command, so that what is seen was not made out of what was visible.
>
> 我们因着信,就知道诸世界是藉 神 话造成的,这样,所看见的,并不是从显然之物造出来的。


##### 来11:4
> By faith Abel offered God a better sacrifice than Cain did. By faith he was commended as a righteous man, when God spoke well of his offerings. And by faith he still speaks, even though he is dead.
>
> 亚伯因着信,献祭与 神,比该隐所献的更美,因此便得了称义的见证,就是 神指他礼物作的见证。他虽然死了,却因这信,仍旧说话。


##### 来11:5
> By faith Enoch was taken from this life, so that he did not experience death; he could not be found, because God had taken him away. For before he was taken, he was commended as one who pleased God.
>
> 以诺因着信,被接去,不至于见死,人也找不着他,因为 神已经把他接去了。只是他被接去以先,已经得了 神喜悦他的明证。


##### 来11:6
> And without faith it is impossible to please God, because anyone who comes to him must believe that he exists and that he rewards those who earnestly seek him.
>
> 人非有信,就不能得 神的喜悦;因为到 神面前来的人,必须信有 神,且信他赏赐那寻求他的人。


##### 来11:7
> By faith Noah, when warned about things not yet seen, in holy fear built an ark to save his family. By his faith he condemned the world and became heir of the righteousness that comes by faith.
>
> 挪亚因着信,既蒙 神指示他未见的事,动了敬畏的心,预备了一只方舟,使他全家得救。因此就定了那世代的罪,自己也承受了那从信而来的义。


##### 来11:8
> By faith Abraham, when called to go to a place he would later receive as his inheritance, obeyed and went, even though he did not know where he was going.
>
> 亚伯拉罕因着信,蒙召的时候,就遵命出去,往将来要得为业的地方去;出去的时候,还不知往哪里去。


##### 来11:9
> By faith he made his home in the promised land like a stranger in a foreign country; he lived in tents, as did Isaac and Jacob, who were heirs with him of the same promise.
>
> 他因着信,就在所应许之地作客,好像在异地居住帐棚,与那同蒙一个应许的以撒、雅各一样。


##### 来11:10
> For he was looking forward to the city with foundations, whose architect and builder is God.
>
> 因为他等候那座有根基的城,就是 神所经营、所建造的。


##### 来11:11
> By faith Abraham, even though he was past age--and Sarah herself was barren--was enabled to become a father because he considered him faithful who had made the promise.
>
> 因着信,连撒拉自己,虽然过了生育的岁数,还能怀孕,因她以为那应许她的是可信的。


##### 来11:12
> And so from this one man, and he as good as dead, came descendants as numerous as the stars in the sky and as countless as the sand on the seashore.
>
> 所以从一个仿佛已死的人就生出子孙,如同天上的星那样众多,海边的沙那样无数。


##### 来11:13
> All these people were still living by faith when they died. They did not receive the things promised; they only saw them and welcomed them from a distance. And they admitted that they were aliens and strangers on earth.
>
> 这些人都是存着信心死的,并没有得着所应许的,却从远处望见,且欢喜迎接,又承认自己在世上是客旅,是寄居的。


##### 来11:14
> People who say such things show that they are looking for a country of their own.
>
> 说这样话的人,是表明自己要找一个家乡。


##### 来11:15
> If they had been thinking of the country they had left, they would have had opportunity to return.
>
> 他们若想念所离开的家乡,还有可以回去的机会。


##### 来11:16
> Instead, they were longing for a better country--a heavenly one. Therefore God is not ashamed to be called their God, for he has prepared a city for them.
>
> 他们却羡慕一个更美的家乡,就是在天上的。所以 神被称为他们的 神,并不以为耻,因为他已经给他们预备了一座城。


##### 来11:17
> By faith Abraham, when God tested him, offered Isaac as a sacrifice. He who had received the promises was about to sacrifice his one and only son,
>
> 亚伯拉罕因着信,被试验的时候,就把以撒献上;这便是那欢喜领受应许的,将自己独生的儿子献上。


##### 来11:18
> even though God had said to him, "It is through Isaac that your offspring will be reckoned."
>
> 论到这儿子,曾有话说:“从以撒生的才要称为你的后裔。”


##### 来11:19
> Abraham reasoned that God could raise the dead, and figuratively speaking, he did receive Isaac back from death.
>
> 他以为 神还能叫人从死里复活,他也仿佛从死中得回他的儿子来。


##### 来11:20
> By faith Isaac blessed Jacob and Esau in regard to their future.
>
> 以撒因着信,就指着将来的事给雅各、以扫祝福。


##### 来11:21
> By faith Jacob, when he was dying, blessed each of Joseph's sons, and worshiped as he leaned on the top of his staff.
>
> 雅各因着信,临死的时候,给约瑟的两个儿子各自祝福,扶着杖头敬拜 神。


##### 来11:22
> By faith Joseph, when his end was near, spoke about the exodus of the Israelites from Egypt and gave instructions about his bones.
>
> 约瑟因着信,临终的时候,提到以色列族将来要出埃及,并为自己的骸骨留下遗命。


##### 来11:23
> By faith Moses' parents hid him for three months after he was born, because they saw he was no ordinary child, and they were not afraid of the king's edict.
>
> 摩西生下来,他的父母见他是个俊美的孩子,就因着信,把他藏了三个月,并不怕王命。


##### 来11:24
> By faith Moses, when he had grown up, refused to be known as the son of Pharaoh's daughter.
>
> 摩西因着信,长大了就不肯称为法老女儿之子。


##### 来11:25
> He chose to be mistreated along with the people of God rather than to enjoy the pleasures of sin for a short time.
>
> 他宁可和 神的百姓同受苦害,也不愿暂时享受罪中之乐。


##### 来11:26
> He regarded disgrace for the sake of Christ as of greater value than the treasures of Egypt, because he was looking ahead to his reward.
>
> 他看为基督受的凌辱比埃及的财物更宝贵,因他想望所要得的赏赐。


##### 来11:27
> By faith he left Egypt, not fearing the king's anger; he persevered because he saw him who is invisible.
>
> 他因着信,就离开埃及,不怕王怒;因为他恒心忍耐,如同看见那不能看见的主。


##### 来11:28
> By faith he kept the Passover and the sprinkling of blood, so that the destroyer of the firstborn would not touch the firstborn of Israel.
>
> 他因着信,就守逾越节(“守”或作“立”),行洒血的礼,免得那灭长子的临近以色列人。


##### 来11:29
> By faith the people passed through the Red Sea as on dry land; but when the Egyptians tried to do so, they were drowned.
>
> 他们因着信,过红海如行干地;埃及人试着要过去,就被吞灭了。


##### 来11:30
> By faith the walls of Jericho fell, after the people had marched around them for seven days.
>
> 以色列人因着信,围绕耶利哥城七日,城墙就倒塌了。


##### 来11:31
> By faith the prostitute Rahab, because she welcomed the spies, was not killed with those who were disobedient.
>
> 妓女喇合因着信,曾和和平平地接待探子,就不与那些不顺从的人一同灭亡。


##### 来11:32
> And what more shall I say? I do not have time to tell about Gideon, Barak, Samson, Jephthah, David, Samuel and the prophets,
>
> 我又何必再说呢?若要一一细说,基甸、巴拉、参孙、耶弗他、大卫、撒母耳和众先知的事,时候就不够了。


##### 来11:33
> who through faith conquered kingdoms, administered justice, and gained what was promised; who shut the mouths of lions,
>
> 他们因着信,制伏了敌国,行了公义,得了应许;堵了狮子的口,


##### 来11:34
> quenched the fury of the flames, and escaped the edge of the sword; whose weakness was turned to strength; and who became powerful in battle and routed foreign armies.
>
> 灭了烈火的猛势,脱了刀剑的锋刃;软弱变为刚强,争战显出勇敢,打退外邦的全军。


##### 来11:35
> Women received back their dead, raised to life again. Others were tortured and refused to be released, so that they might gain a better resurrection.
>
> 有妇人得自己的死人复活,又有人忍受严刑,不肯苟且得释放(“释放”原文作“赎”),为要得着更美的复活。


##### 来11:36
> Some faced jeers and flogging, while still others were chained and put in prison.
>
> 又有人忍受戏弄、鞭打、捆锁、监禁各等的磨炼,


##### 来11:37
> They were stoned; they were sawed in two; they were put to death by the sword. They went about in sheepskins and goatskins, destitute, persecuted and mistreated--
>
> 被石头打死,被锯锯死,受试探,被刀杀,披着绵羊、山羊的皮各处奔跑,受穷乏、患难、苦害,


##### 来11:38
> the world was not worthy of them. They wandered in deserts and mountains, and in caves and holes in the ground.
>
> 在旷野、山岭、山洞、地穴飘流无定,本是世界不配有的人。


##### 来11:39
> These were all commended for their faith, yet none of them received what had been promised.
>
> 这些人都是因信得了美好的证据,却仍未得着所应许的。


##### 来11:40
> God had planned something better for us so that only together with us would they be made perfect.
>
> 因为 神给我们预备了更美的事,叫他们若不与我们同得,就不能完全。


## 希伯来书第12章
##### 来12:1
> Therefore, since we are surrounded by such a great cloud of witnesses, let us throw off everything that hinders and the sin that so easily entangles, and let us run with perseverance the race marked out for us.
>
> 我们既有这许多的见证人,如同云彩围着我们,就当放下各样的重担,脱去容易缠累我们的罪,存心忍耐,奔那摆在我们前头的路程,


##### 来12:2
> Let us fix our eyes on Jesus, the author and perfecter of our faith, who for the joy set before him endured the cross, scorning its shame, and sat down at the right hand of the throne of God.
>
> 仰望为我们信心创始成终的耶稣(或作“仰望那将真道创始成终的耶稣”)。他因那摆在前面的喜乐,就轻看羞辱,忍受了十字架的苦难,便坐在 神宝座的右边。


##### 来12:3
> Consider him who endured such opposition from sinful men, so that you will not grow weary and lose heart.
>
> 那忍受罪人这样顶撞的,你们要思想,免得疲倦灰心。


##### 来12:4
> In your struggle against sin, you have not yet resisted to the point of shedding your blood.
>
> 你们与罪恶相争,还没有抵挡到流血的地步。


##### 来12:5
> And you have forgotten that word of encouragement that addresses you as sons: "My son, do not make light of the Lord's discipline, and do not lose heart when he rebukes you,
>
> 你们又忘了那劝你们如同劝儿子的话,说:“我儿,你不可轻看主的管教,被他责备的时候,也不可灰心。


##### 来12:6
> because the Lord disciplines those he loves, and he punishes everyone he accepts as a son."
>
> 因为主所爱的,他必管教,又鞭打凡所收纳的儿子。”


##### 来12:7
> Endure hardship as discipline; God is treating you as sons. For what son is not disciplined by his father?
>
> 你们所忍受的,是 神管教你们,待你们如同待儿子。焉有儿子不被父亲管教的呢?


##### 来12:8
> If you are not disciplined (and everyone undergoes discipline), then you are illegitimate children and not true sons.
>
> 管教原是众子所共受的,你们若不受管教,就是私子,不是儿子了。


##### 来12:9
> Moreover, we have all had human fathers who disciplined us and we respected them for it. How much more should we submit to the Father of our spirits and live!
>
> 再者,我们曾有生身的父管教我们,我们尚且敬重他;何况万灵的父,我们岂不更当顺服他得生吗?


##### 来12:10
> Our fathers disciplined us for a little while as they thought best; but God disciplines us for our good, that we may share in his holiness.
>
> 生身的父都是暂随己意管教我们;惟有万灵的父管教我们,是要我们得益处,使我们在他的圣洁上有分。


##### 来12:11
> No discipline seems pleasant at the time, but painful. Later on, however, it produces a harvest of righteousness and peace for those who have been trained by it.
>
> 凡管教的事,当时不觉得快乐,反觉得愁苦,后来却为那经练过的人结出平安的果子,就是义。


##### 来12:12
> Therefore, strengthen your feeble arms and weak knees.
>
> 所以,你们要把下垂的手、发酸的腿挺起来,


##### 来12:13
> "Make level paths for your feet," so that the lame may not be disabled, but rather healed.
>
> 也要为自己的脚,把道路修直了,使瘸子不至歪脚,反得痊愈(“歪脚”或作“差路”)


##### 来12:14
> Make every effort to live in peace with all men and to be holy; without holiness no one will see the Lord.
>
> 你们要追求与众人和睦,并要追求圣洁;非圣洁没有人能见主。


##### 来12:15
> See to it that no one misses the grace of God and that no bitter root grows up to cause trouble and defile many.
>
> 又要谨慎,恐怕有人失了 神的恩;恐怕有毒根生出来扰乱你们,因此叫众人沾染污秽。


##### 来12:16
> See that no one is sexually immoral, or is godless like Esau, who for a single meal sold his inheritance rights as the oldest son.
>
> 恐怕有淫乱的,有贪恋世俗如以扫的;他因一点食物把自己长子的名分卖了。


##### 来12:17
> Afterward, as you know, when he wanted to inherit this blessing, he was rejected. He could bring about no change of mind, though he sought the blessing with tears.
>
> 后来想要承受父所祝的福,竟被弃绝,虽然号哭切求,却得不着门路使他父亲的心意回转。这是你们知道的。


##### 来12:18
> You have not come to a mountain that can be touched and that is burning with fire; to darkness, gloom and storm;
>
> 你们原不是来到那能摸的山;此山有火焰、密云、黑暗、暴风、


##### 来12:19
> to a trumpet blast or to such a voice speaking words that those who heard it begged that no further word be spoken to them,
>
> 角声与说话的声音。那些听见这声音的,都求不要再向他们说话。


##### 来12:20
> because they could not bear what was commanded: "If even an animal touches the mountain, it must be stoned."
>
> 因为他们当不起所命他们的话,说:“靠近这山的,即便是走兽,也要用石头打死。”


##### 来12:21
> The sight was so terrifying that Moses said, "I am trembling with fear."
>
> 所见的极其可怕,甚至摩西说:“我甚是恐惧战兢。”


##### 来12:22
> But you have come to Mount Zion, to the heavenly Jerusalem, the city of the living God. You have come to thousands upon thousands of angels in joyful assembly,
>
> 你们乃是来到锡安山,永生 神的城邑,就是天上的耶路撒冷。那里有千万的天使,


##### 来12:23
> to the church of the firstborn, whose names are written in heaven. You have come to God, the judge of all men, to the spirits of righteous men made perfect,
>
> 有名录在天上诸长子之会所共聚的总会,有审判众人的 神和被成全之义人的灵魂,


##### 来12:24
> to Jesus the mediator of a new covenant, and to the sprinkled blood that speaks a better word than the blood of Abel.
>
> 并新约的中保耶稣,以及所洒的血。这血所说的比亚伯的血所说的更美。


##### 来12:25
> See to it that you do not refuse him who speaks. If they did not escape when they refused him who warned them on earth, how much less will we, if we turn away from him who warns us from heaven?
>
> 你们总要谨慎,不可弃绝那向你们说话的。因为那些弃绝在地上警戒他们的,尚且不能逃罪,何况我们违背那从天上警戒我们的呢?


##### 来12:26
> At that time his voice shook the earth, but now he has promised, "Once more I will shake not only the earth but also the heavens."
>
> 当时他的声音震动了地,但如今他应许说:“再一次我不单要震动地,还要震动天。”


##### 来12:27
> The words "once more" indicate the removing of what can be shaken--that is, created things--so that what cannot be shaken may remain.
>
> 这再一次的话,是指明被震动的,就是受造之物都要挪去,使那不被震动的常存。


##### 来12:28
> Therefore, since we are receiving a kingdom that cannot be shaken, let us be thankful, and so worship God acceptably with reverence and awe,
>
> 所以,我们既得了不能震动的国,就当感恩,照 神所喜悦的,用虔诚、敬畏的心事奉 神。


##### 来12:29
> for our "God is a consuming fire."
>
> 因为我们的 神乃是烈火。


## 希伯来书第13章
##### 来13:1
> Keep on loving each other as brothers.
>
> 你们务要常存弟兄相爱的心。


##### 来13:2
> Do not forget to entertain strangers, for by so doing some people have entertained angels without knowing it.
>
> 不可忘记用爱心接待客旅,因为曾有接待客旅的,不知不觉就接待了天使。


##### 来13:3
> Remember those in prison as if you were their fellow prisoners, and those who are mistreated as if you yourselves were suffering.
>
> 你们要记念被捆绑的人,好像与他们同受捆绑;也要记念遭苦害的人,想到自己也在肉身之内。


##### 来13:4
> Marriage should be honored by all, and the marriage bed kept pure, for God will judge the adulterer and all the sexually immoral.
>
> 婚姻,人人都当尊重,床也不可污秽,因为苟合行淫的人, 神必要审判。


##### 来13:5
> Keep your lives free from the love of money and be content with what you have, because God has said, "Never will I leave you; never will I forsake you."
>
> 你们存心不可贪爱钱财,要以自己所有的为足。因为主曾说:“我总不撇下你,也不丢弃你。”


##### 来13:6
> So we say with confidence, "The Lord is my helper; I will not be afraid. What can man do to me?"
>
> 所以我们可以放胆说:“主是帮助我的,我必不惧怕;人能把我怎么样呢?”


##### 来13:7
> Remember your leaders, who spoke the word of God to you. Consider the outcome of their way of life and imitate their faith.
>
> 从前引导你们、传 神之道给你们的人,你们要想念他们,效法他们的信心,留心看他们为人的结局。


##### 来13:8
> Jesus Christ is the same yesterday and today and forever.
>
> 耶稣基督,昨日、今日、一直到永远是一样的。


##### 来13:9
> Do not be carried away by all kinds of strange teachings. It is good for our hearts to be strengthened by grace, not by ceremonial foods, which are of no value to those who eat them.
>
> 你们不要被那诸般怪异的教训勾引了去。因为人心靠恩得坚固才是好的,并不是靠饮食;那在饮食上专心的,从来没有得着益处。


##### 来13:10
> We have an altar from which those who minister at the tabernacle have no right to eat.
>
> 我们有一祭坛,上面的祭物是那些在帐幕中供职的人不可同吃的。


##### 来13:11
> The high priest carries the blood of animals into the Most Holy Place as a sin offering, but the bodies are burned outside the camp.
>
> 原来牲畜的血被大祭司带入圣所作赎罪祭,牲畜的身子被烧在营外。


##### 来13:12
> And so Jesus also suffered outside the city gate to make the people holy through his own blood.
>
> 所以耶稣要用自己的血叫百姓成圣,也就在城门外受苦。


##### 来13:13
> Let us, then, go to him outside the camp, bearing the disgrace he bore.
>
> 这样,我们也当出到营外,就了他去,忍受他所受的凌辱。


##### 来13:14
> For here we do not have an enduring city, but we are looking for the city that is to come.
>
> 我们在这里本没有常存的城,乃是寻求那将来的城。


##### 来13:15
> Through Jesus, therefore, let us continually offer to God a sacrifice of praise--the fruit of lips that confess his name.
>
> 我们应当靠着耶稣,常常以颂赞为祭献给 神,这就是那承认主名之人嘴唇的果子。


##### 来13:16
> And do not forget to do good and to share with others, for with such sacrifices God is pleased.
>
> 只是不可忘记行善和捐输的事,因为这样的祭是 神所喜悦的。


##### 来13:17
> Obey your leaders and submit to their authority. They keep watch over you as men who must give an account. Obey them so that their work will be a joy, not a burden, for that would be of no advantage to you.
>
> 你们要依从那些引导你们的,且要顺服,因他们为你们的灵魂时刻警醒,好像那将来交账的人。你们要使他们交的时候有快乐,不至忧愁;若忧愁就与你们无益了。


##### 来13:18
> Pray for us. We are sure that we have a clear conscience and desire to live honorably in every way.
>
> 请你们为我们祷告,因我们自觉良心无亏,愿意凡事按正道而行。


##### 来13:19
> I particularly urge you to pray so that I may be restored to you soon.
>
> 我更求你们为我祷告,使我快些回到你们那里去。


##### 来13:20
> May the God of peace, who through the blood of the eternal covenant brought back from the dead our Lord Jesus, that great Shepherd of the sheep,
>
> 但愿赐平安的 神,就是那凭永约之血使群羊的大牧人我主耶稣,从死里复活的 神,


##### 来13:21
> equip you with everything good for doing his will, and may he work in us what is pleasing to him, through Jesus Christ, to whom be glory for ever and ever. Amen.
>
> 在各样善事上成全你们,叫你们遵行他的旨意;又藉着耶稣基督在你们心里行他所喜悦的事。愿荣耀归给他,直到永永远远。阿们!


##### 来13:22
> Brothers, I urge you to bear with my word of exhortation, for I have written you only a short letter.
>
> 弟兄们,我略略写信给你们,望你们听我劝勉的话。


##### 来13:23
> I want you to know that our brother Timothy has been released. If he arrives soon, I will come with him to see you.
>
> 你们该知道,我们的兄弟提摩太已经释放了。他若快来,我必同他去见你们。


##### 来13:24
> Greet all your leaders and all God's people. Those from Italy send you their greetings.
>
> 请你们问引导你们的诸位和众圣徒安。从义大利来的人也问你们安。


##### 来13:25
> Grace be with you all.
>
> 愿恩惠常与你们众人同在。阿们。

